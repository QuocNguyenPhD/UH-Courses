% initial settings
\documentclass[12pt]{exam}
\usepackage{geometry}
\usepackage{graphicx}
\usepackage{enumitem}
\usepackage[usenames,dvipsnames]{xcolor}
\usepackage[backend=biber, style=alphabetic]{biblatex}
\usepackage{url,hyperref}

\usepackage{amsmath} % math symbols, matrices, cases, trig functions,
% var-greek symbols.
% var-greek symbols.
\usepackage{amsfonts} % mathbb, mathfrak, large sum and product symbols.
\usepackage{amssymb} % extended list of math symbols from AMS.
% https://ctan.math.washington.edu/tex-archive/fonts/amsfonts/doc/amssymb.pdf
\usepackage{amsthm} % theorem styling.
\usepackage{mathrsfs} % mathscr fonts.
\usepackage{yhmath} % widehat.
\usepackage{empheq} % emphasize equations, extending 'amsmath' and 'mathtools'.
\usepackage{bm} % simplified bold math. Do \bm{math-equations-here}

% geometry of paper
\geometry{
  a4paper, % 'a4paper', 'c5paper', 'letterpaper', 'legalpaper'
  asymmetric, % don't swap margins in left and right pages. as
  % opposed to 'twoside'
  centering, % to center the content between margins
  bindingoffset=0cm,
}

% hyprlink settings
\hypersetup{
  colorlinks = true,
  linkcolor = {red!60!black},
  anchorcolor = red,
  citecolor = {green!50!black},
  urlcolor = magenta,
}

% theorem styles
\theoremstyle{plain} % default; italic text, extra space above and below
\newtheorem{theorem}{Theorem}[section]
\newtheorem{proposition}{Proposition}[section]
\newtheorem{lemma}{Lemma}[section]
\newtheorem{corollary}{Corollary}[theorem]

\theoremstyle{definition} % upright text, extra space above and below
\newtheorem{definition}{Definition}[section]
\newtheorem{example}{Example}[section]

\theoremstyle{remark} % upright text, no extra space above or below
\newtheorem{remark}{Remark}[section]
\newtheorem*{note}{Note} %'Notes' in italics and without counter

% renewcommands for counters
\newcommand{\propositionautorefname}{Proposition}
\newcommand{\definitionautorefname}{Definition}
\newcommand{\lemmaautorefname}{Lemma}
\newcommand{\remarkautorefname}{Remark}
\newcommand{\exampleautorefname}{Example}

\addbibresource{articles.bib}

\begin{document}

\title{MATH6302 - Modern Algebra \\ Homework 8}

% author list
\author{
  Joel Sleeba \\
}

\maketitle
\printanswers
\unframedsolutions

\begin{questions}
  \question
  \begin{solution}
    \begin{parts}
      \part Let $x \in R$, be nilpotent, $(x^n = 0)$ and $R$ be
      commutative. If $x \neq 0$, then $x^{n-1}$ is not zero (assuming
      $n$ to be the smallest $n \in \mathbb{N}$ such that $x^n = 0$)
      but $x^{n-1} x = x^n = 0$, which shows that $x$ is a zero divisor.
      \part By the commutativity of $R$, we get \[
        (rx)^n = r^nx^n = r^n0 = 0
      \]
      which shows that $rx$ is nilpotent.
      \part We claim that $(1+x)^{-1} = 1 - x + x^2 - x^3 \ldots
      x^{n-1}$. To see this, notice that \[
        (1+x)(1 - x + x^2 - x^3 \ldots x^{n-1}) = (1 - x + x^2 - x^3
        \ldots x^{n-1}) + x(1 - x + x^2 - x^3 \ldots x^{n-1}) = 1
      \]
      \part Let $a \in R$ be a unit and $x \in R$ be nilpotent with
      $x^n = 0$. Then $a^{-1}x$ is again nilpotent. Then, by the
      previous part, we see that $(1 + a^{-1}x)$ is a unit. Thus $a(1
      + a^{-1}x) = a + x$ is a unit.
    \end{parts}
  \end{solution}

  \question
  \begin{solution}
    Let $a \in R$. then $a + a = (a + a)^2 = a^2 + a^2 + a^2 + a^2 =
    a+ a + a+ a$ forces $a + a = 0$ i.e $a = -a$ for all $a \in R$.
    Therefore showing $ab = ba$ is equivalent to showing $ab + ba =
    0$ for any $a, b \in R$.
    \begin{align*}
      a + b = (a + b)^2 &= a^2 + ab + ba + b^2 \\
      &= a + b + ab + ba
    \end{align*}
    gives $ab + ba = 0$ and hence we're done.
  \end{solution}

  \question
  \begin{solution}
    \begin{parts}
      \part We have already verified in the first assignment that
      $(\mathcal{P}, \Delta)$ is an Abelian group.
      Notice that since $A \cap B \subset X$ for each $A, B \in
      \mathcal{P}(X)$, $\cap$ is a binary operation.
      Associativity of $\cap$ follows since $(A \cap B) \cap C = A
      \cap B \cap C = A
      \cap (B \cap C)$. Moreover $A \cap B = B \cap A$. Hence we just need
      to verify that $\cap$ distributes over $\Delta$.
      \begin{align*}
        (A \Delta B) \cap C &= ((A \setminus B)  \cup (B \setminus A)) \cap C \\
        &= ((A \setminus B) \cap C) \ \cup \ ((B \setminus A) \cap C) \\
        &= ((A \cap C) \setminus (B \cap C)) \ \cup \ ((B \cap C)
        \ (A \cap C)) \\
        &= (A \cap C) \Delta (B \cap C)
      \end{align*}
      Hence $(\mathcal{P}, \Delta, \cap)$ is a commutative ring.
      \part Since we've already shows that $\cap$ is commutative,
      we'll just verify the rest. Notice that for any $A \in
      \mathcal{P}(X)$, we have $A \cap X = A = X \cap A$, hence $X$
      acts as the multiplicative identity making the ring unital.
      Moreover $A \cap A = A$ shows that it is a Boolean ring.
    \end{parts}
  \end{solution}

  \question
  \begin{solution}
    \begin{parts}
      \part The zero polynomial $\textbf{0}$, which is the additive
      identity will not be in the
      collection. Therefore it won't be a subring, hence not an ideal.

      \part Consider $3x^2 + 1$ in the collection and $x^2 \in
      \mathbb{Z}[x]$. Then $  x^2(3x^2 + 1) = 3x^4 + x^2$ is not in
      the collection. Hence it is not an ideal.

      \part Since any sum and product of such polynomials will have
      their constant term, and coefficients of $x, x^2$ be $0$, the
      collection is a subring. Moreover if $p(x) \neq \textbf{0}$ is
      in the collection, then $p(x) = x^3q(x)$ for $q(x) \in
      \mathbb{Z}[x]$. Then for any $r(x) \in \mathbb{Z}[x]$, $(rp)(x) =
      x^3q(x)r(x)$, is again in the collection. Hence the collection
      is an ideal.

      \part Let $x^2 \in \mathbb{Z}[x^2]$. Then for $x \in
      \mathbb{Z}[x]$, $x\cdot x^2 = x^3 \not\in \mathbb{Z}[x^2]$,
      shows that $\mathbb{Z}[x^2]$ is not an ideal.

      \part It is easy to verify that the collection given is a
      subroup of $\mathbb{Z}[x]$. The closure of the product on the
      collection will be evident once we verify the ideal condition.

      Let $p(x) = \sum_{i = 0}^{n} a_i x^i$, be a polynomial
      with $\sum_{i = 0}^{n} a_i = 0$ and $q(x) = \sum_{j = 0}^{m}
      b_jx^j$ be another polynomial in $\mathbb{Z}[x]$.
      Then
      \begin{align*}
        (qp)(x) = \sum_{j = 0}^{m} \sum_{i = 0}^{n} b_j a_i x^{i+ j}
      \end{align*}
      Then the sum of their co-efficients,
      \begin{align*}
        \sum_{j = 0}^{m} \sum_{i = 0}^{n} b_j a_i = \sum_{j = 0}^{m}
        b_j \Bigg(\sum_{i = 0}^{n} a_n \Bigg) = 0
      \end{align*}
      shows that the collection is an ideal and hence proves the
      closure under multiplication too.

      \part Let $p(x) = x^2 + 1$ and $q(x) = x$. Then $p^\prime(0) =
      2\times 0= 0$. But $(pq)(x) = x^3 + x$ and $(pq)^\prime(x) =
      3x^2 + 1$ gives $(pq)^\prime(0) = 1$. Hence the collection is
      not an ideal.
    \end{parts}
  \end{solution}

  \question
  \begin{solution}
    Consider the map $\phi: \mathbb{C} \to M_2(\mathbb{R})$ defined as
    \begin{align*}
      \phi(a + ib) =
      \begin{pmatrix}%{c c}
        a & b \\
        -b & a
      \end{pmatrix}
    \end{align*}
    The fact that $\phi$ preserves addition follows easily from the
    matrix addition in $M_2(\mathbb{R})$. Hence we'll only verify the
    multiplicativity of the map.
    \begin{align*}
      \phi((a + ib)(p + iq)) &= \phi((ap - bq) + i(aq + bp))  \\
      &=
      \begin{pmatrix}%{c c}
        ap - bq & aq + bp \\
        - aq - bp & ap - bq
      \end{pmatrix} \\
      &=
      \begin{pmatrix}%{c c}
        a & b \\
        -b & a
      \end{pmatrix}
      \begin{pmatrix}%{c c}
        p & q \\
        -q & p
      \end{pmatrix} \\
      & = \phi(a + ib) \phi(p + iq)
    \end{align*}
    shows that $\phi$ is a ring homomorphism. Moreover we see that
    $\phi(a + ib) = \textbf{0}$ if and only if $a = b = 0$. Hence
    $\phi$ is an injective ring homomorphism, which proves our assertion.
  \end{solution}

  \question
  \begin{solution}
    \begin{parts}
      \part Let $\phi: \mathbb{Z} \to R$ be the map given. Then
      $$\phi(m + n) = (m+n)\textbf{1} = m \textbf{1} + n \textbf{1} =
      \phi(m) + \phi(n)$$
      and
      \begin{align*}
        \phi(mn) = mn \textbf{1} = (mn)(\textbf{1}\cdot \textbf{1}) = (m
        \textbf{1})\cdot(n \textbf{1}) = \phi(m)\cdot\phi(n)
      \end{align*}
      shows that $\phi$ is a ring homomorphism.

      We'll show that $\textrm{Ker}(\phi) = n \mathbb{Z}$, where $n$
      is the characteristic of $R$. Let $nk \in n \mathbb{Z}$, then
      $$\phi(nk) = (nk) \textbf{1} = (kn)\textbf{1} = k(n \textbf{1})
      = k \textbf{0} = 0$$
      Conversely if $k \in \textrm{Ker}(\phi)$, then
      \begin{align*}
        \phi(k) = k \textbf{1} = 0
      \end{align*}
      which forces $k$ to be a multiple of $n$. Thus we get
      $\textrm{Ker}(\phi) = n \mathbb{Z}$.

      \part $\mathbb{Q}$ has characteristic $0$, since $1 + 1 +
      \ldots 1 \neq 0$. For the same reason $\mathbb{Z}[x]$ also has
      characteristic $0$. But $\underbrace{1 + 1 + \ldots 1}_{n
      \textrm{ times}} = 0$ in $\mathbb{Z}/n \mathbb{Z} [x]$. Hence
      $\mathbb{Z}/n \mathbb{Z} [x]$ has characteristic $n$.
      \part Let $R$ be a ring of characteristic $p$, then for any $a
      \in R$, $pa= \underbrace{a + a + \ldots a}_{p \textrm{ times}}
      =  a\underbrace{1 + 1 + \ldots 1}_{p \textrm{ times}} = a0 =
      0$. Moreover, when $R$ is a commutative ring,
      \begin{align*}
        (a + b)^p = \sum_{r = 0}^{p} \frac{p!}{(p-r)!r!}a^{p-r}b^r =
        a^p + \sum_{r = 1}^{p-1} \frac{p!}{(p-r)!r!}a^{p-r}b^r + b^p
      \end{align*}
      Since $p$ is a prime, $\frac{p!}{(p-r)!r!}$ is a
      multiple of $p$ whenever $1 \le r \le p -1$. This is because all the
      numbers being multiplied together in the denominator is less
      than $p$ and cannot factor out $p$. Hence we get that $(a +
      b)^p = a^p + b^p$.
    \end{parts}
  \end{solution}

  \question
  \begin{solution}
    Assume that $R$ is an integral domain with characteristic $p \neq
    0$. Then $p \textbf{1} = 0$ for the multiplicative identity
    $\textbf{1} \in R$. If $p$ was not a prime, then
    $p = nk$ for $1 < n, k < p$. Then we'd get
    \begin{align*}
      p \textbf{1} = (n \textbf{1})(k \textbf{1}) = 0
    \end{align*}
    Since $R$ is an integral domain this would force $n \textbf{1} =
    0$ or $k \textbf{1} = 0$, which contradicts our assumption on the
    characteristic of $R$ since $n, k < p$. Hence we see that $p$
    must be a prime.
  \end{solution}

  \question
  \begin{solution}
    Let $I$ be the collection of all nilpotent elements of a
    commutative ring $R$. Let $a, b \in I$ with $a^n = b^m = 0$. Then
    \begin{align*}
      (a + b)^{m+n} = \sum_{i = 0}^{m+n}
      \begin{pmatrix}%{c}
        m+n -i \\
        i
      \end{pmatrix} a^{m+n -i} b^i = \sum_{ i = 0}^{m+n} 0 = 0
    \end{align*}
    shows that $a + b \in I$. If $r \in R$, then
    \begin{align*}
      (ar)^n = a^nr^n = 0
    \end{align*}
    shows that $ar \in I$ for all $ r \in R$, hence proving that $I$
    is an ideal.
  \end{solution}

  \question
  \begin{solution}
    \begin{parts}
      \part Since $I, J$ are subrings of $R$, being the ideals of
      $R$, we see that $I, J \subset I  + J$ ($I = I + e_J$ and $J =
      e_I + J$). If $i + j , p + q \in I + J$, then $(i + j) + (p +
      q) = (i + p) + (j + q) \in I + J$.
      Moreover if $r \in R$ and $i + j \in I + J$, then
      \begin{align*}
        r(i + j) = ri + rj \in I + J
      \end{align*}
      and
      \begin{align*}
        (i + j) r = ir + jr \in I + J
      \end{align*}
      shows that $I + J$ is an ideal which contains $I, J$. Now if
      $K$ is any other ideal that contain $I, J$, then being a
      subring, $K \ni i + j $ for all $i \in I , j \in J$. Hence $K
      \supset I + J$, which shows that $I + J$ is the smallest ideal
      that contain $I, J$.

      \part Let $\sum_{i = 1}^{n} a_ib_i, \sum_{j = 1}^{m} p_jq_j \in
      IJ$, where $a_i, p_j \in I$ and $b_i, q_j \in J$. Then
      \begin{align*}
        \sum_{i = 1}^{n} a_ib_i + \sum_{j = 1}^{m} p_jq_j \in IJ
      \end{align*}
      by the definition of $IJ$.
      Moreover if $r \in R$, then
      \begin{align*}
        r\sum_{i = 1}^{n} a_ib_i = \sum_{i = 1}^{n} (ra_i)b_i \in IJ
      \end{align*}
      and
      \begin{align*}
        \Big(\sum_{i = 1}^{n} a_ib_i \Big)r= \sum_{i = 1}^{n} a_i(b_ir) \in IJ
      \end{align*}
      since $I, J$ are ideals in $R$. Hence we see that $IJ$ is an
      ideal of $R$.

      Also for any $a_i \in I, b_i \in J$, $a_i b_i \in I \cap J$
      since $I, J$ are ideals. Thus we see that $IJ \subset I \cap J$.

      \part Let $I, J = (2) \subset \mathbb{Z}$. Then $I \cap J =
      (2)$. We claim that $IJ = (4)$. Since $4 = 2 \times 2$, we see
      that $4 \in IJ$. Thus $(4) \subset  IJ$.

      Conversely if $ij \in
      IJ$, then $i = 2k, j = 2m$ for $m, k \in \mathbb{Z}$. Thus $ij
      = 4mk \in (4)$. Thus all finite sums of elements $ij$ where $i
      \in I, j \in J$ are also in $(4)$. Thus we see that $IJ = (4)
      \neq (2) = I \cap J$.

      \part Let $R$ be unital, and commutative with $I + J = R$, and let
      $r \in I \cap J$.
    \end{parts}
  \end{solution}

  \question
  \begin{solution}
    Let $a, b \in \mathbb{R}$ such that $ab = 0 \in P$. Without loss
    of generality assume $a \in P$. Since we know that $P$
    contains no zero divisors, this forces $a = 0$. Hence we see that
    $R$ is an integral domain.
  \end{solution}

  \question
  \begin{solution}
    Let $I \cap J \subset P$ and $I \not \subset P$. Then $\exists
    i_p \in I\setminus P$. Now for any $j \in J$,
    \begin{align*}
      i_pj \in IJ \subset I \cap J \subset P
    \end{align*}
    forces $j \in P$, by the primality of $P$. Hence $J \subset P$.
  \end{solution}

  \question
  \begin{solution}
    Consider $x^2 + 4 \in I$. Since $x^2 + 4$ cannot be factorized in
    $\mathbb{Z}[x]$, it cannot be written as $ab$ for any $a, b \in
    I$. Moreover $x^2 = x\cdot x, 4 = 2\cdot2 \in I^2$. Hence we see
    that $x^2 + 4 \in I^2$.
  \end{solution}

  \question
  \begin{solution}
    Let $I = (x)$, be the principal ideal generated by $x \in
    R[[x]]$. Let $p = \sum_{i = 0}^{n} a_i x^i, q = \sum_{j =
    0}^{m} b_jx^j \in R[[x]]$. Then
    \begin{align*}
      pq = \sum_{i = 0}^{n} \sum_{j = 0}^{m} a_ib_j x^{i + j} =
      \sum_{k = 0}^{m+n} c_k x^k
    \end{align*}
    where
    \begin{align*}
      c_k = \sum_{i = 0}^{k} a_{i}b_{k-i}
    \end{align*}
    We notice that $pq \in I$ if and only if $c_0 = a_0b_0 = 0$.
    Since $R$ is an integral domain, this forces either $a_0$ or
    $b_0$ to be zero. Without loss of generality, assume $a_0 = 0$. Then
    \begin{align*}
      p = \sum_{i = 1}^{n} a_i x^i = x \Big( \sum_{i = 0}^{n-1} a_i
      x^i \Big) \in I
    \end{align*}
    Thus we see that $I$ is a prime ideal.

    Now assume that $I$ is a maximal ideal. Then $R[[x]]/I$ must be a
    field. We'll show that $R[[x]]/I \cong R$. Consider the map
    \begin{align*}
      \phi : R[[x]] \to R : p \to p(0)
    \end{align*}
    where $0$ is the additive identity of $R$. Then by way addition
    and multiplication is defined in $R[[x]]$, we see that $\phi$ is
    a ring homomorphism. Moreover if $p = \sum_{i = 1}^{n} a_i x^i
    \in I$, then $p(0) = 0$ shows that $I \subset
    \textrm{Ker}(\phi)$. Maximality of $I$ forces $I =
    \textrm{Ker}(\phi)$, since $\phi$ is a non trivial ring
    homomorphism. Then by the first isomorphism therom, we get
    $R[[x]]/I \cong R$. Hence we see that $R$ is a field.
  \end{solution}

  \question
  \begin{solution}
    Let $P$ be a prime ideal of a finite unital commutative ring $R$.
    Then consider $R/P$, the collection of all additive cosets of
    $P$. Let $r + P \in R/P$. Since $R/P$ is finite $r^n + P = (r +
    P)^n = (r + P)^m =  r^m + P$ for some $n > m \in \mathbb{N}$.
    This forces $r^n - r^m = r^m(r^{n-m} - 1) \in P$. Now there can
    be two choices, either $r^m \in P$ or $r^{m-n} -1 \in P$.

    For the former case if $r^m \in P$, since $r^m = rr^{m-1} \in P$,
    by an induction argument, we see that $r \in P$. Then $r + P = P$
    is the zero element in $R/P$.

    In the latter case, if $r^{n-m} - 1 \in P$, we get $r^{n-m} + P =
    1 + P$, and thus
    \begin{align*}
      (r + P)(r + P)^{n - m - 1} &= r^{n - m} + P \\
      &= 1 + P \\
      &= r^{n - m} + P \\
      &= (r + P)^{n - m - 1}(r + P)
    \end{align*}
  \end{solution}
  Which shows that $r + P$ is invertible. Since $r + P$ was
  arbitrary, we have shown that every non-zero element of $R/P$ is
  invertible making $R/P$ a field. Thus $P$ is maximal.
\end{questions}
\printbibliography[heading=bibintoc]
\end{document}



% initial settings
\documentclass[12pt]{exam}
\usepackage{geometry}
\usepackage{graphicx}
\usepackage{enumitem}
\usepackage[usenames,dvipsnames]{xcolor}
\usepackage[backend=biber, style=alphabetic]{biblatex}
\usepackage{url,hyperref}

\usepackage{amsmath} % math symbols, matrices, cases, trig functions, var-greek symbols.
\usepackage{amsfonts} % mathbb, mathfrak, large sum and product symbols.
\usepackage{amssymb} % extended list of math symbols from AMS. https://ctan.math.washington.edu/tex-archive/fonts/amsfonts/doc/amssymb.pdf
\usepackage{amsthm} % theorem styling.
\usepackage{mathrsfs} % mathscr fonts.
\usepackage{yhmath} % widehat.
\usepackage{multicol} % multiple colums
\usepackage{empheq} % emphasize equations, extending 'amsmath' and 'mathtools'.
\usepackage{bm} % simplified bold math. Do \bm{math-equations-here}

% geometry of paper
\geometry{
  a4paper, % 'a4paper', 'c5paper', 'letterpaper', 'legalpaper'
  asymmetric, % don't swap margins in left and right pages. as opposed to 'twoside'
  centering, % to center the content between margins
  bindingoffset=0cm,
}

% hyprlink settings
\hypersetup{
  colorlinks = true,
  linkcolor = {red!60!black},
  anchorcolor = red,
  citecolor = {green!50!black},
  urlcolor = magenta,
}

% theorem styles
\theoremstyle{plain} % default; italic text, extra space above and below
\newtheorem{theorem}{Theorem}[section]
\newtheorem{proposition}{Proposition}[section]
\newtheorem{lemma}{Lemma}[section]
\newtheorem{corollary}{Corollary}[theorem]

\theoremstyle{definition} % upright text, extra space above and below
\newtheorem{definition}{Definition}[section]
\newtheorem{example}{Example}[section]

\theoremstyle{remark} % upright text, no extra space above or below
\newtheorem{remark}{Remark}[section]
\newtheorem*{note}{Note} %'Notes' in italics and without counter

% renewcommands for counters
\newcommand{\propositionautorefname}{Proposition}
\newcommand{\definitionautorefname}{Definition}
\newcommand{\lemmaautorefname}{Lemma}
\newcommand{\remarkautorefname}{Remark}
\newcommand{\exampleautorefname}{Example}

\addbibresource{articles.bib}

\begin{document}

\title{MATH6320 - Modern Algebra  \\ Homework 5}

% author list
\author{
  Joel Sleeba \\
}

\maketitle
\printanswers
\unframedsolutions

\begin{questions}

  \question
  Prove that if $H$ is a normal subgroup of prime index $p$, then for all $K \leqslant G$ either $K \leqslant H$ or $G = HK$ with $|K:K\cap H| = p$
  \begin{solution}
    Since $H$ is normal in $G$, we see that $N_G(H) = G$, and therefore $K \leqslant N_G(H)$. Now using the second isomorphism theorem we get
    \begin{enumerate}[label=(\roman*)]
      \item $HK$ is a subgroup of $G$
      \item $H \trianglelefteq HK$
      \item $K \cap H \trianglelefteq K$
      \item $HK/H \cong K/K \cap H$
    \end{enumerate}
    Assume that $K$ is not a subgroup of $H$. Then $H$ is a proper normal subgroup of $HK$. Since the cosets of $H$ in $HK$ are also cosets in $G$ we have $HK/K \subset G/K$. By the normality of $H$ in $HK$, we get that $HK/H$ is algebraically closed under the group operation. Since $G/H$ is a finite group, by the subgroup criterion for finite groups $HK/H$ must be a subgroup of $G/H$. Moreover $|G/H| = |G:H| = p$ is prime, and $|HK/H|>1$ (since $H$ is a proper subgroup of $HK$). Then Lagrange's theorem forces $HK/H = G/H$.

    Now if $HK$ is a proper subgroup of $G$, then there is $g \in G$ such that $g \notin HK$. This forces $gH \in G/H$ but $gH \notin HK/H$, which is a contradiction. Hence $HK = G$.

    By (iv) from above, we get that $G/H \cong K/K\cap H$. Then $p = |G:H| = |G/H| = |K/K\cap H| = |K:K\cap H|$
  \end{solution}

  \question
  Let $C$ be a normal subgroup of the group $A$ and let $D$ be a normal subgroup of the group $B$. Prove that $(C \times D) \trianglelefteq (A \times B)$ and $(A \times B)/(C \times D) \cong (A/C)\times(B/D)$.
  \begin{solution}
    First we'll show that $(C \times D) \trianglelefteq (A \times B)$. For this let $(a, b) \in (A \times B)$ and $(c, d) \in C \times D$. Then $(a, b)(c, d)(a, b)^{-1} = (a, b)(c, d)(a^{-1}, b^{-1}) = (aca^{-1}, bdb^{-1}) \in C \times D$ since $C \trianglelefteq A, D \trianglelefteq B$. Hence we get, $(C \times D) \trianglelefteq (A \times B)$.

    Now consider the map $\phi : A \times B \to (A/C) \times (B/D) := (a, b) \to (aC, bD)$. Clearly $\phi$ is surjective, since if $(aC, bD) \in (A/C)\times(B/D)$, then $\phi(a, b) = ( aC, bD)$. Also, it is a group homomorphism. If $(p, q), (r, s) \in A \times B$, then $\phi(pr, qs) = (prC, qsD) = (pC, qD)(rC, sD) = \phi(p, q)\phi(r, s)$.

    Moreover, the $\textrm{Ker}(\phi) = C \times D$, since $\phi(a, b) = (aC, bD) = \textbf{0} \in (A/C) \times (B/D)$ if and only if $aC = C$ and $bD = D$, which is equivalent to $a \in C$ and $d \in D$. Hence by the first isomorphism theorem, we get $(A \times B)/(C \times D) \cong (A/C)\times(B/D)$.
  \end{solution}

  \question
  Let $M, N$ be normal subgroups of $G$ such that $G = MN$. Prove that $G/(M\cap N) \cong (G/M) \times (G/N)$.
  \begin{solution}
    Consider the map $\phi: G \to (G/M)\times(G/N) := g \to (gM, gN)$. We'll show that this map is a surjective group homomorphism with $\textrm{Ker}(\phi) = M \cap N$. Then first isomorphism theorem will give us our required result.

    First to show that it is a group homomorphism, let $ g, h \in G$. Then $\phi(gh) = (ghM, ghN) = (gM, gN)(hM, hN) = \phi(g) \phi(h)$. Hence $\phi$ is a group homomorphism.

    Now to show the surjectivity of $\phi$, let $(gM, hN) \in (G/M)\times(G/N)$. Since $MN = G$ is a group, we see that $MN = G = NM$. Hence there exist $m_g, m^\prime_g, m_h, m^\prime_h \in M$ and $n_g, n^\prime_g, n_h, n^\prime_h \in N$ such that $g = m_gn_g = n^\prime_g m^\prime_g$ and $h = m_hn_h = n^\prime_h m^\prime_h$.
    Then $$gM = n^\prime_g m^\prime_g M  = n^\prime_gM = n^\prime_g m_h M$$
    and $$hN = m_hn_hN = m_hN = n^\prime_g (m_h N )n^{\prime-1}_g= n^\prime_gm_hN$$
    where $m_hN = n^\prime_g (m_h N )n^{\prime-1}_g$ follows from the fact that cosets of a normal subgroup are stable under the conjugation action.
    Therefore we see that $ \phi(n^\prime_gm_h) = (n^\prime_gm_hM, n^\prime_gm_hN) = (gM, hN)$ and hence $\phi$ is surjective.

    Now it only remains to show that $\textrm{Ker}(\phi) = M \cap N$. $\phi(g) = (gM, gN) =  \textbf{0} = (M, N)$ if and only if $g \in M$ and $g \in N$. Hence we see that $\phi(g) = \textbf{0}$ if and only if $g \in M \cap N$. Hence $\textrm{Ker}(\phi) = M \cap N$.
  \end{solution}

  \question
  Let $p$ be a prime and let $G$ be a group of $p$-power roots of 1 in $\mathbb{C}$. Prove that the map $z \to z^p$ is a surjective homomorphism. Deduce that $G$ is isomorphic to a proper quotient of itself.
  \begin{solution}
    Note that the group $G = \{ z \in \mathbb{C}  \ : \   z^{p^n} = 1 \textrm{ for some } n \in \mathbb{N} \}$ and $\phi: G \to G$ is the map with $\phi(z) = z^p$. First we'll show that the map is well defined. Let $z \in G$ with $z^{p^n} = 1$. Then $\phi(z)^{p^{n-1}} = (z^{p})^{p^{n-1}} = z^{p^n} = 1$. Hence $\phi(z) \in G$.

    Now to see that $\phi$ is a homomorphism, let $z, w \in G$. Then $\phi(zw) = (zw)^p  = z^p w^p = \phi(z) \phi(w)$. Hence $\phi$ is a homomorphism.

    To prove the surjectivity of $\phi$, let $z = e^{  i \theta} \in G$ with $z^{p^n} = 1$. This is equivalent to $p^n \theta= 2m\pi$ for some $m \in \mathbb{N}$. Let $w = e^{i \theta/p}$. Since $w^{p^{n+1}} = (e^{i \theta/p})^{p^{n+1}} = e^{i \theta p^{n}} = e^{ 2 m \pi} = 1$, we see that $ w \in G$. Moreover $\phi(w) = z$. Hence $ \phi$ is a surjective homomorphism.

    Now by the first isomorphism theorem, we see that $ G \cong G/\textrm{Ker}(\phi)$.
  \end{solution}

  \question
  Let $p$ be a prime and let $G$ be a group of order $p^{\alpha}m$ where $p \not| m$. Assume $P$ is a subgroup of $G$ of order $p^\alpha$ and $ N$ be a normal subgroup of order $p^{ \beta} n$, where $p \not | n$. Prove that $|P \cap N| = p^\beta$ and $|PN/N| = p^{\alpha -\beta}$.
  \begin{solution}
    Note that since $P \cap N \leqslant P, N$, we have $|P\cap N|\big|p^{\alpha}$ and $|P \cap N| \big| p^{\beta}n$. Since $\beta \le \alpha$ and $(p^{\alpha}, p^{\beta}n) = p^{\beta}$, we get $|P\cap N| \big| p^{\beta}$. Let $|P \cap N | = p^k$ where $k \le \beta$. Since $ N$ is a normal subgroup of $G$, we know that $PN$ is a subgroup of $G$ with $|PN| = \frac{|P||N|}{|P \cap N|} = p^{\alpha + \beta - k}n$. Again since $PN$ is a subgroup of $G$, by Lagrange's theorem, we get $p^{\alpha + \beta - k}n \big| p^{\alpha}n$. Therefore,
    \begin{align*}
      \alpha + \beta - k &\le \alpha \\
      \beta - k & \le 0 \\
      \beta \le k
    \end{align*}
    Hence $k = \beta$, and we get $|P \cap N| = p^{\beta}$.

    Now since $N \trianglelefteq G$ and $P \leqslant N_G(N) = G$, by the second isomorphism theorem $PN/N \cong P/(P \cap N)$. Therefore $|PN/N| = |P/(P \cap N)| = \frac{|P|}{|P \cap N|} = \frac{p^{\alpha}}{p^{\beta}} = p^{\alpha - \beta}$.
  \end{solution}

  \question
  Give the list of invariant factors for all Abelian groups of order 105, 44100.
  \begin{solution}
    \begin{parts}
      \part $105 = 21\times5 = 3\times5\times7$. If $Z_{n_1}\times Z_{n_2} \times \ldots \times Z_{n_k}$ is an invariant factor decomposition of the group of order $105$, we know that $n_{i+1}|n_i$. But since the powers of each of $2, 3, 5$ in the prime factorization of $105$, is just 1, there is no way other than $Z_{105}$ to represent an Abelian group of order 105.
      \part $44100 = 2^2 \times 3^3 \times 5^2 \times 7^2$. Therefore the possible invariant factor decompositions of a group of order $44100$ are as
      \begin{multicols}{4}
        \begin{itemize}
          \item  $Z_{44100}$
          \item $Z_{22050} \times Z_{2}$
          \item $Z_{14700} \times Z_{3}$
          \item $Z_{8820} \times Z_5$
          \item $Z_{6300} \times Z_7$
          \item $Z_{7350} \times Z_6$
          \item $Z_{4410} \times Z_{10}$
          \item $Z_{3150} \times Z_{14}$
          \item $Z_{2920} \times Z_{15}$
          \item $Z_{2100} \times Z_{21}$
          \item $Z_{1050} \times Z_{42}$
          \item $Z_{1260} \times Z_{35}$
          \item $Z_{1470} \times Z_{30}$
          \item $Z_{630}\times Z_{70}$
          \item $Z_{420}\times Z_{105}$
          \item $Z_{210} \times Z_{210}$
        \end{itemize}
      \end{multicols}
    \end{parts}
  \end{solution}

  \question
  Give the list of elementary divisors for all Abelian groups of order 105, 44100
  \begin{solution}
    \begin{parts}
      \part $105 = 3 \times 5 \times 7$. If $A_1 \times A_2 \times \ldots \times A_k$ is the elementary divisor decomposition of the group of order $105$, then each $|A_1| = 3, |A_2| = 5$ and $|A_3| = 7$. Since every group of prime power is cyclic, we get that the only possible decomposition is $Z_3 \times Z_5 \times Z_7$.
      \part $44100 = 2^23^25^27^2$. Hence if $A_2 \times A_3 \times A_5 \times A_7$ is the elementary divisor decomposition, then $|A_1| = 2^2, |A_{2}| = 3^2, |A_{3}| = 5^2$ and $|A_4| = 7^2$. Again each $A_p$ can be isomorphic to either $Z_{p^2}$ or $Z_p \times Z_p$. So the complete list of elementary divisors for all Abelian groups of order $44100$ is
      \begin{itemize}
        \item  $Z_4 \times Z_9 \times Z_{25} \times Z_{49}$
        \item  $( Z_2 \times Z_2) \times Z_9 \times Z_{25} \times Z_{49}$
        \item  $Z_4 \times ( Z_3 \times Z_3) \times Z_{25} \times Z_{49}$
        \item  $Z_4 \times Z_9 \times (Z_5 \times Z_5) \times Z_{49}$
        \item  $Z_4 \times Z_9 \times Z_{25} \times (Z_{7} \times Z_7)$
        \item  $Z_4 \times Z_9 \times (Z_{5} \times Z_5) \times (Z_{7} \times Z_7)$
        \item  $Z_4 \times (Z_3\times Z_3) \times Z_{25} \times (Z_{7} \times Z_7)$
        \item  $(Z_2 \times Z_2) \times Z_9 \times Z_{25} \times (Z_{7} \times Z_7)$
        \item  $(Z_2 \times Z_2) \times Z_9  \times (Z_{5} \times Z_5) \times Z_{49}$
        \item  $(Z_2 \times Z_2) \times (Z_{3} \times Z_3)\times Z_{25} \times Z_{49}$
        \item  $Z_4 \times (Z_{3} \times Z_3)\times ( Z_{5} \times Z_5)\times Z_{49}$
        \item  $(Z_2 \times Z_2) \times (Z_{3} \times Z_3)\times (Z_{5} \times Z_{5} )\times Z_{49}$
        \item  $(Z_2 \times Z_2) \times (Z_{3} \times Z_3) \times Z_{25} \times (Z_{7} \times Z_{7} )$
        \item  $(Z_2\times Z_2) \times Z_9 \times (Z_{5} \times Z_5) \times (Z_{7} \times Z_7)$
        \item  $Z_4 \times (Z_3\times Z_3) \times (Z_{5} \times Z_5) \times (Z_{7} \times Z_7)$
        \item  $(Z_2 \times Z_2) \times (Z_3\times Z_3) \times (Z_{5} \times Z_5) \times (Z_{7} \times Z_7)$
      \end{itemize}
    \end{parts}
  \end{solution}

  \question
  \begin{parts}
    \part Prove that any finite group has a finite exponent
    \part Give an example of an infinite group with finite exponent
    \part Does a finite group of exponent $m$ always contain an element of order $m$?
  \end{parts}
  \begin{solution}
    \begin{parts}
      \part We know that for all finite group $G$ with order $|G|$, $g^{|G|} = e$ for all $g \in G$. Therefore the exponent of the group can be at most $|G|$.
      \part Let $G = \{ z \in \mathbb{C} \ : \  z^n = 1 \textrm{ for some } n \in \mathbb{Z} \}$. Then
    \end{parts}
  \end{solution}

  \question
  Let $A$ be a finite Abelian group and let $p$ be a prime. Let \[
    A^p = \{ a^{p}  \ : \ a \in A  \} \quad \textrm{and} \quad A_p = \{ x  \ : \ x^p = 1  \}
  \]
  \begin{parts}
    \part Prove that $A/A^p \cong A_p$
    \part Prove that the number of subgroups of $A$ of order $p$ equals the number of subgroups of $A$ of index $p$.
  \end{parts}
  \begin{solution}
    \begin{parts}
      \part We'll first show that $A^p, A_p$ are subgroups of $A$. Let $a^p, b^p \in A^p$. Then $(a^{p})^{-1}b^p = a^{-p}b^p = (a^{-1})^pb^p  = (a^{-1}b)^p$. Since $a^{-1}b \in A$, we get that $(a^p)^{-1}b^p = (a^{-1}b)^p \in A^p$. Hence $A^p$ is a subgroup.
      Now let $a, b \in A_p$. Then $a^p = b^p = 1$. Then $(a^{-1}b)^p = (a^{-1})^p b^p = (a^p)^{-1} b^p =  1$. Hence $a^{-1}b \in A_p$. Hence $A_p$ is also is a subgroup.
      Since $A$ is Abelian, both of these are normal in $A$. Now let $A \cong A_1 \times A_2 \times \ldots \times A_n$ with $|A_i| = p_i^{\alpha_i}$ and $p_i > p_{i+1}$ where each $p_i$ are distinct primes. 

      We'll show that both the groups $A/A^p$ and $A_p$ have exponent $p$. Then both their orders must be a power of $p$. Then if we show that the order of both these groups are the same, by the fundamental theorem of finitely generated Abelian groups, we'll get that they are isomorphic.

      To see that $A_p$ has exponent $p$, let $a \in A_p$, then by the definition of $A_p$, we see that $a^p = 1$. Since $p$ is a prime, we see that the exponent of the group should be either $p$ or 1.

      Now for $A/A^p$, let $aA^p \in A/A^p$, then $(aA^p)^p = a^pA^p = A^p$ implies the exponent of $A/A^p$ should be either $p$ or $1$ by the same reasoning.

      If the exponent of $A_p = 1$, $A_p$ must be the trivial group, which forces the homomorphism $\phi: A \to A : a \to a^p$ to be isomorphic, since $\textrm{Ker}(\phi) = A_p$. This gives $\phi(A) = A^p$ and that $A/A^p = A/A = \{ 0 \} = A_p$. 

      If the exponent of the group $A_p = p$, then there is an element in $a \in A$ with order $p$. Then $aA^p$ must also have order $p$ forcing the exponent of $A/A^p$ to be $p$. Hence we see that the groups $A_p$ and $A/A^p$ has the same exponent.

      Now by the same map $\phi$ above, we see that $A/A_p \cong A^p$. Therefore $\frac{|A|}{|A_p|} = |A^p|$ and we get $ \frac{|A|}{|A^p|} = |A_p|$, which gives us that the order of the groups under consideration are the same. Hence we get $ A/A^p \cong A_p$.


      \part Let $G$ be a subgroup of $A$ of order $p$. Then $g^p = 1$ for all $g \in G$. Hence $G \subset A_p$, which gives $G \leqslant A_p$, since $A_p$ itself is a subgroup. Conversely if $H$ is a subgroup of $A_p$ of order $p$, then trivially it is a subgroup of $A$ of the same order. Hence there's a correspondence between subgroups of $A$ of order $p$ and the subgroups of $A_p$ of order $p$. Since $A_p \cong A/A^p$, we get that the number of subgroups of $A_p$ of order $p$ is equal to the number of subgroups of $A/A^p$ of order $p$.

      % Now let $G = \{ [g_1] , [g_2] , \ldots , [g_p] \ : \ g_i \in A\}$ be a subgroup of $A/A^p$ of order $p$. Then we claim that $\tilde{G} = \{ g_ia   \ : \ [g_i] \in G, a \in A^p \} = \cup_{i = 1}^{p}g_iA^p$ is a subgroup of $A$ with $|\tilde{G}:A^{p}| = p$. To see that $\tilde{G}$ is a subgroup, let $g_ia, g_jb \in \tilde{G}$. Then $(g_ia)^{-1}(g_jb) = (a^{-1}g_i^{-1})(g_jb) = a^{-1}(g_i^{-1}g_j)b = (g_i^{-1}g_j)a^{-1}b$. Hence $[(g_ia)^{-1}(g_jb)] = [(g_i^{-1}g_j)a^{-1}b] = [g_i^{-1}g_j] = [g_k]$ for some $[g_k] \in G$. This gives that $(g_ia)^{-1}(g_jb) = g_ka_k$ for some $[g_k] \in G, a_k \in A^p$. Hence $(g_ia)^{-1}(g_jb) \in \tilde{G}$ and therefore $\tilde{G}$ is a subgroup of $A$. Moreover we see that $A^p \leqslant \tilde{G}$ and $|\tilde{G}: A^p| = p$ by the construction of $\tilde{G}$. Conversely if $H$ is any subgroup of $A$ with $|H:A^p| = p$, then $H/A^p$ will be a subgroup of $A/A^p$ of order $p$. Hence the number of subgroups of $A/A^p$ of order $p$ is equal to the number of subgroups $H \leqslant A$ with $|H/A^p| = p$

      Now let $G$ be a subgroup $A$ of index $p$. Then $A/G = \{ [h_1], [h_2], \ldots,  [h_p] \}$. Then $h_i^p \in G$. Since $h_i$ can be chosen to be any $a \in A$, we see that $A^p \leqslant G$. Therefore, by the third isomorphism theorem, we see that $p = |A/G| = |\frac{A/A^p}{G/A^p}|$. Hence $G/A^p$ is a subgroup of $A/A^p$ of index $p$. Conversely if $H/A^p$ is a subgroup of $A/A^p$ of index $p$, let $\frac{A/A^p}{H/A^p} = \{[a_1A^p], [a_2A^p], \ldots, [a_pA^p] \}$ then \[
           \tilde{H} = \bigcup_{i = 1}^{p}a_iA^p
      \]
    \end{parts}

  \end{solution}

  \question
  Let $n$ and $k$ be positive integers and let $A$ be the free Abelian group of rank $n$. Prove that $A/kA$ is isomorphic to the direct product of $n$ copies of $\mathbb{Z}/k \mathbb{Z}$.
  \begin{solution}
  Since we know that $A \cong  \mathbb{Z}^n$, in essence, we just need to show that $\mathbb{Z}^n/ k \mathbb{Z}^n \cong (\mathbb{Z}/ k \mathbb{Z})^n$. Consider the map $\phi: \mathbb{Z}^n \to (\mathbb{Z}/ k \mathbb{Z})^n$ as $\phi(a_1 , a_2 , \ldots , a_n) = ([a_1] , [a_2] , \ldots , [a_n])$, where $[a_i] = a_i + k \mathbb{Z}$.

  Clearly $\phi$ is a homomorphism. Let $a = (a_1 , a_2 , \ldots, a_n), b = (b_1 , b_2 , \ldots , b_n) \in \mathbb{Z}^n$. Then $\phi(a^{-1}b) = \phi(-a_1 + b_1 , -a_2 + b_2 , \ldots , -a_n + b_n) = ([b_1 - a_1], [b_2 - a_2], \ldots, [b_n - a_n]) = ([b_1] , [b_2] , \ldots , [b_n]) - ([a_1] , [a_2] , \ldots , [a_n]) = \phi(b) - \phi(a)$.

  Moreover, $\phi$ is a surjection. If $([a_1] , [a_2] , \ldots , [a_n]) \in (\mathbb{Z}/k \mathbb{Z})^n$, then $\phi(a_1 , a_2 , \ldots , a_n) =([a_1] , [a_2] , \ldots , [a_n])$, shows that $\phi$ is surjective.

  Now we'll show that $\textrm{Ker}(\phi) = k \mathbb{Z}^n$. $\phi(a_1 , a_2 , \ldots , a_n) = ([a_1] , [a_2] , \ldots , [a_n]) = 0$ if and only if $[a_i] \in k \mathbb{Z}$ for each $1 \le i \le n$. This is equivalent to $(a_1 , a_2 , \ldots , a_n) \in (k \mathbb{Z})^n = k \mathbb{Z}^n$. Hence we get that $\textrm{Ker}(\phi) = k \mathbb{Z}^n$. 

  Now by the first isomorphism theorem, we get that $\mathbb{Z}^n/ k \mathbb{Z}^n \cong (\mathbb{Z}/ k \mathbb{Z})^n$.
  \end{solution}



\end{questions}
\printbibliography[heading=bibintoc]
\end{document}



% initial settings
\documentclass[12pt]{exam}
\usepackage{geometry}
\usepackage{graphicx}
\usepackage{enumitem}
\usepackage{tikz-cd}
\usepackage[usenames,dvipsnames]{xcolor}
\usepackage[backend=biber, style=alphabetic]{biblatex}
\usepackage{url,hyperref}

\usepackage{amsmath} % math symbols, matrices, cases, trig functions, var-greek symbols.
\usepackage{amsfonts} % mathbb, mathfrak, large sum and product symbols.
\usepackage{amssymb} % extended list of math symbols from AMS. https://ctan.math.washington.edu/tex-archive/fonts/amsfonts/doc/amssymb.pdf
\usepackage{amsthm} % theorem styling.
\usepackage{mathrsfs} % mathscr fonts.
\usepackage{yhmath} % widehat.
\usepackage{empheq} % emphasize equations, extending 'amsmath' and 'mathtools'.
\usepackage{bm} % simplified bold math. Do \bm{math-equations-here}

% geometry of paper
\geometry{
  a4paper, % 'a4paper', 'c5paper', 'letterpaper', 'legalpaper'
  asymmetric, % don't swap margins in left and right pages. as opposed to 'twoside'
  centering, % to center the content between margins
  bindingoffset=0cm,
}

% hyprlink settings
\hypersetup{
  colorlinks = true,
  linkcolor = {red!60!black},
  anchorcolor = red,
  citecolor = {green!50!black},
  urlcolor = magenta,
  }

% theorem styles
\theoremstyle{plain} % default; italic text, extra space above and below
\newtheorem{theorem}{Theorem}[section]
\newtheorem{proposition}{Proposition}[section]
\newtheorem{lemma}{Lemma}[section]
\newtheorem{corollary}{Corollary}[theorem]

\theoremstyle{definition} % upright text, extra space above and below
\newtheorem{definition}{Definition}[section]
\newtheorem{example}{Example}[section]

\theoremstyle{remark} % upright text, no extra space above or below
\newtheorem{remark}{Remark}[section]
\newtheorem*{note}{Note} %'Notes' in italics and without counter 

% renewcommands for counters
\newcommand{\propositionautorefname}{Proposition}
\newcommand{\definitionautorefname}{Definition}
\newcommand{\lemmaautorefname}{Lemma}
\newcommand{\remarkautorefname}{Remark}
\newcommand{\exampleautorefname}{Example}

\addbibresource{articles.bib}


\begin{document}

\title{MATH 6302 - Modern Algebra \\ Homework 3}

% author list
\author{
Joel Sleeba \\
}

\maketitle
\printanswers
\unframedsolutions

\begin{questions}
  
  \question
  \begin{solution}
    \[\begin{tikzcd}
    	& {Q_8} \\
    	{\langle i \rangle} & {\langle j \rangle} & {\langle k \rangle} \\
    	& {\langle -1\rangle} \\
    	& {\langle 1 \rangle}
    	\arrow[from=2-1, to=1-2]
    	\arrow[from=2-2, to=1-2]
    	\arrow[from=2-3, to=1-2]
    	\arrow[from=3-2, to=2-1]
    	\arrow[from=3-2, to=2-2]
    	\arrow[from=3-2, to=2-3]
    	\arrow[from=4-2, to=3-2]
    \end{tikzcd}\]
  \end{solution}

  \question
  \begin{solution}
    \[\begin{tikzcd}
    	& {\langle 1\rangle} \\
    	{\langle 3\rangle} && {\langle 2\rangle} \\
    	{\langle 9\rangle} && {\langle 6\rangle} \\
    	{\langle 18\rangle} && {\langle 12\rangle} \\
    	& {\langle 0\rangle}
    	\arrow[from=2-1, to=1-2]
    	\arrow[from=2-3, to=1-2]
    	\arrow[from=3-1, to=2-1]
    	\arrow[from=3-3, to=2-1]
    	\arrow[from=3-3, to=2-3]
    	\arrow[from=4-1, to=3-1]
    	\arrow[from=4-3, to=3-3]
    	\arrow[from=5-2, to=4-1]
    	\arrow[from=5-2, to=4-3]
    \end{tikzcd}\]
  \end{solution}

  \question
  \begin{solution}
    Consider the subset $\mathbb{N}$ of the group $\mathbb{Q}$ under addition. Since adding any two natural numbers give another natural number, we see that $\mathbb{N}$ is closed under addition. But there is no element in $\mathbb{N}$ which acts as an identity in $\mathbb{N}$. Hence it is an example of a subset of a group which closed under the group, yet not a subgroup.
  \end{solution}

  \question
  \begin{solution}
    Let $S_i$ be a finite generating subset of $G_i$. Consider the set 
    $$S = \{ (e_1, \ldots, e_{i+1}, s_i, e_{i+1}, \ldots, e_n )  \ \big| \  s_i \in S_i, e_i \textrm{ is the identity in } G_i, 1 \le i \le n \}$$
    Clearly $|S| = \sum_{i = 1}^{n} |S_i| < \infty$. We claim that $S$ generate $G = G_1 \times G_2 \times \cdots \times G_n$. For any element $(g_1 , g_2 , \ldots , g_n) \in G$, we see that $$(g_1 , g_2 , \ldots , g_n) = (g_1, e_2, \ldots, e_n)(e_1, g_2, \ldots , e_n) \ldots (e_1, e_2, \ldots g_n)$$
    Hence it is enough if we show $S$ generate $(e_1, e_2, \ldots, g_i, \ldots , e_n)$ for an arbitrary $1\le i \le n$. 
    But since $S_i$ is a generating set for $G_i$, there exists a collection $s_{ij} \in S_i$ such that $g_i = s_{i1}s_{i2}\ldots s_{ik}$. Then \begin{align*}(e_1, e_2, \ldots, s_{i1}, \ldots, e_n)\ldots(e_1, e_2, \ldots, s_{ik}, \ldots, e_n) &= (e_1, e_2, \ldots, (s_{i1}s_{i2}\ldots s_{ik}), \ldots, e_n) \\ 
    &= (e_1, e_2, \ldots, g_i, \ldots, e_n)
    \end{align*}
    Hence we get that $S$ generate $G$.
  \end{solution}

  \question
  \begin{solution}
    Assume $GL_2(\mathbb{Q})$ is finitely generated. We see that for every $r \in \mathbb{Q}$, the matrix \[
    \begin{pmatrix}
      1 & 0 \\
      0 & r
    \end{pmatrix}
  \]
    has determinant $r$. Now since we know $\det(AB) = \det(A)\det(B)$ for all $A, B \in GL_2(\mathbb{Q})$. Hence $\det: GL_2(\mathbb{Q}) \to \mathbb{Q}\setminus \{ 0 \}$ is a surjective group homomorphism. Now if $GL_2(\mathbb{Q})$ is finitely generated by a set $S \subset GL_2(\mathbb{Q})$, we must have $\det(S)$ generate $\mathbb{Q} \setminus \{ 0 \}$. But this gives a contradiction since we know $\mathbb{Q} \setminus \{ 0 \}$ is not finitely generated.
  \end{solution}

  \question
  \begin{solution}
    Since $G$ is Abelian and finitely generated by $g_1 , g_2 , \ldots , g_n$, every element of $G$ can be written as $g_1^{\alpha_1}g_2^{\alpha_2}\ldots g_n^{\alpha_n}$ where $0 \le \alpha_i < |g_i|$. 
    Now consider the map $\phi: Z_{|g_1|}\times Z_{|g_2|} \times \ldots Z_{|g_n|} \to G$ as $(\alpha_1, \alpha_{2}, \ldots, \alpha_n) \to g_1^{\alpha_1}g_2^{\alpha_2}\ldots g_n^{\alpha_n}$. Clearly $\phi$ is a surjection. Therefore the cardinality of the domain is greater than the cardinality of the range, which gives our required inequality.
  \end{solution}

  \question
  \begin{solution}
    \begin{parts}
      \part $\sigma = (1\  11\  3\ 9)(2\ 12\ 4)(5\ 6\ 8)(7)(10)$, $\tau = (1\ 3\ 5)(2\ 10)(4\ 12\ 6\ 8\ 7\ 11\ 9)$
      \part $\sigma = (1\ 9)(1 \ 3)(1 \ 11)(2 \ 4)(2 \ 12)(5 \ 8)(5 \ 6) \\ $
      $\tau = (1 \ 5)(1 \ 3)(2 \ 10)(4 \ 9)(4 \ 11)(4 \ 7)(4 \ 8)(4 \ 6)(4 \ 12)$
      \part $\sigma^2\tau = (1)(2\  10\  4)(3\  8\  7\  9\  12\  5)(6)(11)$
      \part We see that $\sigma \tau = (1\ 9\  2\ 10\ 12\ 8\ 7\ 3\ 6\ 5\ 11)(4)$. Hence we'll get $( \sigma \tau)^{-1} = (11\ 5\ 6\ 3\ 7\ 8\ 12\ 10\ 2\ 9\ 1)(4)$
    \end{parts}
  \end{solution}

  \question
  \begin{solution}
    We'll use the combinatorial distinct balls in similar holes problem. Assume the places to put numbers in the 5 cycle representation (a, b, c, d, e) as holes and the numbers in $Z_{10}$ as balls. There are $10P5 = 10\times 9 \times 8 \times 7 \times 6$ ways to place balls in these holes. But we see that for every 5 cycle, there are 5 distinct ways to represent them like this. That is $(a, b, c, d, e), ( b, c, d, e, a), ( c, d, e, a, b), (d, e, a, b, c), (e, a, b, c, d)$ all correspond to the same cycle. Therefore the number of distinct  cycles is $(10P5)/5 = \frac{10\times9\times8\times7\times6}{5} = 6048$
  \end{solution}

  \question
  \begin{solution}
    \begin{parts}
      \part Since it is given that $\sigma$ is a 36 cycle, $|\sigma| = 36$. Moreover we know that for any group $G$ with $g \in G$, $|g^k| = \frac{|g|}{(|g|, k)}$. Hence $|\sigma^k| = \frac{36}{(36, k)}$. And the possible values for $|\sigma^k|$ for $1 \le k \le 36$ are precisely the factors of $36$. $36 = 2^23^2$. Hence the possible values are $\{ 36, 18, 12, 9, 6, 4, 3, 2, 1 \}$.
      \part Since $\langle  \sigma  \rangle$ is cyclic with order $36$, it is isomorphic with $Z_{36}$. Therefore the question translates to finding the number of generators for $Z_{36}$, which is $\phi(36) = 12$.
    \end{parts}
  \end{solution}

  \question
  \begin{solution}
    From question 13, we see that $A_4 = \langle (1\ 2)(3\ 4), (1\ 2\ 3) \rangle$. We know $|A_4| = 4!/2 = 12$ while $|(1\ 2)(3\ 4)| = 2$ and $|(1\ 2\ 3)| = 3$. Since $12 > 2\times3 = 6$, this works as an example.
  \end{solution}

  \question
  \begin{solution}
    Since disjoint cycles commute in $S_n$, we get that $(\sigma_1\sigma_2 \cdots \sigma_l)^m = \sigma_1^m \sigma_2^m \cdots \sigma_l^m$. Let $n = \textrm{lcm}(|\sigma_1|, |\sigma_2|, \ldots, |\sigma_l|)$. Then $|\sigma_i|\big|n$ for all $1 \le i \le l$ and therefore $\sigma_i^n = e$ for all $1 \le i \le l$. Hence we see that $(\sigma_1\sigma_2 \cdots \sigma_l)^n = (\sigma_1^n \sigma_2^n \cdots \sigma_l^n) = e$. Therefore $ |\sigma_1\sigma_2 \cdots \sigma_l|\big|n$. Hence $$|\sigma_1\sigma_2 \cdots \sigma_l|\  \Big| \ \textrm{lcm}(|\sigma_1|, |\sigma_2|, \ldots, |\sigma_l|)$$

    Now if $m \in \mathbb{N}$ such that $(\sigma_1\sigma_2 \cdots \sigma_l)^m = \sigma_1^m \sigma_2^m \cdots \sigma_l^m = e$,  we must have $\sigma_i^m = e$ for each $1 \le i \le l$. This is because each of the cycle $\sigma_i$ are pairwise disjoint, so must be their powers. This implies $|\sigma_i|\big| m$ for each $1 \le i \le l$. This gives that $\textrm{lcm}(|\sigma_1|, |\sigma_2|, \ldots, |\sigma_l|) \big| m$. Now take $ m = |\sigma_1\sigma_2 \cdots \sigma_l|$ to get $$\textrm{lcm}(|\sigma_1|, |\sigma_2|, \ldots, |\sigma_l|)\  \Big| \  |\sigma_1\sigma_2 \cdots \sigma_l|$$

    Therefore $\textrm{lcm}(|\sigma_1|, |\sigma_2|, \ldots, |\sigma_l|) = |\sigma_1\sigma_2 \cdots \sigma_l|$
  \end{solution}

  \question
  \begin{solution}
     \begin{parts}
       \part We will use a counting procedure that will exhaust all possible orders of elements using disjoint cycle representation. Note from the previous problem that the order of $\sigma_1\sigma_2 \cdots \sigma_n$ is $\textrm{lcm}(|\sigma_1|, |\sigma_2|, \ldots, |\sigma_l|) \big||\sigma_1\sigma_2 \cdots \sigma_l|$ if $\sigma_i$ are disjoint cycles. Moreover in the disjoint cycle representation $\sigma_1\sigma_2 \cdots \sigma_n$ we can demand that the cycles be ordered in the descending order of their orders. That is $\sigma_1\sigma_2 \cdots \sigma_n \in S_m$ must have $|\sigma_1| \ge |\sigma_{2}| \ge \cdots \ge |\sigma_{n}|$

       Now we will iterate over the number of elements in $Z_{10}$ fixed by elements in $S_{10}$. We will denote the number of elements being fixed using the variable $r$. \begin{itemize}
         \item $r=10$. If every elements of $Z_{10}$ are fixed by an element in $S_{10}$, it must be the identity element. Hence the order of such elements is $1$.
         \item $r=9$. If an element fixes $9$ elements in $Z_{10}$, it must also fix the last element, since $S_{10}$ is the collection of bijections of $Z_{10}$. Hence the only possibility is if the element in $S_{10}$ is the identity which has order $1$.
         \item $r=8$. If an element in $S_{10}$ fixes $8$ elements (avoiding the case where it fixed more than $8$ elements) in $Z_{10}$, it must be of the form $(a\ b)$, which has order $2$.
         \item $r=7$. Then the element must be of the form $(a\ b\ c)$, which has order $3$
         \item $r=6$. Then the element must be either of the two forms $(a \ b \ c\ d)$ or $(a\ b)(c \ d)$. Therefore the possible orders are $4$ and $2$. 
         \item $r=5$. Then the element must be of the following possible forms
                  \begin{enumerate}[label=(\roman*)]
                    \item $(a\ b \ c \ d \ e)$
                    \item $(a \ b \ c)(d \ e)$
                  \end{enumerate}
                  Note that since we showed every element can be arranged with the disjoint cycles in the descending order of their orders, we omit $(a\ b)( c\ d\ e)$. 

                  Hence the possible orders are $5$ and $6$. 
         \item $r=4$. The possible forms are 
           \begin{enumerate}[label=(\roman*)]
             \item $(a\ b\ c\ d\ e\ f)$
             \item $(a\ b\ c\ d)( e\ f)$
             \item $(a\ b\ c)(d\ e\ f)$
           \end{enumerate}
           Hence the possible orders are $6, 8$ and $9$.
         \item $r=3$. The possible forms are 
           \begin{enumerate}[label=(\roman*)]
             \item $(a\ b\ c\ d\ e\ f \ g)$
             \item $(a\ b\ c\ d\ e)( f\ g)$
             \item $(a\ b\ c\ d)(e\ f\ g)$
             \item $(a\ b\ c)(d\ e)(g\ f)$
           \end{enumerate}
           Hence the possible orders are $7, 10, 12, 6$.
         \item $r=2$. The possible forms are
           \begin{enumerate}[label=(\roman*)]
             \item $(a\ b\ c\ d\ e\ f \ g\ h)$
             \item $(a\ b\ c\ d\ e\ f)( g\ h)$
             \item $(a\ b\ c\ d\ e)(f\ g\ h)$
             \item $(a\ b\ c\ d)(e\ f\ g\ h)$
             \item $(a\ b\ c\ d)( e\ f)(g\ h)$
             \item $(a\ b\ c)(d\ e\ g)(f\ h)$
             \item $(a\ b)( c\ d)( e\ f)(g\ h)$
           \end{enumerate}
           Hence the possible orders are $8, 12, 15, 4, 6, 2$.
         \item $r=1$. The possible forms are 
           \begin{enumerate}[label=(\roman*)]
             \item $(a\ b\ c\ d\ e\ f \ g\ h\ i)$
             \item $(a\ b\ c\ d\ e\ f \ g)(h\ i)$
             \item $(a\ b\ c\ d\ e\ f)(g\ h\ i)$
             \item $(a\ b\ c\ d\ e)( f\ g\ h\ i)$
             \item $(a\ b\ c\ d\ e)( f\ g)(h\ i)$
             \item $(a\ b\ c\ d)(e\ f\ g)(h\ i)$
             \item $(a\ b\ c)(d\ e\ f)(g\ h\ i)$
             \item $(a\ b\ c)(d\ e)(f\ g)(h\ i)$
           \end{enumerate}
           Hence the possible orders are $9, 14, 18, 20, 10, 12, 3, 6$
         \item $r=0$. Then the possible forms are 
           \begin{enumerate}[label=(\roman*)]
             \item $(a\ b\ c\ d\ e\ f \ g\ h\ i\ j)$
             \item $(a\ b\ c\ d\ e\ f \ g\ h)(i\ j)$
             \item $(a\ b\ c\ d\ e\ f \ g)(h\ i\ j)$
             \item $(a\ b\ c\ d\ e\ f)( g\ h\ i\ j)$
             \item $(a\ b\ c\ d\ e\ f)( g\ h)(i\ j)$
             \item $(a\ b\ c\ d\ e)(f\ g\ h\ i\ j)$
             \item $(a\ b\ c\ d\ e)(f\ g\ h)(i\ j)$
             \item $(a\ b\ c\ d)(e\ f\ g\ h)(i\ j)$
             \item $(a\ b\ c\ d)(e\ f\ g)(h\ i\ j)$
             \item $(a\ b\ c\ d)(e\ f)( g\ h)(i\ j)$
             \item $(a\ b\ c)(d\ e\ f)(g\ h)(i\ j)$
             \item $(a\ b)( c\ d)( e\ f)(g\ h)(i\ j)$
           \end{enumerate}
           Hence the possible orders are $10, 16, 21, 24, 12, 5, 30, 4, 12, 6, 2$
       \end{itemize}
       Therefore all the possible orders of elements in $S_{10}$ are 30, 24, 21, 20, 18, 16, 15, 14, 12, 10, 9, 8, 7, 6, 5, 4, 3, 2, 1.
       \part $30 = 2\times3\times5$. Hence the possible ways to write 30 as a product of co-primes are $1 \times 30, 2 \times 15, 3\times10, 6\times 5, 2\times3\times5$, where the sum of the coprimes is the lowest at $2\times3\times5$. Hence $S_{10}$ is the group we are looking for.
     \end{parts}
  \end{solution}

  \question
  \begin{solution}
    Clearly $(12)(34), (123) \in A_4$. Since we know that $A_4$ have $12$ elements once we find $7$ distinct elements in $\langle (12)(34), (123)  \rangle$, using Lagrange's theorem we can be sure that $  \langle (12)(34) , (123) \rangle  = A_4$. \begin{itemize}
      \item $(123)^{-1} = (132)$
      \item $(12)(34)(123) = (243)$
      \item $(123)(12)(34) = (134)$
      \item $(12)(34)(123) = (143)$
      \item $(12)(34)(132) = (143)$
      \item $(132)(12)(34) = (234)$
    \end{itemize}
    Since all the 7 elements above (including $(12)(34), (123)$) are in $A_4$ and generated by $(12)(34)$ and $(123)$, we conclude that $A_4 = \langle (12)(34) , (123) \rangle $
  \end{solution}

  \question
  \begin{solution}
    \begin{parts}
      \part Isomorphic. $\phi:\mathbb{Z} \to 8 \mathbb{Z}:= n \to 8n$ is an isomorphism. $\phi(a + b) = 8(a+b) = 8a + 8b = \phi(a) + \phi(b)$.
      \part Not isomorphic. $\mathbb{Z}$ is cyclic but $\mathbb{Q}$ is not.
      \part Not isomorphic. By Cantor's diagonalization argument we know that there does not exist a bijection between $\mathbb{Q}$ and $\mathbb{R}$.
      \part Not isomorphic. The only element in $\mathbb{R}$ of finite order is $0$. But the same does not hold for $\textrm{SL}_2(\mathbb{R})$. For example $I$ and $-I$.
      \part Isomorphic. Since $49 = 7^2$, by primitive root theorem, we see that $\mathbb{Z}_{49}^{*}$ has $\phi(49) = 42$ elements and is cyclic. Since every cyclic group of same order is isomorphic we get $\mathbb{Z}_{49}^{*} \cong \mathbb{Z}_{42}$. We can verify that $3$ is a primitve root modulo 49, hence the map $\phi: Z_{49}^{*} \to Z_{42}: 3^a \to a \mod 42$ is an isomorphism.
      \part Isomorphic. We first notice that $C_2 \times C_3 \cong C_6$ by the map $\phi: (a, b) \to ab$ is an isomorphism. (Note that here we're identifying $C_6, C_3$ and $C_2$ with $Z_6, Z_3$ and $Z_2$ respectively). Then $\psi: C_2 \times C_2 \times C_3 \to C_2 \times C_6 := (a, b, c) \to (a, \phi(b, c))$ is an isomorphism which shows the groups are isomorphic. 
      \part Not isomorphic. In last assignment we proved that the group $(\mathcal{P}(\{ 1, 2 \}), \Delta) \cong V_4$. Moreover we know that $V_4 \not \cong C_4$.
      \part Isomorphic. Consider the map $\phi: (\mathcal{P}(\{ 1, 2, 3 \}), \Delta) \to C_2 \times C_2 \times C_2 := A \to (\chi_A(1), \chi_A(2), \chi_A(3))$. Since $A \Delta B = (A \setminus  A \cap B) \cup (B \setminus  A\cap B)$, we get
      \begin{align*}
        \chi_{A \Delta B} & = \chi_{A \setminus A\cap B} + \chi_{B \setminus A \cap B} \\
        & = (\chi_A - \chi_A \chi_B) + (\chi_B - \chi_A \chi_B) \\ 
        & = (\chi_A + \chi_B) \mod 2
      \end{align*}
      which proves that our map $\phi$ is a group homomorphism. Moreover it is bijective, hence a group isomorphism.
    \end{parts}
  \end{solution}

  \question
  \begin{solution}
    Since $Z_n$ is cyclic, if $\phi: Z_n \to Z_n$ is a homomorphism, it is completely determined by where it sends its generator. Moreover $\phi$ should be an isomorphism, then it must send generators to generators. 

    Let's fix $1 \in Z_n$ as the generator of the domain. We see that every integer less than $n$, which are relatively prime to $n$ will again generate $Z_n$. Moreover by the definition of Euler-totient function, we know that there are exactly $\varphi(n)$ such numbers. Therefore $\phi$ to be isomorphism, $ \phi(1)$ has $  \varphi(n)$ many choices, and each of them give a different isomorphism. Hence there are $\varphi(n)$ automorphisms of $Z_n$.
  \end{solution}

  \question
  \begin{solution}
    Since we know that $C_2 \times C_2 \cong V_4$, there is a correspondence between homomorphims of $C_2 \times C_2$ and $V_4$, and specifically automorphisms. Now, we know that $V_4 = \langle a, b \ : \ a^2 = b^2 = 1, ab = ba \rangle = \{ e, a, b, c \}$ where $c = ab$. If $\phi: V_4 \to V_4$ is an automorphism, then $\phi(e) = e$. Since the order of all the rest of the elements are $2$, we have freedom over how $\phi$ permutes elements of the set $\{ a, b, c \}$. Hence we see that there are $3! = 6$ automorphisms of $V_4$.
  \end{solution}

  \question
  \begin{solution}
    $23$ is a prime. Therefore $|Z_{23}^*| = \phi(23) = 22$ elements and is cyclic by primitive root theorem. Therefore $Z_{23}^* \cong Z_{22}$. Since $Z_{23}^*$ is cyclic and $5$ is a generator, if $\phi: Z_{23}^* \to Z_{23}^*$ is any homomorphism, it is completely determined by the $\phi(5)$. Moreover if $\phi$ has to an automorphism, then $\phi(5)$ must also be a generator for $Z_{23}^*$. Since $Z_{23}^* \cong Z_{22}$, we see that there are $\varphi(22) = 10$ generators for $Z_{23}^{*}$. Hence there are $10$ choices for $\phi(5)$ in $Z_{23}^*$ which makes $\phi$ an automorphism.
  \end{solution}


\end{questions}
\printbibliography[heading=bibintoc]
\end{document}

% initial settings
\documentclass[12pt]{exam}
\usepackage{geometry}
\usepackage{graphicx}
\usepackage{enumitem}
\usepackage[usenames,dvipsnames]{xcolor}
\usepackage[backend=biber, style=alphabetic]{biblatex}
\usepackage{url,hyperref}

\usepackage{amsmath} % math symbols, matrices, cases, trig functions,
% var-greek symbols.
\usepackage{amsfonts} % mathbb, mathfrak, large sum and product symbols.
\usepackage{amssymb} % extended list of math symbols from AMS.
% https://ctan.math.washington.edu/tex-archive/fonts/amsfonts/doc/amssymb.pdf
\usepackage{amsthm} % theorem styling.
\usepackage{mathrsfs} % mathscr fonts.
\usepackage{yhmath} % widehat.
\usepackage{empheq} % emphasize equations, extending 'amsmath' and 'mathtools'.
\usepackage{bm} % simplified bold math. Do \bm{math-equations-here}

% geometry of paper
\geometry{
  a4paper, % 'a4paper', 'c5paper', 'letterpaper', 'legalpaper'
  asymmetric, % don't swap margins in left and right pages. as
  % opposed to 'twoside'
  centering, % to center the content between margins
  bindingoffset=0cm,
}

% hyprlink settings
\hypersetup{
  colorlinks = true,
  linkcolor = {red!60!black},
  anchorcolor = red,
  citecolor = {green!50!black},
  urlcolor = magenta,
}

% theorem styles
\theoremstyle{plain} % default; italic text, extra space above and below
\newtheorem{theorem}{Theorem}[section]
\newtheorem{proposition}{Proposition}[section]
\newtheorem{lemma}{Lemma}[section]
\newtheorem{corollary}{Corollary}[theorem]

\theoremstyle{definition} % upright text, extra space above and below
\newtheorem{definition}{Definition}[section]
\newtheorem{example}{Example}[section]

\theoremstyle{remark} % upright text, no extra space above or below
\newtheorem{remark}{Remark}[section]
\newtheorem*{note}{Note} %'Notes' in italics and without counter

% renewcommands for counters
\newcommand{\propositionautorefname}{Proposition}
\newcommand{\definitionautorefname}{Definition}
\newcommand{\lemmaautorefname}{Lemma}
\newcommand{\remarkautorefname}{Remark}
\newcommand{\exampleautorefname}{Example}

\addbibresource{articles.bib}

\begin{document}

\title{MATH 6320 \\ Homework 7}

% author list
\author{
  Joel Sleeba \\
}

\maketitle
\printanswers
\unframedsolutions

\begin{questions}

  \question
  \textcolor{red}{not finished}

  \begin{solution}

  \end{solution}

  \question
  \begin{solution}
    Since $r \in D_8$, has order $4$, if $\phi: D_8 \to D_8$ is any
    automorphism, then $\phi(r)$ must also have the same order. Hence
    the possible $\phi(r)$ are $r, r^{-1} \in D_8$. Similarly  since
    $|s| = 2$, $\phi(s)$ also  must have order $2$, which gives
    $\phi(s) \in \{ s, r^2, sr, sr^2, sr^3\}$. But since $\phi(r) \in
    \{ r , r^3  \}$, if $\phi(s) = r^2$, $\phi(D_8) = \langle  r
    \rangle $, and $\phi$ will not be an automorphism. Hence $\phi(s)
    \in \{  s, sr, sr^2, sr^3 \}$. Since $s, r$ generate $D_8$, and
    each of them have $4$ and $2$ possible options, by the counting
    argument, $\textrm{Aut}(D_8)$ can have at most 8 elements.
  \end{solution}

  \question
  \textcolor{red}{not finished}
  \begin{solution}
    Since $D_8 \trianglelefteq D_{16}$, we see that $\phi: D_{16} \to
    \textrm{Aut}(D_8): g \to \phi_g$, where $\phi_g: h \to ghg^{-1}$
    is a well defined map. Since
    \begin{align*}
      \phi_g \phi_{g^\prime}(h) & = \phi_g(g^\prime h(g^\prime)^{-1}) \\
      & = gg^\prime h (g^\prime)^{-1} g^{-1} \\
      & = (gg^\prime) h (gg^\prime)^{-1} \\
      & = \phi_{gg^\prime}(h)
    \end{align*}
    we see that $\phi$ is a group homomorphism. Moreover, we know
    that $\textrm{Ker}(\phi) = C_{D_{16}}(D_8) = \langle r \rangle =
    \{r, e\}$. Hence by
    the first isomorphism theorem, we see that $\phi(D_{16}) =
    \frac{D_{16}}{\langle r^4 \rangle} \cong D_8$. Moreover,
  \end{solution}

  \question
  \textcolor{red}{not finished}

  \begin{solution}
    From what we proved in the class, we know that if $H \leqslant
    G$, then $N_G(H)/C_G(H)$ is isomorphic to a subgroup of
    $\textrm{Aut}(H)$. Hence in the question, we know that
    $N_{S_p}(P)/C_{S_p}(P)$ is isomorphic to a subgroup of $\textrm{Aut}(P)$.

    Since $P$ is a cyclic group of order $p$, $P \cong
    \mathbb{Z}/p\mathbb{Z}$ and hence the number of automorphisms of
    $P$ are precisely $p-1$.

    Also $C_{S_p}(P) = P$. To show this, we first notice that every
    non-identity element in $P$ must be a $p$-cycle. If $p = 2$,
    $P = S_p$ and we've noting to prove. Hence assume $p >
    2$, then without loss of generality, assume $(1 \ 2 \ldots \ p)
    \in P$. Then $(1\ 2)(1 \ 2 \ldots \ p) = (2 \ 3 \ldots\ p) \neq
    (1 \ 3 \ 4 \ldots p) = (1 \ 2 \ldots \ p)(1\ 2)$.
  \end{solution}

  \question
  \begin{solution}
    Let $(1, k) \in C_K(H)$. Then for any $(h, 1) \in G$, $$(h, k) =
    (h\varphi(1)(1), k) = (h, 1)(1,k)  = (1, k)(h, 1) =  (1\varphi(k)(h), k)$$
    forces $\varphi(k)(h) = h$. Since this is true for all $h \in H$,
    we see that $\phi(k)$ is the trivial automorphism of $H$. Hence
    $k \in \textrm{Ker}(\phi)$.

    Conversely, if $k \in \textrm{Ker}(\phi)$, then $\phi(k)(h) = h$
    for all $h \in H$. Then for any $(h, 1) \in H$(identified as a
    subgroup of $G$) \[
      (h, 1)(1, k) = (h \varphi(1)(1), k) = (h, k) = (\phi(k)(h), k)
      = (1, k)(h, 1)
    \]
    shows that $(1, k) \in C_K(H)$. Hence $C_K(H) = \textrm{ Ker}(\varphi)$.
  \end{solution}

  \question
  \textcolor{red}{not finished}

  \begin{solution}
    We know that $\textrm{Hol}(H) = H \rtimes_\phi \textrm{Aut}(H)$,
    where $\phi: \textrm{Aut}(H) \to \textrm{Aut}(H)$ is the identity map.
    \begin{parts}
      \part We notice that $H = Z_2 \times Z_2 \cong V_4$, the Klein
      4 group. Therefore, let $H = V_4 = \{ 1, a, b, c \}$. Since we know that
      any two of $a, b, c$ generate the group $V_4$ we see that
      any permutation of $a, b, c$ will be a group automorphism.
      Hence we see that $\textrm{Aut}(H) \cong S_3$. Hence we see
      that $\textrm{Hol}(Z_2 \times Z_2) \cong H \rtimes K$, where $H
      = Z_2 \times Z_2$ and $K \cong S_3$.
      Also, $|H \rtimes K| = |H \times K| = |H| \times |K| = 4 \times 6 = 24$

      \part
    \end{parts}
  \end{solution}

  \question
  \textcolor{red}{not finished}
  \begin{solution}
    We know that since $75 = 3\times5^2$, the fundamental theorem for
    Abelian groups immediately gives two groups $Z_3 \times Z_{5^2}
    \cong Z_{75}$ and $Z_3 \times Z_5 \times Z_5$.
  \end{solution}

\end{questions}
\printbibliography[heading=bibintoc]
\end{document}



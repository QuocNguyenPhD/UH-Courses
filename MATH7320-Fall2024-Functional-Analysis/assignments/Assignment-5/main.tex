% initial settings
\documentclass[12pt]{exam}
\usepackage{geometry}
\usepackage{graphicx}
\usepackage{enumitem}
\usepackage[usenames,dvipsnames]{xcolor}
\usepackage[backend=biber, style=alphabetic]{biblatex}
\usepackage{url,hyperref}

\usepackage{amsmath} % math symbols, matrices, cases, trig functions,
% var-greek symbols.
\usepackage{amsfonts} % mathbb, mathfrak, large sum and product symbols.
\usepackage{amssymb} % extended list of math symbols from AMS.
% https://ctan.math.washington.edu/tex-archive/fonts/amsfonts/doc/amssymb.pdf
\usepackage{amsthm} % theorem styling.
\usepackage{mathrsfs} % mathscr fonts.
\usepackage{yhmath} % widehat.
\usepackage{empheq} % emphasize equations, extending 'amsmath' and 'mathtools'.
\usepackage{bm} % simplified bold math. Do \bm{math-equations-here}

% geometry of paper
\geometry{
  a4paper, % 'a4paper', 'c5paper', 'letterpaper', 'legalpaper'
  asymmetric, % don't swap margins in left and right pages. as
  % opposed to 'twoside'
  centering, % to center the content between margins
  bindingoffset=0cm,
}

% hyprlink settings
\hypersetup{
  colorlinks = true,
  linkcolor = {red!60!black},
  anchorcolor = red,
  citecolor = {green!50!black},
  urlcolor = magenta,
}

% theorem styles
\theoremstyle{plain} % default; italic text, extra space above and below
\newtheorem{theorem}{Theorem}[section]
\newtheorem{proposition}{Proposition}[section]
\newtheorem{lemma}{Lemma}[section]
\newtheorem{corollary}{Corollary}[theorem]

\theoremstyle{definition} % upright text, extra space above and below
\newtheorem{definition}{Definition}[section]
\newtheorem{example}{Example}[section]

\theoremstyle{remark} % upright text, no extra space above or below
\newtheorem{remark}{Remark}[section]
\newtheorem*{note}{Note} %'Notes' in italics and without counter

% renewcommands for counters
\newcommand{\propositionautorefname}{Proposition}
\newcommand{\definitionautorefname}{Definition}
\newcommand{\lemmaautorefname}{Lemma}
\newcommand{\remarkautorefname}{Remark}
\newcommand{\exampleautorefname}{Example}

\addbibresource{articles.bib}

\begin{document}

\title{MATH7320 - Functional Analysis \\ Homework 5}

% author list
\author{
  Joel Sleeba \\
}

\maketitle
\printanswers
\unframedsolutions

\begin{questions}

  \question
  \begin{solution}
    Since $\phi$ is distance preserving, we get
    \begin{align*}
      \|x\|^2 + \|y\|^2 - 2 \Re \langle x , y \rangle
      &=\langle x-y, x-y \rangle  \\
      &= \|x -y\|^2 \\
      &= \|\phi(x) - \phi(y)\|^2 \\
      &= \langle \phi(x) - \phi(y) ,  \phi(x) - \phi(y) \rangle \\
      &= \|\phi(x)\|^2 + \|\phi(y)\|^2 - 2 \Re \langle \phi(x) ,
      \phi(y) \rangle \\
      &=  \|x\|^2 +  \|y\|^2 - 2 \Re \langle \phi(x) ,  \phi(y) \rangle
    \end{align*}
    Thus we get $\Re \langle x , y \rangle  = \Re \langle \phi(x) ,
    \phi(y) \rangle $ for all $x, y \in \mathcal{H}$.

    Now let $x, y \in \mathcal{H}$ and $p, q \in \mathcal{H}$ such
    that $\phi(p) = r \phi(x), \phi(q) = s \phi(y)$, for $ r, s \in
    \mathbb{R}$. Surjectivity of $\phi$ allows us to find $p, q$. Then
    \begin{align*}
      \|\phi(rx + sy) - \phi(p) - \phi(q)\|^2 &= \langle  \phi(rx +
      sy) - \phi(p) - \phi(q) , \phi(rx + sy) - \phi(p) - \phi(q) \rangle \\
      &= \|\phi(rx + sy)\|^2 + \|\phi(p)\|^2 + \|\phi(q)\|^2 - 2 \Re
      \langle \phi(p) ,  \phi(q) \rangle   \\
      & \quad  - 2 \Re \langle \phi(rx + sy) ,  \phi(p) \rangle - 2
      \Re \langle \phi(rx + sy) ,  \phi(q) \rangle \\
      &= \|\phi(rx + sy)\|^2 + r^2\|\phi(x)\|^2 + s^2\|\phi(y)\|^2 - 2\Re
      \langle r \phi(x) ,  s\phi(y) \rangle   \\
      & \quad  - 2\Re \langle \phi(rx + sy) , r\phi(x) \rangle - 2
      \Re \langle \phi(rx + sy) , s\phi(y) \rangle \\
      &= \|\phi(rx + sy)\|^2 + r^2\|\phi(x)\|^2 + s^2\|\phi(y)\|^2 - 2rs\Re
      \langle \phi(x) ,  \phi(y) \rangle   \\
      & \quad  - 2r\Re \langle \phi(rx + sy) , \phi(x) \rangle - 2s
      \Re \langle \phi(rx + sy) , \phi(y) \rangle \\
      &= \|rx + sy\|^2 + r^2\|x\|^2 + s^2\|y\|^2 - 2 \Re
      \langle rx ,  sy \rangle   \\
      & \quad  - 2 \Re \langle rx + sy ,  rx \rangle - 2
      \Re \langle rx + sy ,  sy \rangle \\
      &= \| rx + sy - rp - sq\|^2 \\
      &= 0
    \end{align*}
  \end{solution}

  \question
  \textcolor{red}{not finished}
  \begin{solution}
    Since $S \perp S^\perp$, clearly $S \subset (S^\perp)^\perp$.
    Moreover, we know that $(S^\perp)^\perp =
    \textrm{Ker}(P_{S^\perp})$. Therefore $(S^\perp)^\perp$ is a
    closed subspace. Hence $\overline{\textrm{span}}(S) \subset
    (S^\perp)^\perp$. Conversely if $x \in (S^\perp)^\perp$, then $x
    \perp S^\perp$
  \end{solution}

  \question
  \textcolor{red}{not finished}
  \begin{solution}
    Let $T_n \in \mathcal{K}(\mathcal{X}, \mathcal{Y})$ be a Cauchy sequence.
  \end{solution}

  \question
  \begin{solution}
    Since we know that $I = P_{\mathcal{M}} + P_{\mathcal{M}^\perp}$,
    we see that $X = \mathcal{M} \oplus \mathcal{M}^\perp$. Then for
    all $x \in X$, $x = m + m^\prime$ for unique $m \in \mathcal{
    M}, m^\prime \in \mathcal{M}^\perp$. Then $\pi(x) = m^\prime +
    \mathcal{M}$ for $\pi: X \to X/\mathcal{M}$. Moreover
    \begin{align*}
      \|x\| = \|m\| + \|m^\prime\| \quad \textrm{ and } \quad
      \|\pi(x)\| = \|m^\prime\|
    \end{align*}
    Thus we see that $\pi|_{\mathcal{M}^\perp}$ is isometric.
  \end{solution}

  \question
  \textcolor{red}{verify}
  \begin{solution}
    From what's given, it is evident that whenever $f_i \to 0$ weak
    *, $Tf_i \to 0$. Since the spaces $X, Y$ are linear this gives us that
    $T$ is weak * continuous. Since the closed unit ball,
    $\overline{B}$ is weak * compact by Banach-Alaoglu, and the
    continuity of $ T$ gives that $T(\overline{B})$ is compact.

  \end{solution}

  \question
  \label{q:6}
  \begin{solution}
    Without loss of generality, assume that $\|T\| \le 1$. Notice that
    $\overline{T(B_1)}$, the closure of the image of the unit ball is
    compact since $   T$ is a compact operator.
    Let $T(e_{i_n})$ be an arbitrary subsequence of $T(e_{i})$.
    Since $T(e_{i_n})$ is a sequence in a compact space, it has a
    convergent subsequence $T(e_{i_{n_k}})$.
    We claim $T(e_{i_{n_k}})$ converge to zero. Let $x = \lim_{k \to
    \infty} T(e_{i_{n_k}})$. Since the convergence is in norm, we see
    that $T(e_{i_{n_k}}) \to x$ weakly.

    Now, for any $x \in \mathcal{H}$,
    \begin{align*}
      \|x\|^2 \ge \sum_{n \in \mathbb{N}} \langle x , e_n \rangle
    \end{align*}
    Hence $\langle  e_n , x \rangle  \to 0$ for any $x \in
    \mathcal{H}$. Thus we see that $e_n \to 0$, weakly. Specifically,
    we see that
    \begin{align*}
      \langle Te_n , y \rangle  = \langle e_n , T^*y \rangle \to 0
    \end{align*}
    for any $y \in \mathcal{H}$. Shows that $T(e_n) \to 0$ weakly.
    Since weak topology is Hausdorff, we see that $x = 0$.

    Since we have shown that any arbitrary subsequence of $T(e_n)$
    has a subsequence that converge to $0$, we get that $T(e_j) \to
    0$. Hence we are done.
  \end{solution}

  \question
  \begin{solution}
    \begin{parts}
      \part $(1 \implies 2)$ If $T$ is an isometry, then expanding
      $\langle  Tx , Ty \rangle  = \langle x , y \rangle$ for some
      $x, y \in \mathcal{H}$, we get
      \begin{align*}
        \langle Tx , Ty \rangle  + \langle Ty , Tx \rangle  = \langle
        x , y \rangle  + \langle y , x \rangle
      \end{align*}
      which gives $\Re \langle x , y \rangle = \Re \langle Tx , Ty
      \rangle$. Now replace $x$ with $ix$ to get $\Im \langle Tx , Ty
      \rangle  = \Im \langle x , y \rangle $. Since real and
      imaginary parts are equal, we see that $\langle Tx , Ty \rangle
      = \langle x , y \rangle $
      \part $(2 \implies 3)$ If $\langle Tx , Ty \rangle  = \langle x
      , y \rangle $ for all $x, y \in \mathcal{H}$, then $\langle
      T^*T x , y \rangle = \langle x , y \rangle $, which implies
      $\langle (T^*T - I) x , y \rangle = 0$ for all $x , y \in
      \mathcal{H}$. Then Reisz Representation theorem shows that if
      $y \neq 0$, we must have $T^*T - I = 0$.
      \part $(3 \implies 2)$ If $T^*T = I$, then
      \begin{align*}
        \|Tx\|^2 &= \langle Tx , Tx \rangle  \\
        &= \langle T^*T x , x \rangle  \\
        &= \langle x , x \rangle  \\
        &= \|x\|^2
      \end{align*}
      which shows that $T$ is an isometry.
    \end{parts}
  \end{solution}

  \question
  \begin{solution}
    \begin{parts}
      \part $(1 \implies 3)$ If $T$ is normal isometry, we see that $
      TT^* = T^*T$, and previous question proves that $TT^* = T^*T = I$.
      \part ($3 \implies 2$) If $TT^* = T^*T = I$, then it is clear
      from the previous question that $T$ is an isometry. To see that
      it is a bijection, let $x \in H$, since $T(T^*(x)) = x$, we see
      that $   x \in T(\mathcal{H})$. Hence $T$ is a bijection.
      \part $(2 \implies 1)$. We just need to show normality of $T$.
      Since $T$ is given to be an isometric bijection, $T$ has an
      inverse, $P$. Since $T$ is bijective $P$ is also isometric and
      linear. To see linearity, notice that
      \begin{align*}
        P(x + y) = P(T(Px) + T(Py)) = P(T(Px + Py)) = Px + Py
      \end{align*}
      We claim $P = T^*$. To see this, note that
      \begin{align*}
        \langle PTx , y \rangle = \langle  x , y \rangle = \langle
        Tx , Ty \rangle = \langle T^*Tx , y \rangle
      \end{align*}
      Hence we see that $\langle  (PT - T^*T)x , y \rangle = 0 $ for
      al $ x, y \in \mathcal{H}$. Therefore by Reisz representation,
      we have $  PT - T^*T = (P-T^*)T = 0$. Since $T$ is bijective,
      this forces $P = T^*$ and we get the normality.
    \end{parts}
  \end{solution}

  \question
  \begin{solution}
    Let $x \in \textrm{Ker}(T)$. Then $0 = \langle Tx , y \rangle =
    \langle x , T^*y \rangle$ for any $y \in \mathcal{H}$, shows that
    $y \in T(\mathcal{H})^\perp$.

    Conversely if $x \in T^*(H)^\perp$. Then $0 = \langle x , T^*y
    \rangle = \langle Tx , y \rangle = $ for any $y \in \mathcal{H}$
    shows that $x \in \textrm{Ker}(T)$ by Reisz representation theorem.
  \end{solution}

  \question
  \begin{solution}
    Consider the map $T|_{\textrm{Ker}(T)^\perp}:
    \textrm{Ker}(T)^\perp \to T(H)$. Clearly the map is surjective and
    linear. Hence $ \textrm{Ker}(T)^\perp \cong T(H)$, which is
    finite dimensional. From the previous question, we know that
    $T^*(\mathcal{H}) = \textrm{Ker}(T)^\perp$. Hence we see that
    $T^*$ is finite rank.
  \end{solution}

  \question
  \begin{solution}
    \begin{align*}
      \|Tf\|^2 &= \int |xf(x)|^2 \ d \mu \\
      &\le \int |f|^2 \ d \mu \\
      &= \|f\|^2
    \end{align*}
    shows that $T$ is a contraction. Hence $T \in B(L^2[0, 1])$.
  \end{solution}
  Moreover for any $f, g \in L^2([0, 1])$, we get
  \begin{align*}
    \langle Tf , g \rangle  = \int |xf(x)\bar{g}(x)|^2 \ d \mu =
    \int |f(x)  \bar{x}g(x)|^2 \ d \mu = \langle f , Tg \rangle
  \end{align*}
  Hence $T^* =T$. To prove injectivity of $T$, assume $T(f) = T(g)$, then
  \begin{align*}
    0 = (T(f) - T(g))(x) = x(f(x) - g(x))
  \end{align*}
  forces $f = g$. (Note that the equalities above is almost
  everywhere). To see that $T$ is not surjective, we claim that there
  $T(f) \neq \chi_{[0, 1]} \in L^2([0, 1])$ for any $f \in L^2([0,
  1])$. If such $f$ exist, then $f(x) = \frac{\chi_{[0, 1]}(x)}{x}
  \not \in L^2([0, 1])$.

  For the sake of contradiction, assume that $\lambda \in \mathbb{C}$
  such that $Tf = \lambda f$ for some $f \in L^2([0, 1])$. Then
  we must have
  \begin{align*}
    x f(x) = T(f)(x) = \lambda f(x)
  \end{align*}
  almost everywhere. This forces $x = \lambda$ or $f = 0$ almost
  everywhere. Since $x$ cannot be equal to $\lambda$ except possibly
  only at $x = \lambda$ (measure zero set), we see that $f = 0$
  almost everywhere. Thus $\lambda$ cannot be an eigenvector of $T$.

  \question
  \begin{solution}
    \begin{parts}
      \part Let $x = (x_n) \in \ell^2$. Then
      \begin{align*}
        \|T(x)\|_2^2 &= \sum_{n \in \mathbb{N}} |\alpha_n x_n|^2 \\
        &\le \|(\alpha_n)\|_\infty^2 \sum_{n \in \mathbb{N}} |x_n|^2 \\
        &= \|(\alpha_n)\|_\infty \|(x_n)\|_2^2
      \end{align*}
      shows that $\|T\| \le \|(\alpha_n)\|_\infty$. Moreover since
      each $\|\delta_n\| = 1$, and
      \begin{align*}
        \|T(\delta_n)\| = |\alpha_n|
      \end{align*}
      taking supremum over $n$, we see $T$ attains the norm
      $\|(\alpha_n)\|_\infty$.
      \part One direction is the direct application of question
      \ref{q:6}. Conversely if $(\alpha_n) \in c_\textbf{0}$, then
      for any open cover $U$ which cover $T(B)$, we can find a finite
      subcover by first taking an element which cover $0$, then there
      can only by at most finite $T(\delta_n)$ outside that open ball.
      Now by the fact that finite dimensional closed unit balls are
      compact, we get compactness of $T$.

      \part Notice that if $x = (x_n), y = (y_n) \in \ell^2$, then
      \begin{align*}
        x =  \sum_{n \in \mathbb{N}} x_n \delta_n, \quad
        y =  \sum_{n \in \mathbb{N}} y_n \delta_n
      \end{align*}
      and
      \begin{align*}
        \langle Tx , y \rangle  = \sum_{n \in \mathbb{N}} \alpha_n
        x_n \overline{y_n} = \sum_{n \in \mathbb{N}} x_n \overline{\alpha_n y_n}
      \end{align*}
      Since this is true for all $x, y \in \ell^2$, by the uniqueness
      of the adjoint, we get our assertion.
    \end{parts}
  \end{solution}

  \question
  \begin{solution}
    \begin{parts}
      \part If $M$ is invariant under $T$, then $T(M) \subset M$,
      This shows $T(m) = P_MT(m)$ for all $m \in M$. Thus $TP_M =
      P_MTP_M$. Conversely if $TP_M = P_MTP_M$, then $T(m) = TP_M(m)
      = P_MTP_M(m) = P_MT(m)$ for all $m \in M$, shows that $T(m)
      \subset M$ for all $m \in M$. Thus $M$ is invariant under $T$.

      \part If $M$ reduces $T$, then $M$ and $M^\perp$ is invariant
      under $T$. Thus we see that for $x = m + m^\prime$ for $m \in
      M, m^\prime \in M^\perp$,
      \begin{align*}
        P_MT(x) = P_M(T(m) + T( m^\prime)) = P_M(T(m)) = T(m) = T(P_M(x))
      \end{align*}
      Thus $P_MT = TP_M$.

      Conversely, if $P_MT = TP_M$, then for $m \in M$, we get
      \begin{align*}
        P_MT(m) = TP_M(m) = T(m)
      \end{align*}
      which shows $T(m) \in M$ and for $m^\prime \in M^\perp$, we get
      \begin{align*}
        P_MT(m^\prime) = TP_M(m^\prime) = T(0) = 0
      \end{align*}
      Hence $T(m^\prime) \perp M$ which implies $T(m^\prime) \in M^\perp$.
      Thus we see that $M$ reduces $T$.

      \part If $M$ reduces $T$, then it is clear that $M$ is
      invariant under $T$.
      Let $m \in M$. Then for $x \in M^\perp$, we get
      \begin{align*}
        \langle T^*m , x \rangle = \langle m , Tx \rangle  = 0
      \end{align*}
      since $Tx \in M^\perp$. Thus $T^*(m) \in M$.

      \part If $M$ reduces $T$ Then $\mathcal{H} = M \oplus M^\perp$
      and $TP_M = P_MT$. Notice also that $T|_{M} = TP_M$. Then
      \begin{align*}
        P_MT^* = P_M^*T^* = (TP_M)^*
      \end{align*}
      Moreover, we know $M$ reduces $T^*$ also. Hence we get $P_MT^*
      = T^*P_M$. Thus we get
      \begin{align*}
        (T|_M)^* = (TP_M)^* = P_MT^* = T^*P_M = T^*|_M
      \end{align*}

      \part No. Let $T: \mathbb{C}^2 \to \mathbb{C}^2$ be represented by
      \begin{align*}
        T =
        \begin{pmatrix}%{c c}
          1 & 1 \\
          0 & 1
        \end{pmatrix}
      \end{align*}
      Now take $M = \textrm{span}(e_1)$. The fact is evident.
    \end{parts}
  \end{solution}

  \question
  \textcolor{red}{not finished}
  \begin{solution}
    \begin{parts}
      \part Given $P, Q$ are projections, if $P + Q$ is a projection,
      then
      \begin{align*}
        P + Q = (P + Q)^2 = P^2 + PQ + QP + Q^2 = P + QP + PQ + Q
      \end{align*}
      shows that $PQ = - QP$. Then for any $x, y \in \mathcal{H}$,
      \textcolor{red}{verify}

      Conversely if $P(\mathcal{H}) \perp Q(\mathcal{H})$, then for
      all $x, y \in \mathcal{H}$, we get
      \begin{align*}
        \langle x , PQy \rangle = \langle Px , Qy \rangle  = \langle
        QPx , y \rangle = 0
      \end{align*}
      Then Reisz representation theorem forces $PQ = QP = 0$, thus we see
      \begin{align*}
        (P + Q)^2 = P^2 + PQ + QP + Q^2 = P + QP + PQ + Q = P + Q
      \end{align*}
      Also $(P + Q)^* = P^* + Q^* =  P + Q$. Hence we see that $P+ Q$
      is a projection.

      If this happens, then it is clear that $\textrm{Ker}(P) \cap
      \textrm{Ker}(Q) \subset \textrm{Ker}(P+Q)$. Conversely let
      $(P+Q)(y) = 0$, then $P(y) = -Q(y)$. But since $P(\mathcal{H})
      \perp Q(\mathcal{H})$, this forces $P(y) = Q(y) = 0$. Hence
      $\textrm{Ker}(P + Q) = \textrm{Ker}(P) \cap \textrm{Ker}(Q)$.

      Moreover since $P(\mathcal{H}) \perp Q(\mathcal{H})$, we
      immediately see that $(P+Q)(\mathcal{H}) = P(\mathcal{H})\oplus
      Q(\mathcal{H})$

      \part
      \begin{itemize}[]
        \item
          If $PQ$ is a projection, we must have $(PQ)^* = PQ$. Then
          for all $x, y \in \mathcal{H}$,
          \begin{align*}
            \langle PQ(x) ,  y \rangle = \langle Qx , Py \rangle =
            \langle x , QPy \rangle
          \end{align*}
          Then uniqueness of the adjoint forces $PQ = QP$.
        \item
          If $PQ = QP$, then
          \begin{align*}
            (P + Q - QP)^2 &= P^2 + Q^2 + QPQP + PQ + QP - PQP - QQP
            - QPP - QPQ \\
            &= P + Q + QQPP + QP + QP - QPP - QQP - QPP - QQP \\
            &= P + Q + QP + QP + QP - QP - QP - QP - QP \\
            &= P + Q - QP
          \end{align*}
          and $(P + Q - QP)^* = P^* + Q^* - P^*Q^* = P + Q - PQ = P + Q - QP$.
          shows that $P + Q - QP$ is a projection.
        \item
          If $P + Q - QP$ is a projection, then
          \begin{align*}
            P + Q - QP = (P + Q - QP)^* = P^* + Q^* - P^*Q^* = P + Q - PQ
          \end{align*}
          forces $QP = PQ$
      \end{itemize}

      If the above happens, then for any $x \in \mathcal{H}$, $PQ(x) =
      QP(x)$ forces that $PQ(x) \in P(\mathcal{H})$ and $QP(x) \in
      Q(\mathcal{H})$. Hence we see that $PQ(\mathcal{H}) \subset
      P(\mathcal{H}) \cap Q(\mathcal{H})$. Conversely, if $y \in
      P(\mathcal{H}) \cap Q(\mathcal{H})$, then for some $a, b \in
      \mathcal{H}$, $P(a) = y = Q(b)$. Then $Q(P(a)) = Q(y) = Q^2(b)
      = Q(b) = y$ and $P(Q(b)) = P(y) = P^2(a) = P(a) = y$ shows that
      $y \in PQ(\mathcal{H})$. Hence we see that $P(\mathcal{H}) \cap
      Q(\mathcal{H}) = PQ(\mathcal{H})$.

      Similarly, if $y \in \textrm{Ker}(P), z \in \textrm{Ker}(Q)$,
      then since $QP = PQ$, we get
      \begin{align*}
        (P+Q)(y + z) &= (P+Q)^2(y + z) \\
        &= (P+Q)(P(z) + Q(y)) \\
        &= P(z) + Q(y) + QP(z) + PQ(y) \\
        &= P(z) + Q(y)
      \end{align*}
      \textcolor{red}{verify}
    \end{parts}
  \end{solution}

\end{questions}
\printbibliography[heading=bibintoc]
\end{document}



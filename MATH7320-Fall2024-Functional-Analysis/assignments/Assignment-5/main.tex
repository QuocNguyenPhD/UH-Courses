% initial settings
\documentclass[12pt]{exam}
\usepackage{geometry}
\usepackage{graphicx}
\usepackage{enumitem}
\usepackage[usenames,dvipsnames]{xcolor}
\usepackage[backend=biber, style=alphabetic]{biblatex}
\usepackage{url,hyperref}

\usepackage{amsmath} % math symbols, matrices, cases, trig functions,
% var-greek symbols.
\usepackage{amsfonts} % mathbb, mathfrak, large sum and product symbols.
\usepackage{amssymb} % extended list of math symbols from AMS.
% https://ctan.math.washington.edu/tex-archive/fonts/amsfonts/doc/amssymb.pdf
\usepackage{amsthm} % theorem styling.
\usepackage{mathrsfs} % mathscr fonts.
\usepackage{yhmath} % widehat.
\usepackage{empheq} % emphasize equations, extending 'amsmath' and 'mathtools'.
\usepackage{bm} % simplified bold math. Do \bm{math-equations-here}

% geometry of paper
\geometry{
  a4paper, % 'a4paper', 'c5paper', 'letterpaper', 'legalpaper'
  asymmetric, % don't swap margins in left and right pages. as
  % opposed to 'twoside'
  centering, % to center the content between margins
  bindingoffset=0cm,
}

% hyprlink settings
\hypersetup{
  colorlinks = true,
  linkcolor = {red!60!black},
  anchorcolor = red,
  citecolor = {green!50!black},
  urlcolor = magenta,
}

% theorem styles
\theoremstyle{plain} % default; italic text, extra space above and below
\newtheorem{theorem}{Theorem}[section]
\newtheorem{proposition}{Proposition}[section]
\newtheorem{lemma}{Lemma}[section]
\newtheorem{corollary}{Corollary}[theorem]

\theoremstyle{definition} % upright text, extra space above and below
\newtheorem{definition}{Definition}[section]
\newtheorem{example}{Example}[section]

\theoremstyle{remark} % upright text, no extra space above or below
\newtheorem{remark}{Remark}[section]
\newtheorem*{note}{Note} %'Notes' in italics and without counter

% renewcommands for counters
\newcommand{\propositionautorefname}{Proposition}
\newcommand{\definitionautorefname}{Definition}
\newcommand{\lemmaautorefname}{Lemma}
\newcommand{\remarkautorefname}{Remark}
\newcommand{\exampleautorefname}{Example}

\addbibresource{articles.bib}

\begin{document}

\title{MATH7320 - Functional Analysis \\ Homework 5}

% author list
\author{
  Joel Sleeba \\
}

\maketitle
\printanswers
\unframedsolutions

\begin{questions}

  \question
  \textcolor{red}{not finished}
  \begin{solution}

  \end{solution}

  \question
  \textcolor{red}{not finished}
  \begin{solution}
    Since $S \perp S^\perp$, clearly $S \subset (S^\perp)^\perp$.
    Moreover, we know that $(S^\perp)^\perp =
    \textrm{Ker}(P_{S^\perp})$. Therefore $(S^\perp)^\perp$ is a
    closed subspace. Hence $\overline{\textrm{span}}(S) \subset
    (S^\perp)^\perp$. Conversely if $x \in (S^\perp)^\perp$, then $x
    \perp S^\perp$
  \end{solution}

  \question
  \textcolor{red}{not finished}
  \begin{solution}
    Let $T_n \in \mathcal{K}(\mathcal{X}, \mathcal{Y})$ be a Cauchy sequence.
  \end{solution}

  \question
  \begin{solution}
    Since we know that $I = P_{\mathcal{M}} + P_{\mathcal{M}^\perp}$,
    we see that $X = \mathcal{M} \oplus \mathcal{M}^\perp$. Then for
    all $x \in X$, $x = m + m^\prime$ for unique $m \in \mathcal{
    M}, m^\prime \in \mathcal{M}^\perp$. Then $\pi(x) = m^\prime +
    \mathcal{M}$ for $\pi: X \to X/\mathcal{M}$. Moreover
    \begin{align*}
      \|x\| = \|m\| + \|m^\prime\| \quad \textrm{ and } \quad
      \|\pi(x)\| = \|m^\prime\|
    \end{align*}
    Thus we see that $\pi|_{\mathcal{M}^\perp}$ is isometric.
  \end{solution}

  \question
  \begin{solution}

  \end{solution}

  \question
  \begin{solution}

  \end{solution}

  \question
  \begin{solution}
    \begin{parts}
      \part $(1 \implies 2)$ If $T$ is an isometry, then expanding
      $\langle  Tx , Ty \rangle  = \langle x , y \rangle$ for some
      $x, y \in \mathcal{H}$, we get
      \begin{align*}
        \langle Tx , Ty \rangle  + \langle Ty , Tx \rangle  = \langle
        x , y \rangle  + \langle y , x \rangle
      \end{align*}
      which gives $\Re \langle x , y \rangle = \Re \langle Tx , Ty
      \rangle$. Now replace $x$ with $ix$ to get $\Im \langle Tx , Ty
      \rangle  = \Im \langle x , y \rangle $. Since real and
      imaginary parts are equal, we see that $\langle Tx , Ty \rangle
      = \langle x , y \rangle $
      \part $(2 \implies 3)$ If $\langle Tx , Ty \rangle  = \langle x
      , y \rangle $ for all $x, y \in \mathcal{H}$, then $\langle
      T^*T x , y \rangle = \langle x , y \rangle $, which implies
      $\langle (T^*T - I) x , y \rangle = 0$ for all $x , y \in
      \mathcal{H}$. Then Reisz Representation theorem shows that if
      $y \neq 0$, we must have $T^*T - I = 0$.
      \part $(3 \implies 2)$ If $T^*T = I$, then
      \begin{align*}
        \|Tx\|^2 &= \langle Tx , Tx \rangle  \\
        &= \langle T^*T x , x \rangle  \\
        &= \langle x , x \rangle  \\
        &= \|x\|^2
      \end{align*}
      which shows that $T$ is an isometry.
    \end{parts}
  \end{solution}

  \question
  \begin{solution}
    \begin{parts}
      \part $(1 \implies 3)$ If $T$ is normal isometry, we see that $
      TT^* = T^*T$, and previous question proves that $TT^* = T^*T = I$.
      \part ($3 \implies 2$) If $TT^* = T^*T = I$, then it is clear
      from the previous question that $T$ is an isometry. To see that
      it is a bijection, let $x \in H$, since $T(T^*(x)) = x$, we see
      that $   x \in T(\mathcal{H})$. Hence $T$ is a bijection.
      \part $(2 \implies 1)$. We just need to show normality of $T$.
      Since $T$ is given to be an isometric bijection, $T$ has an
      inverse, $P$. Since $T$ is bijective $P$ is also isometric and
      linear. To see linearity, notice that
      \begin{align*}
        P(x + y) = P(T(Px) + T(Py)) = P(T(Px + Py)) = Px + Py
      \end{align*}
      We claim $P = T^*$. To see this, note that
      \begin{align*}
        \langle PTx , y \rangle = \langle  x , y \rangle = \langle
        Tx , Ty \rangle = \langle T^*Tx , y \rangle
      \end{align*}
      Hence we see that $\langle  (PT - T^*T)x , y \rangle = 0 $ for
      al $ x, y \in \mathcal{H}$. Therefore by Reisz representation,
      we have $  PT - T^*T = (P-T^*)T = 0$. Since $T$ is bijective,
      this forces $P = T^*$ and we get the normality.
    \end{parts}
  \end{solution}

  \question
  \begin{solution}
    Let $x \in \textrm{Ker}(T)$. Then $0 = \langle Tx , y \rangle =
    \langle x , T^*y \rangle$ for any $y \in \mathcal{H}$, shows that
    $y \in T(\mathcal{H})^\perp$.

    Conversely if $x \in T^*(H)^\perp$. Then $0 = \langle x , T^*y
    \rangle = \langle Tx , y \rangle = $ for any $y \in \mathcal{H}$
    shows that $x \in \textrm{Ker}(T)$ by Reisz representation theorem.
  \end{solution}

  \question
  \textcolor{red}{not finished}
  \begin{solution}

  \end{solution}

  \question
  \textcolor{red}{not finished}

  \begin{solution}

  \end{solution}

  \question
  \textcolor{red}{not finished}

  \begin{solution}

  \end{solution}

  \question
  \textcolor{red}{not finished}
  \begin{solution}
    \begin{parts}
      \part If $M$ is invariant under $T$, then $T(M) \subset M$,
      This shows $T(m) = P_MT(m)$ for all $m \in M$. Thus $TP_M =
      P_MTP_M$. Conversely if $TP_M = P_MTP_M$, then $T(m) = TP_M(m)
      = P_MTP_M(m) = P_MT(m)$ for all $m \in M$, shows that $T(m)
      \subset M$ for all $m \in M$. Thus $M$ is invariant under $T$.

      \part If $M$ reduces $T$, then $M$ and $M^\perp$ is invariant
      under $T$. Thus we see that for $x = m + m^\prime$ for $m \in
      M, m^\prime \in M^\perp$,
      \begin{align*}
        P_MT(x) = P_M(T(m) + T( m^\prime)) = P_M(T(m)) = T(m) = T(P_M(x))
      \end{align*}
      Thus $P_MT = TP_M$.

      Conversely, if $P_MT = TP_M$, then for $m \in M$, we get
      \begin{align*}
        P_MT(m) = TP_M(m) = T(m)
      \end{align*}
      which shows $T(m) \in M$ and for $m^\prime \in M^\perp$, we get
      \begin{align*}
        P_MT(m^\prime) = TP_M(m^\prime) = T(0) = 0
      \end{align*}
      Hence $T(m^\prime) \perp M$ which implies $T(m^\prime) \in M^\perp$.
      Thus we see that $M$ reduces $T$.

      \part If $M$ reduces $T$, then it is clear that $M$ is
      invariant under $T$.
      Moreover from before, we see that $TP_M =
      P_MT$, then
      \begin{align*}
        \langle P_MT(x) ,  y \rangle = \langle TP_M(x) ,  y \rangle
        = \langle P_M(x) ,  T^*(y) \rangle
      \end{align*}

      \part

      \part
    \end{parts}

  \end{solution}

\end{questions}
\printbibliography[heading=bibintoc]
\end{document}



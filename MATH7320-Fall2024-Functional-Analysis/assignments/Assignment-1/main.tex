% initial settings
\documentclass[12pt]{exam}
\usepackage{geometry}
\usepackage{graphicx}
\usepackage{enumitem}
\usepackage[usenames,dvipsnames]{xcolor}
\usepackage[backend=biber, style=alphabetic]{biblatex}
\usepackage{url,hyperref}

\usepackage{amsmath} % math symbols, matrices, cases, trig functions, var-greek symbols.
\usepackage{amsfonts} % mathbb, mathfrak, large sum and product symbols.
\usepackage{amssymb} % extended list of math symbols from AMS. https://ctan.math.washington.edu/tex-archive/fonts/amsfonts/doc/amssymb.pdf
\usepackage{amsthm} % theorem styling.
\usepackage{mathrsfs} % mathscr fonts.
\usepackage{yhmath} % widehat.
\usepackage{empheq} % emphasize equations, extending 'amsmath' and 'mathtools'.
\usepackage{bm} % simplified bold math. Do \bm{math-equations-here}

% geometry of paper
\geometry{
  a4paper, % 'a4paper', 'c5paper', 'letterpaper', 'legalpaper'
  asymmetric, % don't swap margins in left and right pages. as opposed to 'twoside'
  centering, % to center the content between margins
  bindingoffset=0cm,
}

% hyprlink settings
\hypersetup{
  colorlinks = true,
  linkcolor = {red!60!black},
  anchorcolor = red,
  citecolor = {green!50!black},
  urlcolor = magenta,
  }

% theorem styles
\theoremstyle{plain} % default; italic text, extra space above and below
\newtheorem{theorem}{Theorem}[section]
\newtheorem{proposition}{Proposition}[section]
\newtheorem{lemma}{Lemma}[section]
\newtheorem{corollary}{Corollary}[theorem]

\theoremstyle{definition} % upright text, extra space above and below
\newtheorem{definition}{Definition}[section]
\newtheorem{example}{Example}[section]

\theoremstyle{remark} % upright text, no extra space above or below
\newtheorem{remark}{Remark}[section]
\newtheorem*{note}{Note} %'Notes' in italics and without counter 

% renewcommands for counters
\newcommand{\propositionautorefname}{Proposition}
\newcommand{\definitionautorefname}{Definition}
\newcommand{\lemmaautorefname}{Lemma}
\newcommand{\remarkautorefname}{Remark}
\newcommand{\exampleautorefname}{Example}

\addbibresource{articles.bib}


\begin{document}

\title{MATH 7320, Functional Analysis \\ Homework  I}

% author list
\author{
Joel Sleeba \\
}

\maketitle
\printanswers
\unframedsolutions

\begin{questions}
  
  \question
  \begin{solution}
    Minkowski's inequality states that if $a = (a_n)_{n \in \mathbb{N}}, b = (b_n)_{n \in \mathbb{N}}$ are elements of $\ell_p$, then $\|a+b\|_p \le \|a\|_p + \|b\|_p$. To prove this, we'll use the generalized Young's inequality which states that
  \end{solution}

  \question
  \begin{solution}
    
  \end{solution}

  \question
  
  \begin{solution}
    
  \end{solution}

  \question
  
  \begin{solution}
    
  \end{solution}

  \question
    Show that a normed space $\chi$ is a Banach space if and only if whenever $(x_n)$ is a sequence with $\sum_{n \in \mathbb{N}} \|x_n\| \le \infty$ implies $\sum_{n \in \mathbb{N}} x_n$ converges.
  \begin{solution}
    $(\implies)$. Assume that $\chi$ is a Banach space and $(x_n)$ is a sequence with $\sum_{n \in \mathbb{N}} \|x_n\| \le \infty$.

    Consider the sequence $s_n = \sum_{i = 1}^{n} x_i$. Then \[
         \|s_n - s_m\| = \Big\|\sum_{i = n}^{m} x_i\Big\| \le \sum_{i = n}^{m} \|x_i\|
    \]
    Since we know that $ \sum_{n \in \mathbb{N}} \|x_i\| \le \infty$, for any given $ \epsilon \ge 0$, there exists an $N_\epsilon \in \mathbb{N}$ such that for all $ m, n \ge N_\epsilon$, $\|s_n - s_m\| < \epsilon$. This implies $s_n$ is a Cauchy sequence in $\chi$. Now since by assumption we know that the space is complete, hence $s_n$ converges. This implies $ \sum_{n \in \mathbb{N}} x_i$ converges.

    $(\impliedby)$ Assume that $y_n$ is a Cauchy sequence in $\chi$. Then consider a subsequence $y_{n_k}$ with $\|y_{n_{k}} - y_{n_{k-1}}\| < \frac{\epsilon}{2^k}$. (This choice can be made by choosing $y_{n_k} = y_j$, where $j \ge N_{\frac{\epsilon}{2^k}}$). Now we construct another sequence $x_k$ from this such that $ x_1 = y_{n_1}$ and $x_k = y_{n_k} - y_{n_{k-1}}$. Then \[
        \Big \|\sum_{n \in \mathbb{N}} x_n \Big\| \le \sum_{n \in \mathbb{N}} \|x_n\| = \sum_{n \in \mathbb{N}} \|y_{n_k} - y_{n_{k-1}}\| \le \sum_{n \in \mathbb{N}} \frac{\epsilon}{2^k} = \epsilon
    \]
     and our assumption gives that $\sum_{k \in \mathbb{N}} x_k = \lim_{k \to \infty} y_{n_k}$ converges. Therefore since a subsequence of a Cauchy sequence converge, the original sequence must also converge to the same limit. Hence we get that $y_n$ also converges. Since $y_n$ was an arbitrary Cauchy sequence in $\chi$, this gives that every Cauchy sequence in $ \chi$ converges, completing the space.
  \end{solution}

  \question
  Show that $c_0$ with the sup norm is separable, while $\ell_\infty$ is not.
  \begin{solution}
    Let $e_i$ be the sequence with $i$-th entry 1, with the rest of the entries $0$. Now we claim the set $A = \{ \sum_{i \in \mathbb{N}} r_i e_i \ : \ r_i \in \mathbb{Q}, \textrm{ finitely many $r_i$s are non-zero }\}$ is a countable dense set for $c_0$. $A$ is countable since it is indexed by $\otimes_{i \in \mathbb{N}} Q_i$, where $Q_i = \mathbb{Q}$. Clearly we see that each $e_i \in c_0$. Hence $A \subset c_0$. 

    Now let $x = (x_1, x_2, \ldots , x_n, 0, 0, \ldots)$ be a sequence in $c_0$. Then for any $\epsilon > 0$, consider $y_i \in \mathbb{Q}$ such that $|x_i - y_i| \le \frac{\epsilon}{2^i}$. Density of $\mathbb{Q}$ in $\mathbb{R}$ guarantees the existence of such $y_i$s. Then $y = (y_1, y_2, \ldots , y_n, 0, 0, \ldots) \in A$ and $\|x - y\|_\infty \le \epsilon$, which guarantees $A$ is dense in $c_0$. Hence $c_0$ is separable.

    Now for $I \subset \mathbb{N}$, let $e_I$ be defined as \[
      e_I(x) = \begin{cases}
        1, & x \in I \\ 
        0, & x \notin I
      \end{cases}
    \]
    we see that $\|e_I - e_J\|_\infty = 1$ for all $I \neq J$. Since the number of subsets of $\mathbb{N}$ is uncountable, the balls $B_I = B(e_I, \frac{1}{2})$ is an uncountable disjoint collection of open balls in $\ell_\infty$. Hence $\ell_\infty$ cannot have a countable dense subset.
  \end{solution}

  \question
  
  \begin{solution}
    
  \end{solution}

  \question
  Show that any infinite dimensional Banach Space cannot have a countable Hamel basis (Hint: Use Baire's category theorem)
  \begin{solution}
    Baire's category theorem states that in a complete metric space no open set can be constructed as a countable union of nowhere dense sets.
    
    Now assume $B$ is a Banach space (over $\mathbb{F} = \mathbb{C} \textrm{ or } \mathbb{R}$) with a countable Hamel basis $\beta = \{ b_1, b_2, \ldots \}$ with $|b_i| = 1$ (Even if the initial choice is not of norm 1, we can always normalize it). Consider the subsets $B_i = \{ rb_i \ : \ r \in \mathbb{F} \}$.
    Since every open ball $B(0, r)$ around $\textbf{0}$ must contain scalar multiples $ \frac{r}{2}b_i $ for all $ i \in \mathbb{N}$, we get that none of $B_i$s contains an open set. Therefore each of $B_i$ are nowhere dense with $B = \cup_{n = 1}^{\infty}B_i$. This contradicts the Baire's category theorem. Hence a Banach space cannot have a countable Hamel basis. 
    \textcolor{red}{Verify where are we using the completion of the space.}
  \end{solution}

  \question
  
  \begin{solution}
    
  \end{solution}

  \question
   Show that an orthonormal basis for an infinite dimensional Hilbert space cannot be a Hamel basis.
  \begin{solution}
    Let $E$ be an orthonormal basis for the Hilbert space $H$. We will find an $x \in H$ which cannot be written as a finite linear combination of elements in $E$, hence showing that $E$ is not a Hamel basis.

    Let $(e_i)_{ i \in \mathbb{N}} \subset E$ be a sequence of elements in $E$, and $a_n = \frac{1}{2^n}$. Consider the element \[
        x = \sum_{n \in \mathbb{N}} a_ne_n = \sum_{n \in \mathbb{N}} \frac{e_n}{2^n}
    \]
    $x \in H$, since $H$ is a complete space and $x$ is the limit of the sequence $s_n = \sum_{i = 1}^{n} a_ie_i$. Moreover we see that  \[
      \|x\|^2 = |\langle  x , x \rangle| = \sum_{n \in \mathbb{N}} a_n^2 = \sum_{n \in \mathbb{N}} \frac{1}{2^{2n}} = \sum_{n \in \mathbb{N}} \frac{1}{4^n} = \frac{\frac{1}{4}}{1- \frac{1}{4}} = \frac{1}{3}
    \]
    We claim that $x$ cannot be written as a finite linear combination of elements in $E$. On contrary assume that \[x = \sum_{j = 1}^{m} b_j e_j^\prime\]
    where $e_j^\prime \in E$. 
    Then by orthogonality we get that $\langle  e_i , e_j^\prime \rangle = 1$ if and only if $e_i = e_j^\prime$ and otherwise $0$. This implies $a_n = b_j$ if and only if $e_n = e_j^\prime$. Then we will get \[
      \textbf{0} = x - x = \sum_{n \in \mathbb{N}} a_ne_n - \sum_{j = 1}^{n} b_je_j^\prime = \sum_{\substack{n \in \mathbb{N}\\ a_n \not\in \{ b_1, b_2, \ldots , b_m \} }} a_n e_n
    \]
    which implies \[
      0 = \|\textbf{0}\|^2 = \sum_{\substack{n \in \mathbb{N} \\ a_n \notin \{ b_1, b_2, \ldots , b_m \}}} |a_n|^2
    \]
    This happens if and only if $a_n = 0$ for all $a_n \not\in \{ b_1, b_2, \ldots , b_m \}$ which contradicts our choice for $a_n$.
  \end{solution}



\end{questions}
\printbibliography[heading=bibintoc]
\end{document}

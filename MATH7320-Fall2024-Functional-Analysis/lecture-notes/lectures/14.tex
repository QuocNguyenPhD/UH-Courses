% TeX_root = ../main.tex

\chapter{}

\begin{definition}
  Recall that a complex inner product o a complex vector space is a
  map \[
    \langle \ , \  \rangle : X \times X \to \mathbb{C}
  \]
  such that
  \begin{enumerate}[label=(\arabic*)]
    \item $\langle x , x \rangle \ge 0$ for all $x \in X$
    \item $\langle  x , y \rangle  = \overline{\langle y , x \rangle }$
    \item $\langle \alpha x + z, y \rangle  = \alpha \langle x , y
      \rangle  + \langle z , y \rangle $
  \end{enumerate}
\end{definition}

Recall the norms induced by the inner product and the Cauchy-Schwarz inequality.

\begin{definition}
  Complete inner product spaces are called Hilbert spaces
\end{definition}

\begin{proposition}
  Let $X$ be an inner product space. Then the inner product of $X$
  extends to an inner product on the completion (unique metric space
  completion) of $X$, turns it into
  a Hilbert space.
\end{proposition}

\begin{definition}
  If $x, y \in H$, the Hilbert space, we say $x \perp y$ if $ \langle
  x , y \rangle  = 0$
\end{definition}

\begin{definition}
  Given a set $S \subset H$, $S^\perp = \{ y \in H  \ : \  \langle x
  , y \rangle  = 0 \}$
\end{definition}

\begin{proposition}
  Let $H, K$ be Hilbert spaces and $T: H \to K$ be linear. Then the
  following are equivalent.
  \begin{enumerate}[label=(\arabic*)]
    \item $T$ is isometry
    \item $\langle  Tx , Ty \rangle = \langle x , y \rangle $ for all
      $x, y \in H$
  \end{enumerate}
\end{proposition}
\begin{proof}
  See Homework-5
\end{proof}

\begin{proposition}
  For all $x, y \in H$, a Hilbert space, then \[
    \|x + y\|^2 + \|x-y\|^2  = 2(\|x\|^2 + \|y\|^2)
  \]
\end{proposition}

\begin{example}
  Show that $c_{00}$ under the usual inner product is not complete
  and its completion is $\ell^2(\mathbb{N})$.
\end{example}
\begin{proof}
  Consider the sequence $x_n = (1, \frac{1}{2}, \frac{1}{3}, \ldots,
    \frac{1}{n}, 0,
  \ldots)$. Then clearly $x_n \in \textbf{c}_0$. Moreover $x_n \to x
  = (1, \frac{1}{2}, \ldots \frac{1}{n}, \frac{1}{n+1}, \ldots)$ in
  the norm by the inner product. But $x \notin \textbf{c}_{00}$. But $ x
  \in \ell^2(\mathbb{N})$. Moreover, the same process can be used to
  approximate any  element in $\ell^2(\mathbb{N})$ by elements in
  $\textbf{c}_{00}$. Thus we see that $ \textbf{c}_{00}$ is dense in
  $\ell^{2}(\mathbb{N})$
\end{proof}

\begin{example}
  $L^2(\mathbb{R},  \mu)$ with \[
    \langle f , g \rangle  = \int_\mathbb{R}  f \overline{g} \ d \mu
  \]
  is a Hilbert space.
\end{example}

\begin{example}
  Let $J$ be any set $\ell^2(J) = \{  f: J \to \mathbb{C}  \ : \
  \sum_{j \in J} |f(j)|^2 < \infty  \}$ with the usual inner product
  is a Hilbert space.
\end{example}

\begin{definition}
  An orthonormal basis for $H$ is a maximal orthonormal set.
\end{definition}

\begin{theorem}
  Let $H$ be a Hilbert space and $J$ be an orthonormal basis for $H$.
  Then there exists a bijective linear isometry $T : H \to \ell^2(J)$.
\end{theorem}


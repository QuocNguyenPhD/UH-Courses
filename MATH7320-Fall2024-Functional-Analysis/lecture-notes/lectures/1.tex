% TeX_root = ../main.tex

\chapter{Banach Spaces}

\textbf{Textbook :} A Course in Functional Analysis, John Conway

Functional analysis is the study of Topological Vector Spaces.

\begin{definition}
  Let $X$ be a vector space (over $\mathbb{R}$ or $\mathbb{C}$). A
  seminorm on $X$ is a map $\|\cdot \|: X \to [0, \infty)$ such that
  \begin{itemize}
    \item  $\|\alpha x\| = |\alpha|\| x \|, \forall \alpha \in
      \mathbb{F}, \forall x \in X$
    \item $\|x+y\| \le \|x\| + \|y\|$
  \end{itemize}
  In addition if $\forall x\neq 0, \|x\| \neq 0$, we say $\|\cdot\|$
  is a norm on $X$
\end{definition}

Norm induces a metric $ d(x, y) = \|x-y\|$

\begin{note}
  Let $X$ be a normed space. Then the maps
  \begin{itemize}
    \item $+: X \times X \to X : (x, y) \to x+y$
    \item $\cdot: \mathbb{F} \times X \to X: (\alpha, x) \to \alpha x$
  \end{itemize}
  are continuous.
\end{note}

Hence every normed space is a topological vector space.

\begin{example}
  $\mathbb{F}^n$ with $\ell_p$-norm defined as \[
    \|(\alpha_1, \alpha_2, \ldots, \alpha_n)\|_p = \Big(\sum_{i =
    1}^{n} |a_i|^p\Big)^{ \frac{1}{p} }
  \]
\end{example}

\begin{example}
  $\mathbb{F}^n$ with $\ell_\infty$-norm defined as \[
    \|(\alpha_1, \alpha_2, \ldots, \alpha_n)\|_\infty = \max \{ |a_i| \}
  \]
\end{example}

\begin{example}
  Consider $C_{00} = \{ (a_n)_{n \in \mathbb{N}}  \ : \   a_n \in
    \mathbb{F}, \forall n \in \mathbb{N}, a_n = 0  \textrm{ except for
  finitely many }  n \in \mathbb{N}\}$ which can be identified by
  collection of functions $  f: \mathbb{N} \to \mathbb{F}$ with finite support.

  Then $$\|(a_n)\|_p = \Big(\sum_{n = 1}^{\infty} |a_n|^p\Big)^{ \frac{1}{p} }$$
  is a norm on $C_{00}$
\end{example}

\begin{proposition}
  Let $X, Y$ be normed space, and let $T: X \to Y$ be linear. Then
  the following are equivalent.
  \begin{itemize}
    \item $T$ is continuous
    \item $T$ is continuous on 0
    \item $T$ is continuous on any point $x \in X$
    \item $\exists M >0$ such that $\|T(x)\|_Y \le M \|x\|_X$ for all $x \in X$
  \end{itemize}
\end{proposition}
\begin{proof}
  ($1 \implies 2$) It is clear that if $T$ is continuous, then it is
  continuous at $0$ from the definition of continuity.

  $(2 \implies 3)$ Let $x \in X$ and $\{ x_n \}_{n \in \mathbb{N}}$
  be any sequence in $X$ that converge to $ x$. Then the sequence $\{
  y_n = x_n - x \}$ converge to zero by the algebra of limits. By the
  continuity of $T$ at zero, $\{ T(y_n) = T(x_n) - T(x) \}$ converge
  to $0$. Therefore $T(x_n) \to T(x)$. And this shows $T$ is
  sequentially continuous at $x \in X$. Since the space is a metric
  space, sequential continuity is equivalent to continuity.

  $(4 \implies 2)$ Let $x \in X$. Then $\|T(0) - T(x)\| = \|T(x)\|
  \le M \|x\| = M \|0 - x\|$. Hence $T$ is continuous at $0$.

  $(3 \implies 1)$

  $(2 \implies 4)$

\end{proof}

\begin{example}
  Let $T: \mathbb{F}^n \to \mathbb{F}^n$ be defined as $T(\alpha_1,
  \alpha_2, \ldots, \alpha_n) = (\alpha_1, 0, \ldots, 0)$. Is $T$
  convergent for any norm $ \|\cdot\|_1, \|\cdot\|_2$ in the domain and range?

  % No. Let $\|\cdot\|_1 = \|\cdot\|_$
\end{example}
\begin{proof}
  \textcolor{red}{verify}
\end{proof}

\begin{example}
  Consider identity function $I: C_{00} \to C_{00}$. Let the norm in
  domain be $ \|\cdot\|_\infty$ and that in range be $\|\cdot\|_1$.
  Is the function continuous? What if the norms in domain and range
  are switched?
\end{example}
\begin{proof}
  \textcolor{red}{verify}
\end{proof}

\begin{note}
  Let $X$ be a space with two norms $\|\cdot\|_1, \|\cdot\|_2$. When
  is the two norms topologically equivalent?

  When $\exists M, M^\prime$ such that $\|x\|_1 \le M \|x\|_2$ and
  $\|x\|_2 \le M^\prime \|x\|_1$
  Equivalently, when the identity map between the two spaces with
  their respective norms are bi-continuous. (See 4th equivalent
  statement of previous proposition)
\end{note}
\begin{theorem}
  Let $X$ and $Y$ be normed spaces, and $T: X \to Y$ be linear.
  Assume $X$ is finite dimensional. Then $T$ is continuous.
\end{theorem}
\begin{proof}
  Since $T(X) \le Y$ is finite dimensional, we may assume without
  loss of generality that $Y$ is also finite dimensional and $T$ is
  onto. Let $\{ x_1, x_2, \ldots x_n \}$ be a basis for $X$. Define
  another norm on $X$ as follows. For every $x  = \sum_{i = 1}^{n}
  \alpha_i x_i \in X$, \[
    \|x\|^\prime = \sum_{i = 1}^{n} |\alpha_i| (\|T(x_i)\| + \|x_i\|)
  \]
  \textcolor{red}{verify that this is a norm}. Then for every $x \in
  X$, we have \[
    \|T(x)\| \le \sum_{i = 1}^{n} |\alpha_i|\|T(x_i)\| \le \|x\|^\prime
  \]

  Hence $T$ is bound with respect to the norm $\|\cdot\|^\prime$ on
  $X$, since all norms are equivalent on $X$. Therefore $T$ is
  continuous w.r.t to the original norm on $X$.
\end{proof}

\begin{corollary}
  Let $X$ be a finite dimensional vector space. Then any two norms in
  $X$ are equivalent.
\end{corollary}
\begin{proof}
  Let $\{ e_1, e_2, \ldots e_n \}$ be a basis for $X$. For each $x =
  \sum_{i = 1}^{n} \alpha_i e_i \in X$, define \[
    \|x\|_\infty = \max \{ |\alpha_i| \}
  \]
  Then $\|\cdot\|_\infty$ is a norm and we'll show every norm on $X$
  is equivalent to this norm. Let $\|\cdot\|$ be an arbitrary norm on
  $X$. For each $x = \sum_{i = 1}^{n} \alpha_i e_i \in X$, we have
  \begin{align*}
    \|x\| &= \|\sum_{i = 1}^{n} \alpha_i e_i\| \\
    & \le \sum_{i = 1}^{n} |\alpha_i|\|e_i\| \\
    &\le \max \{ |\alpha_i| \} \sum_{i = 1}^{n} \|e_i\| \\
    & \le \|x\|_\infty \sum_{i = 1}^{n} \|e_i\|
  \end{align*}

  % Moreover, we have \[
  %   \|x\|_\infty \le \Big(\sum_{i = 1}^{\infty} |\alpha_i|^2\Big)^{
  % \frac{1}{2} } = \|x\|_2
  % \]
  Therefore the identity map $I: (X, \|\cdot\|_\infty) \to (X,
  \|\cdot\|)$ is continuous. Since the set $ K = \{ x \in X \ :
  \ \|x\|_\infty \le 1 \}$ is compact,  K is also compact in $(X,
  \|\cdot\|)$ and the restriction $\textrm{Id}|_K$ is also a
  homeomorphism. \textcolor{red}{verify}
  In particular, the set $\{ x \in X  \ : \  \|x\|_\infty < 1 \}$ is
  an open neighborhood of $0 \in (X, \|\cdot\|)$
  By the Heine-Borel theorem, the unit ball $B = \{ x \in X  \ : \
  \|x\|_2 \le 1 \}$ is compact. Hence $B$ is compact in $(X,
  \|\cdot\|)$. \textcolor{red}{verify the last line}.
\end{proof}


% TeX_root = ../main.tex

\marginnote{ \scriptsize 13/11/2024}

\begin{lemma}
  Every $T \in B(\mathcal{H})$ can be decomposed as a linear combination of two
  self-adjoint operators.
\end{lemma}
\begin{proof}
  \begin{align*}
    T = \frac{T+T^*}{2} + i \frac{(T-T^*)}{2i}
  \end{align*}
\end{proof}

\begin{lemma}
  $T \in B(\mathcal{H})$ is normal if and only if $T + T^*$ and $T
  - T^*$ commutes
\end{lemma}
\begin{proof}
  \begin{align*}
    (T \mp T^*)( T \pm T^*) = T^2 \pm T^*T \mp TT^* + T^{*2}
  \end{align*}
\end{proof}

\begin{lemma}
  $T \in B(\mathcal{H})$ is normal if and only if $T = T_1 + i T_2$
  where $T_{1}, T_{2} \in S(\mathcal{H})$ and $T_{1}T_{2} = T_{2}T_1$
\end{lemma}
\begin{proof}
  \textcolor{red}{Use above lemma}
\end{proof}

\begin{lemma}
  Let $T \in B(\mathcal{H}), \lambda \in \mathbb{C}$ be an eigenvalue
  of $T$, and let $ S \in B(\mathcal{H})$ with $ST = TS$. Then $
  \textrm{Ker}(T - \lambda I)$ is invariant under $S$
\end{lemma}
\begin{proof}
  \textcolor{red}{verify}
\end{proof}

\begin{lemma}
  Let $T \in B(\mathcal{H})$, be self-adjoint. Let $\lambda_1 \neq
  \lambda_2 \in \sigma(T)$ and $\xi_1, \xi_2 \in \mathcal{H}$ their
  corresponding eigenvectors, then $ \langle \xi_1 , \xi_2 \rangle = 0$
\end{lemma}

\begin{theorem}[Spectral theorem for compact normal operators]
  Let $\mathcal{H}$ be a separable Hilbert space and $T \in K(\mathcal{H})$ be
  normal. Then there is an orthonormal basis $ \{ e_n \}_{n \in
  \mathbb{N}}$ and a sequence $\{\alpha_n\} \in c_{\textbf{0}}$ such that
  \begin{align*}
    T e_n = \alpha_n e_n
  \end{align*}
  for all $n \in \mathbb{N}$.
\end{theorem}
\begin{proof}
  Let $T \in \mathcal{K}(\mathcal{H})$ be normal. So $\exists S_1,
  S_{2} \in \mathcal{K}(\mathcal{H})\cap S(  \mathcal{H})$ such that
  $T = S_1 + i S_2$ and $S_{1}S_{2} = S_{1}S_{2}$. Then by spectral
  theorem for self-adjoint operators, we get $\sigma(S_{1})$, the set
  of eigenvalues of $S_{1}$ such that
  \begin{align*}
    H = \bigoplus_{\lambda \in \sigma(S_{1})}\textrm{Ker}(S_{1} - \lambda I)
  \end{align*}
  where each $\textrm{Ker}(S_{1} - \lambda I)$ is finite dimensional.

  Since $S_2$ commutes with $S_{1}$, $ \textrm{Ker}(S_{1} - \lambda
  I)$ is invariant under $S_2$ and $S_2|_{\textrm{Ker}(S_{1} -
  \lambda I)}$ is self-adjoint and compact. Thus, by the first part
  of the proof, for each $\lambda \in \sigma(S_1)$, we can choose and
  orthonormal basis $E_\lambda$ for $\textrm{Ker}(S_1 - \lambda I )$
  consisting of eigenvectors of $S_2$.

  Observe that if $\xi \in \mathcal{H}$ is such that $S_{1} \xi
  =\lambda \xi$ and $S_2 \xi = \beta \xi$, then $S_2 \xi = \beta \xi$. Then
  \begin{align*}
    T \xi = (\lambda + i \beta) \xi
  \end{align*}

  Now let $E = \cup_{\lambda \in  \sigma(S_1)} E_\lambda$. Then $E$
  is an orthonormal basis for $\mathcal{H}$ consisting of eigenvectors of $T$.
\end{proof}

\begin{lemma}
  Let $V$ be a vector space and $f_1 , f_2 , \ldots , f_n: V \to
  \mathbb{C}$ linear. Then $ f \in \textrm{span} \{ f_1 , f_2 ,
  \ldots , f_n \}$ iff
  \begin{align*}
    \bigcap_{k = 1}^{n}\textrm{Ker}(f_k) \subset \textrm{Ker}(f)
  \end{align*}
\end{lemma}
\begin{proof}
  One part is easy
\end{proof}

\begin{lemma}
  \begin{align*}
    (X, \textrm{weak})^* = X^*
  \end{align*}
\end{lemma}



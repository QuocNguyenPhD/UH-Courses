% TeX_root = ../main.tex

\begin{proposition}
  $I-P_K = P_{K^\perp}$
\end{proposition}
\begin{proof}
  Let $x \in X$ and $k \in K$. Then
  \begin{align*}
    \langle (I - P_k)(x), k \rangle &= \langle  x - P_k(x) , k \rangle \\
    &= \langle  x , k \rangle - \langle P_k(x) ,  k \rangle  \\
    &= \langle x , k \rangle  - \langle x , P_k(k) \rangle \\
    &= \langle x , k \rangle  - \langle  x , k \rangle  \\
    &= 0
  \end{align*}
  Shows that
\end{proof}

\begin{proposition}
  Let $K$ be a closed subspace of $H$. Let $E \subset K$ be an orthonormal basis
  for $K$. Extend $E$ to an orthonormal basis $\tilde{E}$ for $H$. Then \[
    P_K|_E = I_K, \quad P_K|_{\tilde{E} \setminus E} = 0
  \]
\end{proposition}

\begin{remark}[Parserval's Inequality]
  Let $H$ be a Hilbert space. Let $E$ be an orthonormal set. Then for
  every vector $\eta \in H$, \[
    \|\eta\|^2 \ge \sum_{e \in E} |\langle \eta ,  e \rangle |^2
  \]
\end{remark}

\begin{lemma}
  Let $S$ be a nonempty subset of $H$. Then \[
    (S^\perp)^\perp  = \overline{\textrm{span}(S)}
  \]
\end{lemma}
\begin{proof}
  Notice from the above proposition that $\textrm{Ker}(P_K) =
  K^\perp$. Thus $(S^\perp)^\perp = \textrm{Ker}(P_{S^\perp})$. Since
  $P_{S^\perp}(s) = 0$ for all $s \in S$, we see that $S \subset
  \textrm{Ker}(P_{S^\perp})$. Moreover, since
  $\textrm{Ker}(P_{S^\perp})$ is a closed subspace, we see that
  $\overline{ \textrm{\textrm{span}}}(S) \subset
  \textrm{Ker}(P_{S^\perp}) = ( S^\perp)^\perp$.
\end{proof}

\begin{corollary}
  Let $E$ be an orthonormal subset of $H$. Then the following are equivalent.
  \begin{enumerate}[label=(\arabic*)]
    \item $E$ is an orthonormal basis
    \item $(E^\perp)^\perp = H$
    \item $\overline{\textrm{Span}(S)} = H$
    \item $\|\eta\|^2 = \sum_{e \in E} |\langle \eta ,  e \rangle
      |^2, \ \forall \eta \in H$
  \end{enumerate}
\end{corollary}
\begin{proof}
  \textcolor{red}{verify}
\end{proof}

\begin{proposition}
  Let $K$ be a closed subspace of a Hilbert space $H$. Then $P_K = P_K^*$
\end{proposition}
\begin{proof}
  Let $k \in K$ and $x \in H$. Then \[
    \langle x , P_K^*(k) \rangle = \langle P_K(x) , k \rangle =
    \langle  P_K(x)-x , k \rangle + \langle x , k \rangle = \langle x
    , k \rangle
  \]
  for all $x \in H$.
  Hence $P_K^*(k) = k$.
  Conversely, \textcolor{red}{verify}
\end{proof}

\begin{theorem}
  Let $P \in B(H)$ be an idempotent. Then the following are equivalent.
  \begin{enumerate}[label=(\arabic*)]
    \item $P = P_K$ for some closed subspace $K \leqslant H$
    \item $P = P^*$
    \item $\|P\| = 1$
  \end{enumerate}
\end{theorem}
\begin{proof}
  $1 \implies 2, 1 \implies 3$ is easily known from above.
  To see $2 \implies 1$. Let $K = \textrm{Im}(P)$. Let $\rho \in
  K^\perp$. Then for all $\eta \in H$, \[
    \langle P(  \rho) ,  \eta \rangle  = \langle \rho ,  P^*(\eta)
    \rangle  = \langle \rho ,  P(\eta) \rangle = 0
  \]
  So $P|_{K^\perp} = 0$. Hence $P = P_k$.

  Conversely, assume $P \neq P_K$. Then $\exists \rho \in K^\perp$
  such that $P(\rho) = 0$. For each $n \in \mathbb{N}$, we have
  $\|\rho + n P(\rho)\|^2 = \|\rho\|^2 + n^2 \|P(\rho)\|^2$. And \[
    \|P(\rho + n P(\rho))\|^2 = (n+1)^2 \|P(\rho)\|^2
  \]
  So for large $n$, we have \[
    \|P(\rho + n P(\rho))\| > \|\rho + P(\rho)\|
  \]
  so $\|P\| > 1$.
\end{proof}


% TeX_root = ../main.tex

\marginnote{ \scriptsize 12/11/2024}

\begin{lemma}
  If $T \in B(\mathcal{H})$ is self adjoint. Then every eigenvalue of
  $T$ is real.
  \label{SelfAdjointOperatorsHaveRealEigenvals}
\end{lemma}
\begin{proof}
  Let $\lambda \in \mathbb{C}$ be an eigenvalue of $T$ with
  eigenvector $v_\lambda$. Then
  \begin{align*}
    \lambda \|v\|^2 = \langle \lambda v_\lambda ,  v_\lambda \rangle
    = \langle T(v_\lambda) ,   v_\lambda \rangle  = \langle v_\lambda
    ,  T(v_\lambda) \rangle  = \langle  v_\lambda ,  \lambda
    v_\lambda \rangle = \overline{\lambda}\|v_\lambda\|^2
  \end{align*}
  forces $\lambda =  \overline{\lambda}$.
\end{proof}

\begin{example}
  Let $\alpha_n \in c_{\textbf{0}}$. Define $T: \ell^2 \to \ell^2$,
  $T(\delta_n) = \delta_n \delta_n$ for all $ n \in \mathbb{N}$. Then
  the set of eigenvalues of $T$ is precisely the set of $\alpha_n$s.
\end{example}

\begin{lemma}
  \label{lem:spectrum_norm_self_adj}
  If $T \in B(\mathcal{H})$ is self adjoint, then $\|T\| =
  \sup\{|\langle T \xi ,  \xi \rangle| \ : \ \xi \in \mathcal{H},
  \|\xi\| \le 1\}$
\end{lemma}
\begin{proof}
  $\langle T \xi ,  \xi \rangle \le \|T\|$ inequality is clear from
  Cauchy-Schwarz and definition of operator norm.
  To see the converse, let $\alpha, \beta \in \mathcal{H}$ with $
  \|\alpha\| = \|\beta\| = 1$. Then by
  the self-adjointness of $T$, we get
  \begin{align*}
    \langle T(\alpha + \beta) , \alpha + \beta \rangle  - \langle
    T(\alpha - \beta) ,   \alpha - \beta \rangle  &= 2 (\langle  T
    \alpha ,  \beta \rangle + \langle T \beta ,  \alpha \rangle  ) \\
    &= 4 \Re \langle T \alpha ,  \beta \rangle
  \end{align*}
  Now with enough rotation on $\beta$, we get $\langle  T \alpha ,
  \beta \rangle = \Re \langle T \alpha ,  \beta \rangle $. Moreover, if $A
  = \sup\{|\langle T \xi ,  \xi \rangle| \ : \ \xi \in \mathcal{H},
  \|\xi\| \le 1\}$, notice that $|\langle Tx , x \rangle | \le A
  \|x\|^2$. Then we get
  \begin{align*}
    4 |\langle T \alpha ,  \beta \rangle | & \le A \|\alpha +
    \beta\|^2 + A \|\alpha - \beta\|^2 \\
    &=2 A(\|\alpha\|^2 + \|\beta\|^2) \\
    &=4 A
  \end{align*}
  Thus we see that $\|T\| \le A$.
\end{proof}

\begin{lemma}
  \label{EigenSpaceisFiniteDim}
  Let $T \in \mathcal{K}(\mathcal{H})$ and $0 \neq \lambda \in \mathbb{C}$.
  Then the space
  \begin{align*}
    \textrm{Ker}(T - \lambda I)
  \end{align*}
  is finite dimensional.
\end{lemma}
\begin{proof}
  For the sake of contradiction, assume that $K = \textrm{Ker}(T -
  \lambda I)$ is not finite dimensional. Then $\overline{T(K \cap
  B_1^\mathcal{H})} = \lambda \overline{K \cap B_1^\mathcal{H}}$, is
  a closed non-compact
  (\autoref{ClosedUnitBallisCompactiffFiniteDim}) subspace of
  $\overline{ T(B_1^\mathcal{H})}$. This contradicts the compactness of $T$.
\end{proof}

\begin{theorem}[Spectral theorem for compact self-adjoint operators]
  \label{SpectralTheoremforCompactSAOperators}
  Let $\mathcal{H}$ be a separable Hilbert space and $T \in K(\mathcal{H})$ be
  self-adjoint. Then there is an orthonormal basis $ \{ e_n \}_{n \in
  \mathbb{N}}$ and a sequence $\{\alpha_n\} \in c_{\textbf{0}}$ such that
  \begin{align*}
    T e_n = \alpha_n e_n
  \end{align*}
  for all $n \in \mathbb{N}$.
\end{theorem}
\begin{proof}
  We claim that $T$ has an eigenvalue $\lambda \in \mathbb{R}$
  (\autoref{SelfAdjointOperatorsHaveRealEigenvals}) such
  that $|\lambda| = \|T\|$. By   \autoref{lem:spectrum_norm_self_adj}, $\exists
  (\xi_n)$ of unit vectors in $\mathcal{H}$ such that
  \begin{align*}
    |\langle T  \xi_n , \xi_n \rangle| \to \|T\|
  \end{align*}
  Hence by taking a subsequence $\xi_{n_k}$, we get
  \begin{align*}
    \langle T \xi_{n_k} ,  \xi_{n_k} \rangle \to \lambda
  \end{align*}
  where $\lambda = \pm \|T\|$. Since $T$ is compact, by passing to a
  subsequence, we may also assume $T \xi_{n_{k_j}} \to \eta \in \mathcal{H}$.
  Then we have
  \begin{align*}
    \|(T - \lambda I)\xi_{n_{k_j}}\|^2 &= \|T \xi_{n_{k_j}}\|^2 +
    \lambda^2 \|\xi_{n_{k_j}}\|^2 - 2 \Re \langle T \xi_{n_{k_j}}
    , \lambda \xi_{n_{k_j}}  \rangle  \\
    & \le \|T\|^2 + \lambda^2
  \end{align*}
  Now in the first equality above, since $2 \Re \langle T \xi_{n_{k_j}}
  ,\lambda\xi_{n_{k_j}}  \rangle \to 2\lambda^2$ as $n \to \infty$, we get
  that $T \xi_{n_{k_j}} - \lambda\xi_{n_{k_j}} \to 0$ and since $T \xi_{n_{k_j}}
  \to \eta$, by the uniqueness of the limit, we get $\lambda
  \xi_{n_{k_j}} \to \eta$ and thus applying $T$ to both sides, we get
  \begin{align*}
    \lambda T \xi_{n_{k_j}} \to  T \eta
  \end{align*}
  and hence we get $T \eta = \lambda \eta$. If $\eta= 0$, this will
  contradict our precious assumption that $T$ is nonzero since
  $|\langle T \xi_n , \xi_n \rangle | \to \|T\|$.

  Note that the kernel of $T - \lambda I$, $H_1$ is invariant under
  $T$ being the eigenspace, and has finite dimension
  (\autoref{EigenSpaceisFiniteDim}).
  Moreover, $T(H_1^\perp) \subset H_1^\perp$ by
  \autoref{InvariantSubspacesofSelfAdjointOperatorsReduce}.
  Therefore $T_1:= T|_{H_1^\perp}$ is
  self adjoint. Since $T$ is compact, its restriction $T_1$ is
  compact. Hence $T_1 \in \mathcal{K}(H_1^\perp)$. Applying the claim again it
  follows that $T_1$ has an eigenvalue $\lambda_1$ such that
  $|\lambda_1| = \|T_1\|$.

  Continuing this construction, we get $|\lambda| \ge |\lambda_1| \ge
  |\lambda_2| \ge \ldots$ such that the $\textrm{Ker}(T - \lambda_iI)
  \perp \textrm{ Ker}(T - \lambda_j I)$ for all $i \neq j$.

  For each $k$ (where $\lambda_k$ exist), choose an orthonormal basis
  $E_k$ for $\textrm{Ker}(T - \lambda_k I)$ and let $E = \cup E_k$.
  If $\mathcal{H}$ is finite dimensional, there is nothing new to
  prove. Hence assume $\mathcal{H}$ is infinite dimensional. If
  $\textrm{Ker}(T - \lambda_{n+1}I) = \big(\textrm{Ker}(T - \lambda
    I) \oplus \textrm{Ker}(T - \lambda_1 I ) \oplus \ldots \oplus
  \textrm{Ker}(T - \lambda_{n} I) \big)^\perp$, then the above lemma
  gives $\lambda_{n+1} = 0$. In this case $E$ conatins finitely many
  eigenvectors corresponding to nonzero eigenvalues and infinitely
  many vectors corresponding to the $0$ eigenvalue. Hence we are done
  in this case. This happens only when $T$ is finite rank.

  Otherwise, $T$ has infinitely man eigenvalues and therefore $E$ is
  an infinite set, hence by \autoref{CompactiffEigvalsconverge0},  the sequence
  $\lambda_n \to 0$.

  It remains to show that $E$ is an orthonormal basis for
  $\mathcal{H}$. Let $F$ be the collection of all eigenvectors of $T$
  and let $\mathcal{M} = \overline{ \textrm{span}}(F)$.
  Then $T(\mathcal{M}) \subset \mathcal{M}$, and so
  $T(\mathcal{M}^\perp) \subset \mathcal{M}^\perp$ by
  \autoref{InvariantSubspacesofSelfAdjointOperatorsReduce}. Thus
  $T|_{\mathcal{ M}^\perp}$ has no non-zero eigenvectors. Thus from
  what we have proved in the first part, $\mathcal{M}^\perp$ must be $\{ 0 \}$.
\end{proof}


% TeX_root = ../main.tex

\begin{lemma}
  If $T \in B(H)$ is compact, then $T(H)$ is separable.
\end{lemma}
\begin{proof}
  \textcolor{red}{verify}
\end{proof}

\begin{corollary}
  The set $K(H)$ of all compact operators on $  H$ is a closed
  two-sided ideal in $B(H)$.
\end{corollary}
\begin{proof}
  \textcolor{red}{verify}. Use the fact that compact operators are
  the closure of finite rank operators.
\end{proof}

\begin{corollary}
  $T \in K(H)$ implies $T^* \in K(H)$.
\end{corollary}
\begin{proof}
  \textcolor{red}{verify}. Use the fact that $T$ is finite rank
  implies $T^*$ is finite rank.
\end{proof}

\begin{example}
  Let \[
    T: L^2([0, 1], m) \to L^2([0, 1], m)
  \]
  be defined as $T(f)(x) = xf(x)$. Prove that $T = T^*$ and that $T$
  has no eigenvectors.

  Only for joseph: If $\xi = \eta$ almost everywhere, then show that  $f \xi = f
  \eta$ almost everywhere if $ f \in C([0, 1])$
\end{example}
\begin{proof}
  \textcolor{red}{Homework}
\end{proof}

\begin{example}
  Let $\alpha_n$ be a bounded sequence in $\mathbb{C}$. Consider $T :
  \ell^2 \to \ell^2$, such that \[
    T(\delta_n) = \alpha_n \delta_n
  \]
  for all $n \in \mathbb{N}$. Then $T$ is bounded with $\|T\| =
  \|(\alpha_n)\|_\infty$.
\end{example}

\begin{example}
  Prove that the $T$ above is compact if and only if $(\alpha_n) \in c_{\bf 0}$
  Prove that $T^*(\delta_n) = \overline{\alpha_n}\delta_n$
\end{example}
\begin{proof}
  \textcolor{red}{Homework}
\end{proof}

\begin{theorem}
  Let $H$ be separable Hilbert space and $T\in K(H)$ be normal. i.e
  $TT^* = T^*T$. Then there exist an orthonormal basis $\{ e_n  \ :
  \  n \in \mathbb{N} \}$ of $H$, and a sequence $(\alpha_n) \in
  c_{\bf 0}$, such that $T(e_n) = \alpha_n e_n$.
\end{theorem}

\begin{lemma}
  Let $T \in K(H)$ be self-adjoint. Then $\exists 0 \neq \eta \in
  \mathbb{C}$ such that $ \textrm{Ker}(T - \lambda I) \neq \{ 0 \}$.
\end{lemma}

\begin{lemma}
  If $T$ is a compact operator, and $0 \neq \lambda \in \mathbb{C}$,
  Then $\textrm{Ker}(T - \lambda I)$ if finite dimensional.
\end{lemma}

\begin{definition}
  Let $T \in B(H)$. A subspace $W \leqslant H$ is said to be
  invariant under $T$ if $T(W) \subset W$. We say $W$ reduces $T$, if
  $T(W) \subset W$ and $T(W^\perp) \subset W^\perp$
\end{definition}

\begin{example}
  Let $T \in B(H)$.
  Prove that $W \subset H$ is invariant under $T$ if and only if $P_WT = TP_W$
  Prove that $W$ reduces $T$ if and only if $P_WT = TP_W$ and
  $P_{W^\perp}T = T P_{W^\perp}$.
\end{example}
\begin{proof}
  \textcolor{red}{Homework}
\end{proof}


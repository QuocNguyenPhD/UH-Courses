% TeX_root = ../main.tex

\chapter{}

\begin{definition}
  A sublinear map is a functoin $\rho: X \to \mathbb{R}$ with the properites \begin{itemize}[]
    \item $\rho(rx) = r \rho(x), \forall r \in \mathbb{R}$
    \item $\rho(x+y) \le \rho(x)+\rho(y)$
  \end{itemize}
\end{definition}

\begin{definition}
  Let $X$ be a normed space. Then the dual of $X$, denoted by $ X^{*}$, is the space $B(X, \mathbb{F})$
\end{definition}

\begin{lemma}
  \label{lem:existence_linear_functions_with_norm_peaking_at_point}
  Let $X$ be a normed space and $x \in X$. Then $\exists f \in X^{*}$ such that \[
    \|f\| = 1 \ \textrm{and } \ f(x) = \|x\|
  \]
\end{lemma}
\begin{proof}
  Let $Z = \textrm{Span}\{ x \}$. Define $g: Z \to \mathbb{F}$ as $g(\alpha  x) = \alpha \|x\|$. Then $\|g\| = 1$. By the Hahn Banach theorem, $g$ has an extension $f$ which preserve the norm and extends $g$ to $X$.
\end{proof}

\begin{corollary}
  \label{cor:dual_maps_norm_elements}
   Let $X$ be a normed space and $x \in X$, then we have \[
     \|x\| = \sup \{ |f(x)| \ : \ f \in X^{*}, \|f\| \le 1 \}
   \]
\end{corollary}
\begin{proof}
  If $f$ is any linear functional with $\|f\| \le 1$, then $|f(x)| \le \|f\|\|x\| = \|x\|$. Hence $\|x\| \le \sup \{ |f(x)| \ : \ f \in X^{*}, \|f\| \le 1 \}$.
  Now let $f_x$ be the functional we get from \autoref{lem:existence_linear_functions_with_norm_peaking_at_point}. Then $f_x \in X^{*}$ and $\|f_x\| = 1$ with $f_x(x) = |f(x)| = \|x\|$. Hence we get that the inequality is actually an equality, and this proves the corollary.
\end{proof}

\begin{definition}
  For every $x \in X$, define a linear map $\hat{x}: X^{*} \to \mathbb{F}$ by $\hat{x}(f) = f(x)$
\end{definition}

\begin{theorem}
  For every $x \in X$, $\hat{ x } \in (X^{*})^{*}$. The map $\rho: x \to \hat{x}$ is an isometric linear map.
\end{theorem}
\begin{proof}
  The fact that $ \hat{x}$ is linear and bounded and the map $X \ni x \to \hat{x} \in X^{**}$ is linear follows from the definition of $f+g$ and $\lambda f$.

  By definition and   \autoref{cor:dual_maps_norm_elements}
  \begin{align*}
    \|\hat{x}\| &= \sup \{ |\hat{x}(f) \ : \ f \in X^{*}, \|f\| \le 1 \} \\ 
    & = \sup \{ |f(x)| \ : \ f \in X^{*}, \|f\| \le 1 \} \\ 
    & = \|f\|
  \end{align*}
\end{proof}

\begin{definition}
  A normed space $X$ is said to be reflexive if the map $\rho: X \to X^{**}:= x \to \hat{x}$ is surjective. (This is a stronger condition than $X \equiv X^{**}$)
\end{definition}

\begin{theorem}
  There are isometric isomorphisms between \begin{itemize}[]
    \item $(\textbf{c}_0)^*$ and $\ell^1$
    \item 
    \item $(\ell^1)^{*} \textrm{ and } \ell^\infty$
  \end{itemize} 
\end{theorem}
\begin{proof}
  \begin{itemize}[]
    \item Let $(x_{n}) \in \ell^1$. Then consider the map $\phi_{(x_n)}: \textbf{c}_0 \to \mathbb{F}$ defined as $$\phi_{(x_n)}: (y_n) \to \sum_{n \in \mathbb{N}} x_ny_n$$
      We claim that $\phi_{(x_n)}$ is a continuous linear functional. But first we should see that the sum is well defined. Since $y_n \to 0$, there is an $N \in \mathbb{N}$ such that $|y_n| < 1$ for all $n \ge N$. Since \[
        \Big|\sum_{i = N}^{\infty} x_ny_n \Big| \le \sum_{i = N}^{\infty}  |x_n||y_n| \le \|(x_n)\|_1
      \]
      we see that the sum is well defined and the map makes sense. Also since $(y_n)+(z_n) = (y_n +z_n) \in \textbf{c}_0$ whenever $(y_n), (z_n) \in \textbf{c}_0$, we get that \[
        \sum_{n \in \mathbb{N}} x_n (y_n + z_n) = \sum_{n \in \mathbb{N}} x_n y_n + \sum_{n \in \mathbb{N}} x_n z_n
      \]
      which shows the linearity  of the map $\phi_{(x_n)}$.

      Now we show that $\|\phi_{(x_n)}\| = \|(x_n)\|_1$. We immediately see that for $(y_n) \in c_0$ with $\|(y_n)\|_{\textrm{sup}} = \sup_{n \in \mathbb{N}} y_n = 1$, \[
        |\phi_{(x_n)}((y_n))| = \Big|\sum_{n \in \mathbb{N}} x_ny_n \Big| \le \|(y_n)\|_{\textrm{sup}} \bigg( \sum_{n \in \mathbb{N}} |x_n| \bigg) \le \|(x_n)\|_1
       \]
      which gives $\|\phi_{(x_n)}\| \le \|(x_n)\|_1$. Now let $\theta_j \in [0, 2\pi)$ such that $|x_j| = e^{i \theta_j}x_j$. Now consider the sequence $s_m \in \textbf{c}_0$ defined as $s_m = \sum_{j = 1}^{m} e^{i \theta_j}e_j$, where $e_j$ is  the sequence with $j$th entry $1$ and the rest of the entries $0$. 
      Since $(x_n) \in \ell_1$, for all $\epsilon \ge 0$ there exists an $N_\epsilon \in \mathbb{N}$ such that \[
        \sum_{i = N_\epsilon+1}^{\infty} |x_i| < \epsilon
      \]
      Then since \[
        |\phi_{(x_n)}(s_{N_\epsilon})| = \Big| \sum_{n = 1}^{N_\epsilon} e^{i \theta_j}x_n \Big| = \sum_{i = 1}^{N_\epsilon} |x_n| = \|(x_n)\| - \sum_{i = N_\epsilon +1}^{\infty} |x_n| \ge \|(x_n)\| - \epsilon
      \]
      and $\epsilon > 0$ was arbitrary, we get that $\|\phi_{(x_n)}\| = \|(x_{n})\|$

      Hence we see that the map $(x_n) \to \phi_{(x_n)}$ is an isometric linear map. Now for surjectivity, let $\phi \in \textbf{c}_0^{*}$. We claim that the sequence $(y_n) = (\phi(e_n)) \in \ell^1$ and $\phi = \phi_{(y_n)}$. Let $\theta_j \in [0, 2\pi)$ such that $e^{ i \theta_j}y_j = |y_j|$. Then for any $N \in \mathbb{N}$, we have \begin{align*}
        \sum_{j = 1}^{N} |\phi(e_j)| &=  \sum_{j = 1}^{N} e^{i \theta_j}\phi(e_j) \\ 
        &= \phi \Big(\sum_{j = 1}^{N} e^{i \theta_j} e_j \Big) \\ 
        &\le \|\phi\| \Big \| \sum_{j = 1}^{N} e^{i \theta_j} e_j \Big \| \\ 
        &= \|\phi\|
        \end{align*}
       Since this is true for all $N \in \mathbb{N}$, taking the limits as $N \to \infty$, the inequality is preserved and we get that $(y_n) \in \ell^1$. Moreover $\phi = \phi_{(y_n)}$ follows from the definition of $ \phi_{(x_n)}$. Hence we get that $  \textbf{c}_0^{*} \cong^{\textrm{iso}} \ell^1$.
    \item 
    \item The proof of this will be extremely similar to what we attempted before when we proved $\textbf{c}_0^{*} \cong^{\textrm{iso}} \ell^1$. Let $(x_n) \in \ell^\infty$. Then consider the map $\phi_{(x_n)}: \textbf{c}_0 \to \mathbb{C}$ defined as $$\phi_{(x_n)}: (y_n) \to \sum_{n \in \mathbb{N}} x_ny_n$$
      By a similar way as we did in the above equivalence we see that $\phi_{(x_n)}$ is linear. Moreover since \[
        \Big | \sum_{n \in \mathbb{N}} x_n y_n \Big | \le \|(x_n)\|_\infty \Big | \sum_{n \in \mathbb{N}} y_n \Big | = \|(x_n)\|_\infty \|(y_n)\|_1
      \]
      we see that $\|\phi_{(x_n)}\| \le \|(x_n)\|_{\infty}$. To get the reverse inequality, Let $\|(x_n)\|_\infty = M$, then for any $\epsilon >0$, there exist some $x_k$ in the sequence $(x_n)$ such that $|x_k - M| < \epsilon$. Now consider the sequence $e_k \in \ell^1$ with $k$th entry $1$ and all the rest of them $0$. We get that \[
        |\phi_{(x_n)}(e_k)| = |x_k| \ge \|(x_n)\|_\infty - \epsilon
      \]
      Since $\epsilon$ was arbitrary, we get that $\|\phi_{(x_n)}\| = \|(x_n)\|_{\infty}$. Hence the map $(x_n) \to \phi_{(x_n)}$ is an isometry. To show that it is indeed a bijection, assume $\phi \in (\ell^1)^{*}$, then consider the sequence $y_n = \phi(e_n)$. Since $\phi$ is continuous, it is bounded above by $ \|\phi\|$ and we get that $ y_n \le \|\phi\|$. Therefore $(y_n) \in \ell^\infty$. Moreover we can verify like above that $\phi = \phi_{(y_n)}$ from the definition of $\phi_{(y_n)}$. Hence we get $(\ell^1)^{*} \ \cong^{\textrm{iso}} \ell^\infty$.
  \end{itemize}
\end{proof}

\begin{theorem}
  Let $1 < p < \infty$, and $ q \in \mathbb{R}$ such that $ \frac{1}{p} + \frac{1}{q} = 1 $. Then $(\ell^p)^{*} \cong \ell^q$
\end{theorem}
\begin{proof}
  Let $(a_n) \in \ell^p, (b_n) \in \ell^q$, then $\sum_{n \in \mathbb{N}} a_n \bar{b_n}$ is the map to check for isometric isomorphism.
  Use Holder's inequality as needed. \textcolor{red}{verify}
\end{proof}

\begin{theorem}
  There exists $\phi \in (\ell^\infty)^{*}$ satisfying the following \begin{enumerate}[label=\arabic*]
    \item $\forall (a_n) \in \ell^\infty$ with $a_n \ge 0$ for all $ n \in \mathbb{N}$, $\phi((a_n)) \ge 0$
    \item If $(a_n)$ is convergent, then $\phi((a_n)) = \lim_{n \to \infty}  a_n$
    \item If $(a_n) \in \ell^\infty$ and $b_n = a_{n+1}$, then $\phi((b_n)) = \phi((a_n))$
  \end{enumerate}
  Moreover such $\phi$ is called a Banach limit.
\end{theorem}
\begin{proof}
  We'll prove this later.
\end{proof}


\begin{corollary}
  $\ell^1$ is not reflexive
\end{corollary}
\begin{proof}
  Let $\phi \in (\ell^\infty)^{*}$ be a Banach limit. FTOC, assume $\exists f = (\alpha_n) \in \ell^1$ such that  \[
    \phi((a_n)) = \sum_{i = 1}^{\infty}  a_n  \overline{\alpha_n}
  \]
  Then for all $m \in \mathbb{N}$, $\overline{\alpha_m} = \phi(\delta_m) = 0$, where $\delta_m = (0, 0, \ldots, 1, 0, 0, \ldots)$. But this contradicts since we assumed $\phi \neq 0$ by the Hahn Banach rextension from $c_0$
\end{proof}

\begin{lemma}
  Let $\psi \in (\ell^\infty)^{*}$. then the following are equivalent. \begin{enumerate}[label=\arabic*]
    \item $\|\psi\| = \psi((1, 1, 1, \ldots))$
    \item If $(a_n) \in \ell^\infty$ with $a_n \ge 0, \forall n \in \mathbb{N}$. Then $ \psi((a_n)) \ge 0$
  \end{enumerate}
\end{lemma}
\begin{proof}
  ($1 \implies 2$) FTSOC assume $\exists (a_n) \in \ell^\infty$, $\psi((a_n)) < 0$. WLOG, assume $|a_n| \le 1$ for all $n \in \mathbb{N}$. let $b_n = 1- a_n$. Then $0 \le b_n \le 1$ and \[
    \psi((b_n)) > \psi((1, 1, 1, \ldots)) -    \psi((a_n)) > \psi((1, 1, 1, \ldots))
  \]
   So \[
     \|\psi\| \ge |\psi((b_n))| \ge \psi((1, 1, \ldots))
   \]

   ($2 \implies 1$) Let $(a_n) \in \ell^\infty$ with $|a_n| \le 1$, then $0 \le 1-a_n$. So $\psi((1-a_n)) \ge 0$ and therefore $\psi((1, 1, 1, \ldots)) \ge \psi((a_n))$. Similarly $\psi((-a_n)) \le \psi((1, 1, 1, \ldots))$ which gives $|\psi((a_n))| \le \psi((1, 1, 1, \ldots))$
\end{proof}




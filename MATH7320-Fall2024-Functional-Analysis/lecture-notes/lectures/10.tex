% TeX_root = ../main.tex

\chapter{}

\begin{theorem}[Krein-Milman Theorem]
  Let $X$ be a locally convex space, and let $K$ be a compact convex subset of $X$. Then the $ \textrm{Ext}(K) \neq \emptyset$ and indeed $K = \overline{\textrm{co}}(\textrm{Ext}(K))$
\end{theorem}
\begin{proof}
  We first prove that the $\textrm{Ext}(K) \neq \emptyset$. Observe that a point $x_0 \in \textrm{Ext}(K)$ if and only if $ K \setminus \{ x_0 \}$ is convex. Also note that $K \setminus \{ x_0 \}$ is a relatively open subset of $K$ since $\{ x_0 \}$ is closed and $K \setminus \{ x_0 \} = \{ x_0 \}^c$ relative to $K$.

  Now let $ \mathcal{A}$ be the collection of all relatively open convex proper subsets of $K$. Note that $\emptyset \in \mathcal{A}$, therefore $\mathcal{A}$ is nonempty. Equip $\mathcal{A}$ with the partial order defined by the set inclusion. Let $\mathscr{C}$ be a chain in $\mathcal{A}$ and $F_{\mathscr{C}} = \cup_{C \in \mathscr{C}} C$. $F_{ \mathscr{C}}$ is relatively open being the union of relatively open subsets of $K$. To see that $F_{\mathscr{C}}$ is convex, let $x,  y \in F_{\mathscr{C}}$. Then since $\mathscr{C}$ is a chain, there exist a $C \in \mathscr{C}$ such that $x, y \in \mathscr{C}$. Then clearly $tx + (1-t)y \in C \subset F_{\mathscr{C}}$.

  We claim that $F_{ \mathscr{C}}$ is a proper subset of $K$. For the sake of contradiction, assume $F_{ \mathscr{C}} = K$. Since $K$ is compact and $C$ is open in $K$ for all $C \in \mathscr{C}$, there are finitely many $C_1 \subset C_2 \subset \ldots \subset C_k \in  \mathscr{C}$ which cover $K$ (i.e $K = \cup_{n = 1}^{k}C_n$). Hence we get $C_k = K$, which is absurd since $C_k$ must be a proper subset of $K$.
  Hence $F_{\mathscr{C}} \in \mathcal{A}$ and thus every chain must have an upper bound in $\mathcal{A}$. Now by Zorn's lemma, $\mathcal{A}$ has a maximal element $K_0$.

  Since $K$ is a connected space (being a TVS), we know that the only clopen subsets are $\emptyset$ and $K$. Since we know that $K_0$ is open being in $\mathcal{A}$, we see that $K_0 \neq K$ and $K_0 \neq \emptyset$. Therefore $ \overline{K_0} \neq K_0$. Let $ x_0 \in \overline{K_0} \setminus K_0$, $y_o \in K_0$ and $0 < t <1$. Define $\varphi_{t, y_0}: K \to K$ such that $\varphi_{t, y_0}(z) = ty_0 + (1-t) z$. Then $\varphi_{t, y_0}$ is (Lipschitz) continuous relative to $K$ and thus $\varphi_{t, y_0}^{-1}(K_0)$ is open in $K$. Also \textcolor{red}{$\varphi_{t, y_0}^{-1}(K_0)$ is convex}.

  We claim, $x_0 \in \varphi_{t, y_0}^{-1}(K_0)$.  Let $U$ be a convex neighborhood of $0 \in X$ containing $-x$ for all $x \in U$ (just take $-U$ and intersect with $U$) such that $(y_o + U) \cap K \subset K_0$. Let $w = \varphi_{t, y_0}(x_0)$. Since $x_0 \in \overline{K_0}$, there exists $x_1 \in K_0$ such that $(x_0 + (1-t)U) \cap K_0 \neq \emptyset$. Then $( \frac{t}{1-t}U  )\cap (K_0 - x_0) \neq \emptyset$. Choose $z$ in the above set. Then \[
    y_0 - ( \frac{1-t}{t}z ) \in y_0 + E \subset K_0
  \]
  and $x_0 + z \in K_0$. Since $K_0$ is convex, \[
    t(y_0 - \frac{(1-t)}{t}z) + (1-t)(x_0 + z)  = \phi_{t, y_0}(x_0) \in K_0
  \]
  Then by the maximality of $K_0$, we get that $\varphi^{-1}_{t, y_0}(K_0) = K$.

  Now we claim that $K = K_0 \cup \{ x_0 \}$. For the sake of contradiction assume $\exists p \in K$ such that $p \notin K_0 \cup \{ x_0 \}$. Since the space is Hausdorff and compact, $x_0$ has an open convex neighborhood $E$ in $X$ such that $p \not\in E$. Let $E^\prime = E \cap K$, $a \in K_0, b \in E^\prime$ and $0 < r < 1$. Then since $\varphi_{t, y_0}(K) = K_0$ for all $t, y_0 \in K_0$, we get $\varphi_{r, b}(b) = ra + (1-r)b \in K_0$. So $K_0 \cup E^\prime$ is convex (Sine we know that $K_0, E^\prime$ are compact, we only need to worry about $rx + (1-r)y$ for $x \in K_0, y \in E^\prime$, but $\varphi_{r, b}$ takes care of that). $K_0 \cup E^\prime$ is also open in $K$. Hence by maximality, we get $K_0 \cup E' = K$. But this is a contradiction since $ p \not\in K_0 \cup E^\prime$. Thus by the equivalence at the beginning, we see that $x_0 \in \textrm{Ext}(K)$.

  Next we prove $K = \overline{co}(\textrm{Ext}(K))$. Let $P = \overline{co}(\textrm{Ext}(K))$ and for the sake of contradiction assume $P \neq K$. Let $ x_0 \in K \setminus P$. Let $E$ be an open convex neighborhood of $0 \in X$ such that $(x_0 + E^\prime) \cap P = \emptyset$ for $E^\prime = E \cap K$. Define $\phi: X \to \mathbb{R}$ such that \[
    \phi(x) = \inf \{ 0 \le t \big | x \in tE \}
  \]
  Observe that $E = \{ x \in X \: \phi(x) < 1 \}$. For every $r \ge 0$ and $x \in X$, $\phi(rx) = r \phi(x)$, and for all $x, y \in X$, $ \phi(x+y) \le \phi(x) + \phi(y)$. Define $f:  \mathbb{R}\{ x_0 \} \to \mathbb{R}$, $f(rx_0) = r \phi(x_0)$, for all $r \in \mathbb{R}$. For every $r \ge 0$, we have $f(rx_0) = r \phi(x_0) = \phi(rx_0)$. For $r < 0$, we have $f(rx_0) = r \phi(x_0) = -\phi(-rx_0) \le 0 \le \phi(rx_0)$. So $f \le \phi$ on $\mathbb{R} x_0$. Then by HBT theorem, there is an extension $ \tilde{f}: X \to \mathbb{R}$ such that $\tilde{f}(x) \le \phi(x) $ for all $x \in X$.

  \textcolor{red}{For Fun we'll use something else now.}
  Let $P$ be as before.
\end{proof}



% TeX_root = ../main.tex

\chapter{Topological Vector Spaces}

\begin{example}
  Let $X$ be a vector space and let $f: X \to \mathbb{C}$ be a linear
  map. Define $\phi: X \to R^{+} := \phi(x) = |f(x)|$. Then $\phi$ is
  a seminorm.
\end{example}

\section{Weak and Weak * Topologies}
\begin{remark}
  Let $X$ be a TVS and $A \subset X^{*}$. We denote by $ \sigma(X,
  A)$, the topology on $X$ defined by $A$. ( initial topology).
  Recall that $\sigma(X, A)$ is Hausdorff if and only if $A$ separate
  points of $X$.

  $\sigma(X, X^{*})$ is called the weak topology on $X$.

  Also recall that $X \hookrightarrow X^{**}$ by the evaluation maps.
  Hence we can view $X$ as a subset $X^{**}$. And with this
  identification, we call $\sigma(X^{*}, X)$ the weak $*$ topology on $X^{*}$

\end{remark}

\begin{definition}
  Let $S$ be any set. Let $I$ be a directed set. A net in $S$ indexed
  by $\Lambda$ is a function $f:\Lambda \to S$. We denote the net by
  $(x_\lambda)_{\lambda \in \Lambda}$.

  In addition if $S$ is a topological space, we say a net
  $(x_\lambda)_{\lambda \in \Lambda}$ converges to a point $x \in S$
  if  for all open set $U$ in $S$ with $x \in U$, there exists an
  $\lambda_0 \in \Lambda$ such that for all $\lambda \ge \lambda_0$
  we have $x_\lambda \in U$.
\end{definition}

\marginnote{ \scriptsize See how nets become the topological
  equivalent of sequences in metric spaces. Find what
  property of the metric space makes it enough to be indexed by a
countable totally ordered set for openness.}

\begin{remark}
  By definition, a basis of open neighborhoods of a point $x_0 \in X$
  in $\sigma(X, A)$ is given \[
    \bigcup_{\substack{p_1 , p_2 , \ldots , p_n \in A \\ \epsilon_1 ,
    \epsilon_2 , \ldots , \epsilon_n > 0}}\bigcap_{i = 1}^{n} \{ z
    \in Z | p_k(z-x_0) < \epsilon_k \}
  \]

  So the basis in a weak topology is \[
    \bigcup_{f_1 , f_2 , \ldots , f_n \in X^{*} \ \epsilon_1 ,
    \epsilon_2 , \ldots , \epsilon_n > 0}(x_0) = \{  z \in Z \Big|
    |f_k(z- x_0)|< \epsilon_k, \textrm{ for all } k = 1, 2, \ldots n \}
  \]
\end{remark}

\begin{example}
  Let $S$ be a topological space, $E \subset S$, $x_0 \in
  \overline{E}$. Then there is a net $(x_\alpha) \subset E$ such that
  $x_\alpha \to x_0$.
\end{example}
\begin{proof}
  Consider the collection $\mathscr{T}$ of all open sets of $S$ that
  contain $x_0$. Order $\mathscr{T}$ by the reserve inclusion. That
  is $A \le B$ if $B \subset A$. This makes $\mathscr{T}$, a directed
  set. Now each $ \lambda \in \mathscr{T}$ has a nonempty
  intersection with $E$ being an open set containing the limit point
  $x_0$ of $E$. For each $\lambda \in \mathscr{T}$ choose an
  $x_\lambda \in \lambda\cap E$. Then we claim that the net
  $(x_\lambda) \to x_0$.
\end{proof}

\begin{example}
  Let $X = \ell^{1}$. Then $X^{*} = \ell^{\infty}$. Then the weak *
  topology on $\ell^{\infty}$ is given by the pointwise convergence
  of a net $(f_\alpha) \subset \ell^{\infty}$ converges to $f \in
  \ell^{\infty}$ if and only if $f_\alpha(n) \to f(n)$ for all $ n
  \in \mathbb{N}$
\end{example}


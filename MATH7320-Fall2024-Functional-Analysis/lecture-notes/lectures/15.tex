% TeX_root = ../main.tex

\begin{theorem}
  Let $\mathcal{H}$ be a Hilbert space and let $C$ be a non-empty closed convex
  subset of $\mathcal{H}$. Then there exist a unique vector $x \in C$ such that
  $\|x\| \le \|\eta\|$ for all $\eta \in C$.
\end{theorem}
\begin{proof}
  \marginnote{ \scriptsize \textcolor{red}{construction of the
  proof is a bit tricky}}
  Let $d = \inf \{ \|\eta\|  \ : \ \eta \in C  \}$ and choose a
  sequence $ \eta_n \in C$  such that $\|\eta_n\| \to d$. Let $
  \epsilon > 0$. Choose $N \in \mathbb{N}$ such that $\|\eta_n\|^2 <
  d^2 + \epsilon$ for all $n \ge N$. Then for all $m,n \ge N$, we have \[
    \|\eta_n - \eta_m\|^2 = 2(\|\eta_n\|^2 + \|\eta_m\|^2) - 4 \|
    \frac{1}{2}(\eta_n + \eta_m) \|^2 \le 4(d^2  + \epsilon) - 4d^2 = 4 \epsilon
  \]
  (Note that since $C$ is compact, $\frac{1}{2}\eta_n +
    \frac{1}{2}\eta_m \in C$ and therefore, by the definition of $d$,
  we get $\|\frac{1}{2}(\eta_n + \eta_m)\| \ge d$.) Hence the sequence
  $\eta_n$ is Cauchy and hence convergent since
  the space is complete. Let $\eta = \lim_{n \to \infty} \eta_n$.
  Since $C$ is closed $\eta \in C$ and clearly $\|\eta\| = d$.

  To see the uniqueness, assume $\alpha \in C$, and $\| \alpha\| = d$. Then
  \begin{align*}
    \|\eta - \alpha\|^2 &= 2(\|\eta\|^2 + \|\alpha\|^2) - 4 \|
    \frac{1}{2}(\eta + \alpha) \|^2 \\
    & \le 4d^2 - 4d^2 = 0
  \end{align*}
  Verify the second inequality.
\end{proof}

\begin{corollary}
  Let $\eta \in \mathcal{H}$ and $C$ be as before. Then there exist a unique
  vector $x \in C$ such that $d(\eta, C) = \|x - \eta\|$
\end{corollary}
\begin{proof}
  Apply above theorem to $C^\prime = C - \eta$. Since $C$ is closed
  and convex, so will be its translation $C - \eta$.
\end{proof}

\begin{proposition}[Pythagoras Theorem]
  Let $ x, y \in \mathcal{H}$ an inner product space, and $x \perp y$, then
  $\|x + y\|^2 = \|x\|^2 + \|y\|^2$.
\end{proposition}

\begin{lemma}
  \label{lem:15}
  Let $\mathcal{H}$ be a Hilbert space and $K$ be a nontrivial closed subspace.
  Let $\eta \in \mathcal{H}$. Then $\xi \in K$ satisfy $\|\xi -
  \eta\| = d(\eta, K)$ iff
  $\xi - \eta \perp K$.
\end{lemma}
\begin{proof}
  Let $\xi - \eta \perp K$ and $k \in K$. Then $\eta - k = (\eta -
  \xi) + (\xi - k)$ and by Pythagoras theorem, we get
  \begin{align*}
    \|\eta - k\|^2 = \|\eta - \xi\|^2 + \|\xi - k\|^2 \ge \|\eta - \xi\|^2
  \end{align*}
  Thus we see that $\|\eta - \xi\| = d(\eta, K)$.

  \marginnote{ \scriptsize  \textcolor{red}{cool proof technique}}
  Conversely, let  $\|\eta - \xi\|= d(\eta, K)$. Then for all $\rho
  \in K$ and $t > 0$, we have
  \begin{align*}
    \|\eta - \xi\|^2 &\le \|\eta - (\xi + t\rho)\|^2  \\
    & = \| \eta - \xi -  t\rho\|^2 \\
    & = \|\eta - \xi\|^2 + t^2\|\rho\|^2 - 2t\Re \langle \eta - \xi ,
    \rho \rangle
  \end{align*}
  Hence we see that $|2\Re \langle \eta - \xi , \rho \rangle| \le t\|\rho\|^2$
  Since $t>0$ was arbitrary, limiting it to zero, we get $\Re \langle
  \eta - \xi ,  \rho \rangle = 0$. Replacing $\rho$ with $-i \rho$
  will give the imaginary part is also zero.
\end{proof}

\begin{remark}
  \label{KplusKperpisX}
  If $K$ is a closed subspace of the Hilbert space $\mathcal{H}$,
  then $\mathcal{H} = K \oplus K^\perp$. To see this notice that $K
  \cap K^\perp = \{ 0 \}$. Now if $x \in \mathcal{H}$, then there is
  a $k_x \in K$ such that $d(x, K) = \|x- k_x\|$. Moreover, from
  \autoref{lem:15}, we see that $x - k_x \in K^\perp$. Hence $x =
  k_x + (x - k_x) \in K \oplus K^\perp$.
\end{remark}

\begin{theorem}[Reisz Representation Theorem]
  Let $\mathcal{H}$ be a Hilbert space and $f \in \mathcal{H}^*$.
  Then there exists a
  unique $\eta_f \in \mathcal{H}$ such that $f(x) = \langle x , \eta_f \rangle $
  for all $ x \in \mathcal{H}$. The map $\phi:  \mathcal{H}^* \to
  \mathcal{H} := f \to \eta_f$ is
  conjugate linear isometric bijection.
\end{theorem}
\begin{proof}
  If $f$ is the zero linear functional, $n_f = 0$ and we're done. If
  not let $K  = \textrm{Ker}(f)$. Then $K$ has co-dimension $1$.
  Consider $f|_{K^\perp}$. Let $v \in K^\perp$ such that $f(v) = 1$.
  Clearly $  K^\perp = \textrm{span}(v)$. And therefore for any
  $\alpha  v \in K^\perp$, $f(\alpha  v) = \alpha = \langle \alpha v
  , \frac{v}{\|v\|^2} \rangle $. Choose $\eta_f = \frac{v}{\|v\|^2}$.
  Now verify.
\end{proof}

\section{Orthogonal Projections}

\begin{theorem}
  Let $\mathcal{H}_1, \mathcal{H}_2$ be Hilbert spaces, and $T:
  \mathcal{H}_1 \to  \mathcal{H}_2$ be a bounded
  linear map. Then there exists a unique bounded linear map $T^*: \mathcal{H}_2
  \to \mathcal{H}_1$ satisfying $\langle  Tx , y
  \rangle_{\mathcal{H}_2} = \langle x ,
  T^*y \rangle_{\mathcal{H}_1}$ for all $x \in \mathcal{H}_1, y \in
  \mathcal{H}_2$.
\end{theorem}
\begin{proof}
  For every given $y \in \mathcal{H}_2$ define a linear functional $f^y: H_1
  \to \mathbb{C}$ as $f^y(x) = \langle Tx , y \rangle$. Since $f^{y}$
  is bounded, $f^y \in \mathcal{H}_1^*$. Hence by Reisz representation, there
  is a unique $T^*(y) \in \mathcal{H}_1$ such that $\langle  Tx , y \rangle =
  \langle x , T^*y \rangle$.

  Uniqueness follows from the fact that in any inner product space
  $X$, if $x, y \in X$ such that $ \langle x , z \rangle = \langle  y
  , z \rangle $ for all $z \in X$, then $\langle  x-y , z \rangle =
  0$ for all $z \in X$, in particular $z = x-y$. Then by the positive
  definiteness of the inner product, we get $x = y$.
\end{proof}

\begin{theorem}
  Let $\mathcal{H}$ be a Hilbert space and $K$ a closed subspace. For every
  $\eta \in \mathcal{H}$, denote by $P_K(\eta)$, the unique closest vector in
  $K$, closest to $\eta$. Then
  \begin{enumerate}[label=(\arabic*)]
    \item $P_K: \mathcal{H} \to \mathcal{H}$ is linear, bounded with $\|P_K\| =
      1$ and idempotent.
    \item $P_K^* = P_K$ (self-adjoint)
  \end{enumerate}
\end{theorem}
\begin{proof}
  \begin{enumerate}[label=(\arabic*)]
    \item Let $\eta_1, \eta_2 \in \mathcal{H}$ and $\alpha \in \mathbb{C}$.
      Then using \autoref{lem:15} for all $\xi \in K$, we have
      \begin{align*}
        \langle  \alpha \eta_1
        +\eta_2 - \alpha P_K( \eta_1) - P_K(\eta_2) , \xi \rangle &=
        \alpha \langle \eta_1 - P_K(\eta_1) , \xi  \rangle + \langle
        \eta_2 - P_K(\eta_2) , \xi  \rangle  \\
        &= 0
      \end{align*}
      Thus we get $P_K(\alpha \eta_1 + \eta_2) = \alpha P_K(\eta_1) +
      P_K(\eta_2)$ again by the same \autoref{lem:15}. Moreover since
      $P_K(x)$ is a vector in $K$ it the vector in $K$ closest to
      $P_K(x)$ is $P_K(x)$ itself. Thus we see that $P^2_K(x) =
      P_K(x)$. Hence idempotent. Moreover since $K$ is closed, we
      have $X = K \oplus K^\perp$ (\autoref{KplusKperpisX}). Thus for
      all $x \in X$, we can
      write $ x = k_x + k^\prime_x$, where $ k_x \in K$ and $
      k^\prime_x \in K^\perp$. Then $P_K(x) = k_x$ and triangle
      inequality shows $ \|P_K(x)\| = \|k_x\| \le \|x\|$. Thus $\|P_K\| \le 1$.

    \item Since $X = K \oplus K^\perp$, let $x = k_x + k^\prime_x$
      and $y = k_y + k^\prime_y$. Then
      \begin{align*}
        \langle P(x) ,  y \rangle = \langle k_x , k_y + k^\prime_y
        \rangle = \langle k_x , k_y \rangle = \langle k_x  +
        k^\prime_x , k_y \rangle = \langle x , P_K(y) \rangle
      \end{align*}
      and the uniqueness of the adjoint proves that $P_K^* = P_K$
  \end{enumerate}
\end{proof}


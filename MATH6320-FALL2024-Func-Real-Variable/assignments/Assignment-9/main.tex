% initial settings
\documentclass[12pt]{exam}
\usepackage{geometry}
\usepackage{graphicx}
\usepackage{enumitem}
\usepackage[usenames,dvipsnames]{xcolor}
\usepackage[backend=biber, style=alphabetic]{biblatex}
\usepackage{url,hyperref}

\usepackage{amsmath} % math symbols, matrices, cases, trig functions,
% var-greek symbols.
\usepackage{amsfonts} % mathbb, mathfrak, large sum and product symbols.
\usepackage{amssymb} % extended list of math symbols from AMS.
% https://ctan.math.washington.edu/tex-archive/fonts/amsfonts/doc/amssymb.pdf
\usepackage{amsthm} % theorem styling.
\usepackage{mathrsfs} % mathscr fonts.
\usepackage{yhmath} % widehat.
\usepackage{empheq} % emphasize equations, extending 'amsmath' and 'mathtools'.
\usepackage{bm} % simplified bold math. Do \bm{math-equations-here}

% geometry of paper
\geometry{
  a4paper, % 'a4paper', 'c5paper', 'letterpaper', 'legalpaper'
  asymmetric, % don't swap margins in left and right pages. as
  % opposed to 'twoside'
  centering, % to center the content between margins
  bindingoffset=0cm,
}

% hyprlink settings
\hypersetup{
  colorlinks = true,
  linkcolor = {red!60!black},
  anchorcolor = red,
  citecolor = {green!50!black},
  urlcolor = magenta,
}

% theorem styles
\theoremstyle{plain} % default; italic text, extra space above and below
\newtheorem{theorem}{Theorem}[section]
\newtheorem{proposition}{Proposition}[section]
\newtheorem{lemma}{Lemma}[section]
\newtheorem{corollary}{Corollary}[theorem]

\theoremstyle{definition} % upright text, extra space above and below
\newtheorem{definition}{Definition}[section]
\newtheorem{example}{Example}[section]

\theoremstyle{remark} % upright text, no extra space above or below
\newtheorem{remark}{Remark}[section]
\newtheorem*{note}{Note} %'Notes' in italics and without counter

% renewcommands for counters
\newcommand{\propositionautorefname}{Proposition}
\newcommand{\definitionautorefname}{Definition}
\newcommand{\lemmaautorefname}{Lemma}
\newcommand{\remarkautorefname}{Remark}
\newcommand{\exampleautorefname}{Example}

\addbibresource{articles.bib}

\begin{document}

\title{MATH6320 - Theory of Functions of a Real Variable \\ Assignment 9 }

% author list
\author{
  Joel Sleeba \\
}

\maketitle
\printanswers
\unframedsolutions

\begin{questions}
  \question
  \begin{solution}
    \begin{parts}
      \part Let $r < p < s$, where $ r, s \in E$. Then by the
      convexity of $[r, s] \subset \mathbb{R}$, there is a $ t \in
      [0, 1]$ such that $p = tr + (1-t)s$. Then Holder's
      inequality on $\frac{1}{t}$ and $\frac{1}{(1-t)}$ gives,
      \begin{align*}
        \int |f|^p \ d  \mu &= \int |f|^{tr}|f|^{(1-t)s} \ d \mu \\
        &\le \Bigg( \int |f|^{\frac{tr}{t}} \ d m \Bigg)^{t} \Bigg( \int
        |f|^{\frac{(1-t)s}{(1-t)}} \ d m\Bigg)^{1-t} \\
        &= \Bigg( \int |f|^{r} \ d m \Bigg)^{t} \Bigg( \int
        |f|^{s} \ d m\Bigg)^{1-t} \\
        &= \| f\|_r^{rt} \| f\|_s^{s(1-t)}
      \end{align*}
      Thus we get $\|f\|_p \le \|f\|_r^{\frac{rt}{p}}
      \|f\|_s^{\frac{s(1-t)}{p}}$

      For the sake of contradiction, assume that $\|f\|_p >
      \max\{\|f\|_r, \|f\|_s\}$. Then by the monotonicity of the
      function $x \to x^k$, where $k >0$, we get
      \begin{align*}
        \|f\|_p^{\frac{rt}{p}} > \|f\|_r^{\frac{rt}{p}} \quad
        \textrm{ and } \quad
        \|f\|_p^{\frac{s(1-t)}{p}} > \|f\|_s^{\frac{s(1-t)}{p}}
      \end{align*}
      Then we'll get
      \begin{align*}
        \|f\|_p = \|f\|_p^{\frac{rt}{p}} \|f\|_p^{\frac{s(1-t)}{p}} >
        \|f\|_r^{\frac{rt}{p}} \|f\|_s^{\frac{s(1-t)}{p}}
      \end{align*}
      contradicting our previous result. Hence we see that $\|f\|_p
      \le \max\{\|f\|_r, \|f\|_s\}$

      \part Let $0 < \epsilon$. Consider the set $A_\epsilon = \{ x
      \in X  \ : \  \|f\|_\infty < |f(x)| + \epsilon \}$. Then
      \begin{align*}
        \int_X |f|^p \ d \mu &\ge \int_{A_\epsilon} |f|^p \ d \mu \\
        &\ge \int_{A_\epsilon} (\|f\|_\infty - \epsilon)^p \ d \mu \\
        &= ( \|f\|_\infty - \epsilon)^p \mu(A_\epsilon)
      \end{align*}
      Since we are given that $\|f\|_\infty \in (0, \infty]$, there
      is an $\varepsilon > 0$ such that $ \|f\|_\infty > \varepsilon$.
      Moreover since $\|f\|_r < \infty$, the above inequality forces
      $\mu(A_\varepsilon) < \infty$. Then taking power $\frac{1}{p}$
      to the above inequality, we get
      \begin{align*}
        \|f\|_p \ge (\|f\|_\infty - \epsilon) \mu(A_\varepsilon)^{\frac{1}{p}}
      \end{align*}
      Now taking limits, we get
      \begin{align*}
        \lim_{p \to \infty} \inf \|f\|_p \ge (\|f\|_\infty - \varepsilon)
      \end{align*}
      since $ \mu(A_\varepsilon)^{\frac{1}{p}} \to 1$ as $ p \to \infty$.
      Again since $ \varepsilon >0 $ was arbitrary, we get
      \begin{align*}
        \lim_{p \to \infty} \inf \|f\|_p \ge \|f\|_\infty
      \end{align*}

      Now to get the other inequality, observe that
      \begin{align*}
        \int |f|^p \ d \mu &= \int |f|^r \ d \mu \int |f|^{p-r} \ d \mu \\
        &\le \| f\|_\infty^{p-r} \int |f|^r \ d \mu \\
      \end{align*}
      Hence we get
      \begin{align*}
        \|f\|_p = \Big( \int |f|^p \ d \mu\Big)^{1/p} \le \|
        f\|_\infty^{\frac{p-r}{p}} \Big( \int |f|^r \ d
        \mu\Big)^{\frac{1}{p}} = \|f\|_\infty \|f\|_r^{\frac{r}{p}}
      \end{align*}
      Thus taking limits, we see that
      \begin{align*}
        \lim_{p \to \infty} \sup \|f\|_p \le \|f\|_\infty
      \end{align*}
      as $\|f\|_r^{\frac{r}{p}} \to 0$ as $  p \to \infty$ since $
      \|f\|_r < \infty$

      Combining both the inequalities, we see
      \begin{align*}
        \lim_{p \to \infty} \sup \|f\|_p \le \|f\|_\infty \le \lim_{p
        \to \infty} \inf \|f\|_p
      \end{align*}
      Thus
      \begin{align*}
        \lim_{p \to \infty} \|f\|_p = \|f\|_\infty
      \end{align*}
    \end{parts}

  \end{solution}

\end{questions}
\printbibliography[heading=bibintoc]
\end{document}



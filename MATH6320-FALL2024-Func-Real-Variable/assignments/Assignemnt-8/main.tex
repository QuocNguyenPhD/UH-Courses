% initial settings
\documentclass[12pt]{exam}
\usepackage{geometry}
\usepackage{graphicx}
\usepackage{enumitem}
\usepackage[usenames,dvipsnames]{xcolor}
\usepackage[backend=biber, style=alphabetic]{biblatex}
\usepackage{url,hyperref}

\usepackage{amsmath} % math symbols, matrices, cases, trig functions,
% var-greek symbols.
\usepackage{amsfonts} % mathbb, mathfrak, large sum and product symbols.
\usepackage{amssymb} % extended list of math symbols from AMS.
% https://ctan.math.washington.edu/tex-archive/fonts/amsfonts/doc/amssymb.pdf
\usepackage{amsthm} % theorem styling.
\usepackage{mathrsfs} % mathscr fonts.
\usepackage{yhmath} % widehat.
\usepackage{empheq} % emphasize equations, extending 'amsmath' and 'mathtools'.
\usepackage{bm} % simplified bold math. Do \bm{math-equations-here}

% geometry of paper
\geometry{
  a4paper, % 'a4paper', 'c5paper', 'letterpaper', 'legalpaper'
  asymmetric, % don't swap margins in left and right pages. as
  % opposed to 'twoside'
  centering, % to center the content between margins
  bindingoffset=0cm,
}

% hyprlink settings
\hypersetup{
  colorlinks = true,
  linkcolor = {red!60!black},
  anchorcolor = red,
  citecolor = {green!50!black},
  urlcolor = magenta,
}

% theorem styles
\theoremstyle{plain} % default; italic text, extra space above and below
\newtheorem{theorem}{Theorem}[section]
\newtheorem{proposition}{Proposition}[section]
\newtheorem{lemma}{Lemma}[section]
\newtheorem{corollary}{Corollary}[theorem]

\theoremstyle{definition} % upright text, extra space above and below
\newtheorem{definition}{Definition}[section]
\newtheorem{example}{Example}[section]

\theoremstyle{remark} % upright text, no extra space above or below
\newtheorem{remark}{Remark}[section]
\newtheorem*{note}{Note} %'Notes' in italics and without counter

% renewcommands for counters
\newcommand{\propositionautorefname}{Proposition}
\newcommand{\definitionautorefname}{Definition}
\newcommand{\lemmaautorefname}{Lemma}
\newcommand{\remarkautorefname}{Remark}
\newcommand{\exampleautorefname}{Example}

\addbibresource{articles.bib}

\begin{document}

\title{MATH6320 - Theory of Functions of a Real Variable \\ Assignment 8}

% author list
\author{
  Joel Sleeba \\
}

\maketitle
\printanswers
\unframedsolutions

\begin{questions}

  \question
  \begin{solution}
    Let $x_n$ be a sequence in $R_f$ that converge to $x \in
    \mathbb{C}$. We'll be
    done if we prove that $x \in R_f$. Let $\varepsilon > 0$ be given.
    Then there is a $N_{\frac{\varepsilon}{2}} \in \mathbb{N}$ such that
    $|x_n - x| < \frac{\varepsilon}{2}$ for all $n >
    N_{\frac{\varepsilon}{2}}$. Hence
    $B_\varepsilon(x) \supseteq B_{\frac{\varepsilon}{2}}( x_n)$ for all $n
    > N_{ \frac{\varepsilon}{2}}$. Therefore \[
      f^{-1}(B_\varepsilon(x)) \supseteq f^{-1}(B_{\frac{\varepsilon}{2}}(x_n))
    \]
    But $f^{-1}(B_{\frac{\varepsilon}{2}}(x_n)) = A_{x_n,
    \frac{\varepsilon}{2}}$ and $f^{-1}(B_\varepsilon(x)) = A_{x,
    \varepsilon}$. Since $x_n \in R_f$ by assumption, we see that
    $\mu(A_{x_n , \frac{\varepsilon}{2}}) > 0$. Then by the monotonicity
    of the measure, we see that for all $n > N_{\frac{\varepsilon}{2}}$ \[
      \mu(A_{x, \varepsilon}) = \mu(f^{-1}(B_\varepsilon(x))) \ge
      \mu(f^{-1}(B_{\frac{\varepsilon}{2}}(x_n))) = \mu(A_{x_n,
      \frac{\varepsilon}{2}}) > 0
    \]
    Since $\varepsilon>0$ was chosen arbitrarily, we see that
    $\mu(A_{x, \varepsilon})>0$ for all $\varepsilon>0$. Hence $x \in R_f$,
    by the definition of $R_f$.
  \end{solution}

  \question
  \begin{solution}
    Let $f \in L^1(m)$ be bounded $(|f(x)| < M)$ such that $A = \{ x
    \in \mathbb{R} : f(x) \neq 0 \}$ has finite measure $m(A) <
    \infty$. Note that the Lebesgue measure is a
    regular, Borel measure and the space $\mathbb{R}$ is locally compact and
    Hausdorff. Then by Luzin's theorem, for any given
    $\varepsilon> 0$, there is a $g_\varepsilon \in C_c(\mathbb{R})$ such that
    for $E_\varepsilon = \{ x \in \mathbb{R}  \ : \ f(x) \neq
    g_\varepsilon(x) \}$, we have $\mu(E_\varepsilon) <
    \frac{\varepsilon}{4M}$ and
    $|g_\varepsilon(x)| < M$ for all $x \in \mathbb{R}$. Then
    \begin{align*}
      \int |f - g_\varepsilon| \ d m &= \int_{E_\varepsilon} |f -
      g_\varepsilon| \ d m +
      \int_{E_\varepsilon^c} |f-g_\varepsilon|  \ d m \\
      &= \int_{E_\varepsilon} |f-g_\varepsilon| \ d m + 0 \\
      &\le 2M m(E_\varepsilon) \\
      &< 2M\frac{\varepsilon}{4M} \\
      &= \frac{\varepsilon}{2}
    \end{align*}

    Again, since $g_\varepsilon \in C_c(X)$, it is Riemann integrable
    and there is a partition $P_\varepsilon = \{p_1 < p_2 < \cdots <
    p_n \}$ of the compact support $K = \textrm{supp}(g_\varepsilon)$
    (Without loss of generality, we can assume that this $K$ is an
      interval $[p_1, p_n]$. In case it is not, Extend $K$ to its
    convex closure)
    such that \[
      \int g_\varepsilon(x) \ d  x < m_{P_\varepsilon}(g_\varepsilon) +
      \frac{\varepsilon}{2}
    \]
    where the integral above is the Reimann integral and
    $m_{P_\varepsilon}(\varepsilon)$ is the lower Reimann sum of $g_\varepsilon$
    on the partition $P_\varepsilon$.

    Then consider the step function \[
      h = \sum_{i = 1}^{n-1} \chi_{[p_{i}, p_{i+1})} \inf_{x \in
      [p_{i}, p_{i+1}]}g_\varepsilon(x)
    \]
    By definition, we see that $g_\varepsilon \ge h$. Hence
    $g_\varepsilon - h = |g_\varepsilon - h|$. Moreover, \[
      \int h(x) \ d x = m_{P_\varepsilon}(g_\varepsilon)
    \]
    Therefore, \[
      \int |g_\varepsilon - h| \ d x = \int (g_\varepsilon - h)\ d x =
      \int g_\varepsilon \ dx - \int h \ d x = \int g_\varepsilon \ d
      x - m_{P_\varepsilon}(f) < \frac{\varepsilon}{2}
    \]
    Since Riemann integral and Lebesgue integral agree on Reimann
    integrable functions, we get \[
      \int |g_\varepsilon - h| \ d m = \int |g_\varepsilon - h| \ d x
      < \frac{\varepsilon}{2}
    \]
    By triangle inequality, we know that $|f - h| \le |f -
    g_\varepsilon| + |g_\varepsilon - h|$. Then by the linearity and
    monotonicity of the integral on positive functions, we see that \[
      \int |f - h| \ d m \le \int |f - g_\varepsilon| \ d m + \int
      |g_\varepsilon - h| \ d m < \frac{\varepsilon}{2} +
      \frac{\varepsilon}{2} =  \varepsilon
    \]

    Now let $f \in L^1(m)$ and $\varepsilon > 0$ be
    given. Consider the set $B_n = \{ x \in
    \mathbb{R}  \ : \  \frac{1}{n} \le |f(x)| \le n \}$. Clearly $f_n
    = f \chi_{B_n}$ converge pointwise to $f$. To see this let $x \in
    \mathbb{R}$. If $f(x) = 0$, then each $f_n(x) = 0$ and we've
    nothing to prove. Otherwise there is an $N \in \mathbb{N}$ such
    that $\frac{1}{N} < |f(x)|$. Then $f_n(x) = f(x)$ for all $n >
    N$, and we're done. Hence we see that $|f - f_n|$ converge pointwise to $0$.

    Also, notice that $|f_n| < |f|$. Therefore by triangle
    inequality, $|f - f_n| \le 2|f|$ which is again in $L^1(m)$.
    Therefore by dominated convergence theorem, \[
      \lim_{n \to \infty} \int |f - f_n| \ d m = 0
    \]
    Thus there is an $N_\varepsilon$ such that \[
      \int |f - f_{N_\varepsilon}| \ d m < \frac{\varepsilon}{2}
    \]

    Moreover for every $n \in \mathbb{N}$, $\frac{1}{n}\chi_{B_n} \le
    f \chi_{B_n}$ and therefore \[
      \frac{1}{n} m(B_n) = \int \frac{1}{n}\chi_{B_n} \ d m \le
      \int f \chi_{B_N} \ d m \le \int f \ d m < \infty
    \]
    Shows that $m(B_{N_\epsilon}) < \infty$. Then $f \chi_{B_n}$ is a bounded
    function ($|f \chi_{B_n}|< n$) with $\{ x \in \mathbb{R}  \ :
    \ f\chi_{B_n}(x) \neq 0  \} = B_n$. Thus by the first part of the
    proof there is a step function $h_n$ such that \[
      \int |f_n - h_n| \ d m < \frac{\varepsilon}{2}
    \]
    Then specifically for $n = N_\varepsilon$, by the triangle
    inequality and the linearity and monotonicity of the integral, we get
    \begin{align*}
      \int |f - h_n| \ d m &\le \int |f - f_{N_\varepsilon}| \ d m +
      \int |f_{N_\varepsilon} - h_{N_\varepsilon}| \ d m \\
      &< \frac{\varepsilon}{2} + \frac{\varepsilon}{2} \\
      &= \varepsilon
    \end{align*}
  \end{solution}

  \question
  \begin{solution}
    We'll first show that this holds for step functions. Let \[
      s = \sum_{i = 1}^{n} a_i \chi_{[a_i, b_i)}
    \]
    where $a_i < b_i \le a_{i+1}$ for each $i$. Then \[
      s_t(x) := s(x-t) = \sum_{i = 1}^{n} a_i \chi_{[a_i, b_i)}(x-t)
      = \sum_{i = 1}^{n} a_i\chi_{[a_i+t, b_i+t)}(x)
    \]
    Then \[
      s_t - s = \sum_{i = 1}^{n} a_i\chi_{[a_i+t, b_i+t)} -
      \sum_{i = 1}^{n} a_i \chi_{[a_i, b_i)} = \sum_{i = 1}^{n} a_n
      (\chi_{[a_i + t, b_i +t)} - \chi_{[a_i, b_i)})
    \]
    Now when $0 < t < \min\{b_i - a_i\}$ (such $t$ must exist, since
    $a_i < b_i$ for each $i$) and $M = \max\{|a_i|\}$, we see that \[
      |s_t - s| = \Bigg|\sum_{i = 1}^{n} a_n (\chi_{[b_i, b_i + t)} -
      \chi_{[a_i, a_i + t)})\Bigg| \le M \sum_{i = 1}^{n}
      (\chi_{[b_i, b_i + t)} +
      \chi_{[a_i, a_i + t)})
    \]
    Then   \[
      \int |s_t - s| \ d m \le M \sum_{i = 1}^{n} 2t = 2Mnt
    \]
    Since $M, n$ does not depend on $t$, taking limits as $t \to 0$,
    we see that \[
      0 \le \lim_{t \to 0} \int |s_t - s| \ d m \le \lim_{t \to 0}\  2Mnt = 0
    \]

    Now for the general case, let $f \in L^1(\mu)$ and $\epsilon > 0$
    be given. Then by the previous answer there is a step function
    $s$ such that  \[
      \int |f - s| \ d m <  \frac{\varepsilon}{3}
    \]
    Moreover, by the first part of this proof, there is a
    $t_\varepsilon > 0$ such that for all $t \in [0, t_\varepsilon]$ \[
      \int |s_t - s| \ d m < \frac{\varepsilon}{3}
    \]
    Also notice that $f_t - s_t = (f-s)_t$.
    Since Lebesgue measure is translation invariant, we get that \[
      \int |f_t - s_t| \ d m  = \int |(f-s)_t| \ d m = \int |f - s|
      \ d m < \frac{\varepsilon}{3}
    \]
    Thus we see that for all $t \in [0, t_\varepsilon]$,
    \begin{align*}
      \int |f - f_t| \ d m \le \int |f-s| \ d m + \int |s - s_t| \ d
      m + \int |s_t - f_t| \ d m < \frac{\varepsilon}{3} +
      \frac{\varepsilon}{3} + \frac{\varepsilon}{3} = \varepsilon
    \end{align*}
    Since $\varepsilon$ was arbitrary, we have proved the statement
    for general $f \in L^1(m)$.
  \end{solution}

  \begin{proposition}
    If $\mu$ is a translation invariant measure on $X$, for any measurable
    function $f: X \to \mathbb{C}$, \[
      \int f \ d \mu = \int f_t \ d \mu
    \]
    where $f_t(x) = f(x-t)$ for all $t, x \in X$
  \end{proposition}
  \begin{proof}
    We'll prove this for non-negative function $f$, then the general
    case will follow from decomposing a complex valued $f$ into
    linear combinations of 4 non-negative valued functions.

    Let $f$ be non-negative measurable function and $0 \le s \le f$ be a
    measurable simple function. Let \[
      s = \sum_{i = 1}^{n} a_i \chi_{A_i}
    \]
    Then, \[
      s_t(x) = \sum_{i = 1}^{n} a_i \chi_{A_i}(x-t) \le f(x-t) = f_t(x)
    \]
    Hence we get $s_t \le f_t$. Conversely, let $0 \le h \le f_t$ be a
    simple measurable function of the form
    \[
      h(x) = \sum_{i = 1}^{m} b_i \chi_{B_i}
    \]
    Then, \[
      h_{-t}(x) = \sum_{i = 1}^{m} b_i \chi_{B_i}(x+t) \le f_t(x+t) = f(x)
    \]
    Hence we get $h_{-t} \le f$. Thus we have shown a correspondence
    between simple functions under $f$ and $f_t$. Moreover the
    translation invariance of $\mu$ gives \[
      \int h \ d \mu = \sum_{i = 1}^{m} b_i \mu(B_i) = \sum_{i =
      1}^{m} b_i \mu(B_i + t) =  \int  h_{-t} \ d \mu
    \]
    Thus taking supremums over all measurable simple functions under
    $f$ and $f_t$, we see that \[
      \int f \ d \mu = \int f_{t} \ d \mu
    \]
  \end{proof}

\end{questions}
\printbibliography[heading=bibintoc]
\end{document}



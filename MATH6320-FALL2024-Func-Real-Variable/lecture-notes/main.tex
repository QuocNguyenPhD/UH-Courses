\documentclass[12pt]{report}

\usepackage{geometry} % automatic papersizes, margins.
\usepackage{makeidx} % 'makeidx' make and show index
\usepackage{enumitem} % itemize, enumerate, description.
\usepackage{hyperref} % hyperlinks, cross-references.
\usepackage{xcolor} % foreground and background color management.
% Better color mixing compared to 'color'
\usepackage{graphicx} % provide options for \includegraphics. Builds
% on 'graphic'
\usepackage{caption} % better control over captions of figures and equations.
\usepackage{appendix} % extra control over appendix
\usepackage[backend=biber, style=alphabetic]{biblatex} % better than
% bibtex, people say.
\usepackage{tocbibind} % add ToC/Bibliography/Index to ToC
\usepackage{marginnote}
\usepackage{amsmath} % math symbols, matrices, cases, trig functions,
% var-greek symbols.
\usepackage{amsfonts} % mathbb, mathfrak, large sum and product symbols.
\usepackage{amssymb} % extended list of math symbols from AMS.
% https://ctan.math.washington.edu/tex-archive/fonts/amsfonts/doc/amssymb.pdf
\usepackage{amsthm} % theorem styling.
\usepackage{mathrsfs} % mathscr fonts.
\usepackage{yhmath} % widehat.
\usepackage{empheq} % emphasize equations, extending 'amsmath' and 'mathtools'.
\usepackage{bm} % simplified bold math. Do \bm{math-equations-here}
\usepackage{tikz} % for tikz diagrams
% \usepackage{tikz-cd} % commutative diagrams.
\usepackage{verbatim} % for concealing solutions

\geometry{
  a4paper, % 'a4paper', 'c5paper', 'letterpaper', 'legalpaper'
  asymmetric, % don't swap margins in left and right pages. as
  % opposed to 'twoside'
  centering, % to center the content between margins
  bindingoffset=0cm,
}

\hypersetup{
  colorlinks,
  linkcolor={blue!50!black},
  citecolor={blue!50!black},
  urlcolor={blue!80!black}
}

\theoremstyle{plain} % default; italic text, extra space above and below
\newtheorem{theorem}{Theorem}[section]
\newtheorem{proposition}{Proposition}[section]
\newtheorem{lemma}{Lemma}[section]
\newtheorem{corollary}{Corollary}[theorem]
\newtheorem{problem}{Problem}[section]

\theoremstyle{definition} % upright text, extra space above and below
\newtheorem{definition}{Definition}[section]
\newtheorem{example}{Example}[section]
\newtheorem{exercise}{Exercise}[section]

\theoremstyle{remark} % upright text, no extra space above or below
\newtheorem{remark}{Remark}[section]
\newtheorem*{note}{Note} %'Notes' in italics and without counter

\newcommand{\propositionautorefname}{Proposition}
\newcommand{\lemmaautorefname}{Lemma}
\newcommand{\corollaryautorefname}{Corollary}
\newcommand{\problemautorefname}{Problem}
\newcommand{\definitionautorefname}{Definition}
\newcommand{\exampleautorefname}{Example}
\newcommand{\remarkautorefname}{Remark}
\newcommand{\noteautorefname}{Note}

% For exercise and solutions
% \newif\ifshowsolutions
% \showsolutionstrue % Change to \showsolutionsfalse to hide solutions

% Custom environments for problems and solutions
% \newcounter{exercise}[chapter]
% \newenvironment{problem}[1][]
%   {\par\noindent\textbf{#1.}}
%   {\par}
%
% \newenvironment{solution}
%   {\ifshowsolutions \expandafter\solutioncontent \else
% \expandafter\comment \fi}
%   {\ifshowsolutions \hfill \qedsymbol \else \expandafter\endcomment
% \fi \vspace{1em}}
%
% \newenvironment{solutioncontent}
%   {\par\noindent\textit{Solution.}}
%   {\par}

\addbibresource{articles.bib}

\begin{document}
\title{MATH6320 - Functions of a Real Variable}

% \showsolutionstrue
%\showsolutionsfalse %If need to hide solutions

\author{
  Joel Sleeba \\
  % Cochin University of Science and Technology\\
  joelsleeba1@gmail.com \\
}

\maketitle

\pagenumbering{roman} \setcounter{page}{2}
\tableofcontents
\pagenumbering{arabic} \setcounter{page}{1}

% TeX_root = ../main.tex

\chapter{}

\section{Internal Direct products}

Suppose $G$ is a group and $H, K \leqslant G$. Define \[
     HK = \{ hk  \ : \  h \in H, k \in K \}
\]

\begin{theorem}
  Suppose $|H|, |K| < \infty$, then $|HK| = \frac{|H||K|}{|H \cap K|}$.
\end{theorem}
\begin{proof}
   We write $HK = \cup_{h \in H}hK$. Then $h_1K  = h_2K$ if and only if $h_2^{-1}h_1 \in K$. This is equivalent to saying $h_2^{-1}h_2 \in H\cap K$. \textcolor{red}{verify}
\end{proof}

\begin{theorem}
   If $H, K \leqslant G$ then $HK \leqslant G$ if and only if $HK = KH$.
\end{theorem}

\begin{corollary}
If $H$ is a normal subgroup of $G$, then $HK \leqslant G$
\end{corollary}


\begin{theorem}
  If $H, K \leqslant G$, $|H|, |K| < \infty$, $H, K$ normal in $G$, and $H \cap K = \{ e \}$, then $HK \cong H \times K$. In this case $HK$ is called the internal direct product of $H$ and $K$.
\end{theorem}
\begin{proof}
  By the corollary, $HK \leqslant G$. Therefore consider the map $\phi: HK \to H \times K := hk \to (h, k)$. \begin{itemize}
    \item $\phi$ is well defined. We know that $|HK| = \frac{|H||K|}{|H \cap K|} = |H||K|$. Hence every element in $HK$ has a unique representation as $hk$. Thus we see that $\phi$ is well defined.
    \item $\phi$ is a homomorphism. Let $h_1k_1, h_2k_2 \in HK$. Want to show that $\phi(h_1k_1h_2k_2) = \phi(h_1k_1)\phi(h_2k_2) = (h_1, k_1)(h_2, k_2) = (h_1h_1, k_1k_2)$. Equilvalently we  want to show that $h_1k_1h_2k_2 = h_1h_2k_1k_2$, which is again equivalent to showing $hk = kh$ for all $h \in H, k \in K$. This is again equivalent to showing $[h, k] = h^{-1}k^{-1}hk = e$ for all $ h\in H, k \in K$.

      To see why $[h, k] = \{ e \}$. see that $[h, k] = h^{-1}(k^{-1}hk) = h^{-1}h^\prime \in H$ for some $ h^\prime \in H$. Similarly $[h, k] = (h^{-1}k^{-1}h) k = k^\prime k \in K$ for some $k^\prime \in K$. Both of these are by the normality of $H$ and $K$. But since we know that $H \cap K = \{ e \}$, we see that $[h, k] = e$ for all $h, k$. Hence we are done.
  \end{itemize}
\end{proof}

% TeX_root = ../main.tex

\chapter{}

Assignment 1 is posted. Submissions due Aug 29.

\section{Warm up}
\begin{example}
  Let $X = \{ 1, 2, 3 \}, F = \{ \{ 1, 2 \}, \{ 1, 3 \} \}$. Then the smallest topology containing $F$ is $\{ \emptyset, X, \{ 1 \}, \{ 1, 2 \}, \{ 1, 3 \} \}$, and the $\sigma$-algebra generated by $F$ is the power set, $P(X)$.
\end{example}
\section{continues}
\begin{proof}
  Proof of \autoref{thm:sigma_algeba_generated}.

  Consider all $\sigma$-algebras containing $F$, let $\Omega = \{ \mathscr{N} \subset P(X) \ : \ \mathscr{N} \supset F, \mathscr{N} \textrm{ is a } \sigma-\textrm{algebra} \}$. $\Omega$  is non-empty since $P(X) \subset \Omega$. Let \[
      \mathscr{M} = \cap_{\mathscr{N} \in  \Omega} \mathscr{N}
  \]
   Then we claim $\mathscr{M}$ is a $\sigma$-algebra. To see this 
   \begin{itemize}
     \item $X \in \mathscr{M}$, because $X \in \mathscr{N}$, for each $\mathscr{N} \in  \Omega$.
     \item If $E \in \mathscr{M}$, then $E \in \mathscr{N}$ for each $\mathscr{N} \in \Omega$. Then $E^c \in \mathscr{N}$ for each $\mathscr{N} \in \Omega$ and thus $E^c \in \mathscr{M}$.
     \item If $A_1, A_2, \ldots \in \mathscr{M}$, then $\cup_{j = 1}^{\infty}A_j \in \mathscr{M}$ because since each $A_i \in \mathscr{N}$ and $\mathscr{N}$ is a $\sigma$-algebra, $\cup_{j = 1}^{\infty}A_j \in \mathscr{N}$ for each $\mathscr{N} \in \Omega$.
   \end{itemize}
  Moreover, $F \subset \mathscr{M}$ since $F \subset \mathscr{N}$ for each $\mathscr{N} \in \Omega$.
  Finally, if $\mathscr{N}$ is a $\sigma$-algebra with $\mathscr{N} \supset F$, then $\mathscr{N} \in \Omega$. Then $\mathscr{M} \subset \mathscr{N}$.
  To prove uniqueness, let $\mathscr{M}_0$ be a $\sigma$-algebra which satisfies the required properties defining $\Omega$. By intersection operation giving $\mathscr{M}$, and $\mathscr{M}_0 \in \Omega$, $M \subset M_0$. Additionally, if $\mathscr{M}_0$ satisfies that $\mathscr{M}_0 \subset \mathscr{N}$ for each $\mathscr{N} \in \Omega$, then $\mathscr{M}_0 \subset \mathscr{M}$. Thus $\mathscr{M}_0 = \mathscr{M}$.
\end{proof}

We combine concepts of topologies and $\sigma$-algebras. 
\begin{definition}
  Let $(X, \tau)$ be any topological space. The $\sigma$-algebra, $\mathscr{B}$ generated by the topology $\tau$ is called the Borel $\sigma$-algebra. Elements of $\mathscr{B}$ are called Borel sets.
\end{definition}
\begin{definition}
  Let $X, Y$ be topological spaces. A map $f: X \to Y$ is continuous if the inverse image of any open set is open. The map $f$ is continuous at $x \in X$ if every open set $V \subset Y$ with $f(x) \in V$, there is an open set $W \subset X$ with $f(W) \subset V$.
\end{definition}

\begin{theorem}
  A map $f: X \to Y$ is continuous if and only if it is continuous at each $x \in X$.
\end{theorem}
\begin{proof}
  ($\implies$) If $f$ is continuous and $x \in X$, $V \subset Y$ is open and $f(x) \in V$, then by continuity, $f^{-1}(V)$ is open and $x \in f^{-1}(V)$. This holds for any such $x$ and $V$, thus $f$ is continuous at $x \in X$. Since $x$ was arbitrarily chosen, $f$ is continuous at each $x \in X$.

  ($\impliedby$) Suppose $f$ is continuous at each $x \in X$. Let $V$ be an open subset of $Y$. Need to show that $W = f^{-1}(V)$ is open. For each $x \in W$, there is a $W_x \subset X$ which is open with $x \in W_x$ and $f(W_x) \subset V$ by the continuity of $f$ at $x$. Now take $$Y = \bigcup_{x \in W} W_x$$
  Then $Y$ is open being a union of open sets. Also it contains each $x \in W$. Hence $W \subset Y$. But again, $W_x \subset W = f^{-1}(V)$ for each $x \in W$ and taking the unions preserve the inclusion. Hence we get $W = Y$. Since we already know $Y$ is open, this gives us $W = f^{-1}(V)$ is open.
\end{proof}

\begin{proposition}
  If $f: X \to Y$ and $f: Y \to Z$ are continuous, then so is $g\circ f: X \to Z$.
\end{proposition}
\begin{proof}
  Let $V \subset Z$ be an open set. Then $f^{-1}(V)$ is open in $Y$ by the continuity of $f$. Similarly, $g^{-1}(f^{-1}(V))$ is open in $X$ by the continuity of $g$. But $ g^{-1}(f^{-1}(V)) = (g \circ f)^{-1}(V)$. Since $V$  was arbitrarily open, we get that $g\circ f$ is continuous.
\end{proof}

\begin{definition}
   Let $X$ be a measurable space and $Y$ a topological space. Then a map $f: X \to Y$ is called measurable, if all inverse images of open sets are measurable.
\end{definition}

\begin{proposition}
  \label{prop:measurabel_sets_with_open_sets}
  Let $X$ be a measurable space, $Y$ be a topological space, then $f: X \to Y$ is measurable if and only if $f^{-1}( B)$ is measurable for each Borel set $B$.
\end{proposition}
\begin{proof}
  ($\implies$) Every open set is a Borel set. So this is true by inclusion.

  ($\impliedby$) Suppose $f$ is measurable. Let $M = \{ E \subset Y \ : \ f^{-1}(E) \textrm{  is measurable } \}$. We know $M$ contains all open sets (Since we assume $f$ is measurable). Moreover since $f^{-1}(\cup_{j \in J} U_j) = \cup_{j \in J}f^{-1}(U_j)$ for any open sets $U_j \subset Y$ with index set $J$, and $ f^{-1}(\cap_{i = 1}^{n}U_i) = \cap_{i = 1}^{n} f^{-1}(U_i)$, we get that $M$ is a $\sigma$-algebra.

  Since $M$ contains all open sets, $M$ contains the Borel $\sigma$-algebra in $Y$. Hence $f^{-1}(B)$ is  measurable for every Borel set $B$.
\end{proof}














% TeX_root = ../main.tex

\chapter{Hahn Banach Theorem}
\begin{lemma}
  \label{lem:C_linear_map_from_R_linear}
  Let $X$ be a complex normed space. Let $f: X \to \mathbb{R}$ be an $\mathbb{R}$-linear functional. Then $g: X \to \mathbb{C}$ defined as $g(x) = f(x) - i f(ix)$ is $\mathbb{C}$-linear

  Conversely if $g: X \to \mathbb{C}$ is a $\mathbb{C}$-linear map, then $f:= \Re\circ g: X \to \mathbb{R}$ is $\mathbb{R}$-linear.

  Moreover $\|f\| = \|g\|$.
\end{lemma}
\begin{proof}
  We'll prove that $ \|f\| = \|g\|$ and leave the rest for the reader (\textcolor{red}{verify}).

  Since $|f(x)| \le |g(x)|$, for all $x \in X$, it is easy to see that $\|f\| \le \|g\|$. Conversely, $\forall \epsilon > 0, \exists x_o \in X$ with $\|x_o\| = 1$ such that $|g(x_o)| > \|g\| - \epsilon$. If $g(x_o)= re^{i\theta}$, take  $\alpha = e^{-i\theta}$. Then $f(\alpha x_o) = \Re(r e^{-i \theta}e^{ i \theta}) = r = g(\alpha x_o)$. Then $\|f\| \ge |f(\alpha x_o)| =  |g(\alpha x_o)| = |\alpha||g(x_o)| = |g(x_o)| > \|g\| - \epsilon$. Since $\epsilon$ is arbitrary, this gives $\|f\| \ge \|g\|$
\end{proof}

\begin{theorem}[Hahn-Banach Extension Theorem]
  Let $X$ be a normed space over $\mathbb{R}$, $Z$ be a subspace of $X$ and let $\phi: Z \to  \mathbb{R}$ be a continuous linear functional. Then there exists a linear functional $\psi: X \to \mathbb{R}$ such that $\psi|_Z = \phi$ and $\|\phi\| = \|\psi\|$.
  \label{thm:hahn-banach-extension}
\end{theorem}
\begin{proof}
  Assume $\|\phi\| = 1$ (If this is not the case, we can always scale the functional down to norm 1). Now we'll extend $\phi$ from $Z$ to a subspace with one dimension higher than $Z$, preserving the norm. Let $x_o \in (X \setminus Z)$ and $Y = \textrm{Span}\{ \{ x_o \} \cup Z \}$ be the set one dimension higher than $Z$. Assume $ \psi$ is the extension of $\phi$ to $Y$. Then $\psi$ will be completely characterized, if we know the value of $\psi(x_o)$. We look to see what real values we can assign $\psi(x_o)$ satisfying our conditions. Let $y = z +  x_o \in Y$ where $z \in Z$ (We must be taking an arbitrary element $y = z + \alpha x_o \in Y$, but if we know the image of $y = z + x_o$ for all $z \in Z$ under $\psi$, then we can get the image of $y = z + \alpha x_o$ for any $ \alpha \in \mathbb{R}$ by scaling). Norm preserveness demands that for all $z \in Z$, we must have \[
    -\|z +  x_o\| \le \psi(y) = \psi(z) + \psi(x_o) \le \|z + x_o\| \\
  \]
  Since $\psi$ agrees with $\phi$ on $Z$, this is equivalent to \begin{equation}
    \label{eq:hahn_banach_extension-1}
  -\phi(z) - \|z + x_o\| \le \psi(x_o) \le \|z + x_o\| - \phi(z)
    \end{equation}
  Moreover since we normalized $\phi$ to have norm 1, we know $\psi$ must also have norm 1. Then by triangle inequality, we get that for all $ a, b \in Y$ \[
    \psi(a) - \psi(b)  = \psi(a - b) \le \| a - b\| = \|(a+x_o) - ( b + x_o)\| \le \|a + x_o\| + \|b+x_o\|
  \]
  which gives \[
    - \psi(b) - \|b + x_o\| \le \|a + x_o\| - \psi(a)
  \]
  Since this inequality is true for all $a, b \in Y$, taking supremum and infimum over all the possible $ a, b \in Y$ preserves the inequality. Hence we get \begin{equation}
    \label{eq:hahn_banach_extension-2}
    \sup_{b \in Y} \Big \{- \psi(b) - \|b + x_o\|\Big \} \le \inf_{a \in Y} \Big\{ \|a + x_o\| - \psi(a) \Big \}
  \end{equation}
  Substituting $a = b = z$ in \autoref{eq:hahn_banach_extension-2} guarantees the existence of $\psi(x_o)$ satisfying \autoref{eq:hahn_banach_extension-1}. Hence we get an extension (namely $\psi$) of $\phi$ to $Y$ preserving the norm. Since $Z$ was an arbitrary subspace of $X$, this is true for all such subspaces of $X$.


  Now we will employ Zorn's lemma to get an extension of $\phi$ from $Z$ to the whole of $X$. Consider the collection of all linear extensions of $\phi$, i.e \[ \mathcal{S} = \Big \{ (\psi_Y, Y) \ : \ Z \subset Y, \ \psi_Y|_Y = \phi, \ \| \psi_Y\| = \|\phi\| \Big \} \]
  Then we define a partial order in the collection $\mathcal{S}$ as $(\psi_X, X) \le (\psi_Y, Y)$ if and only if $X \subset Y$ and $\psi_Y|_X = \psi_X$. Now let $\mathscr{C}$ be a chain in $\mathcal{S}$. Consider the set \[
    \tilde{Y}_\mathscr{C} = \bigcup_{(\psi_Y, Y) \in \mathscr{C}} Y
  \]
  and the map $\psi_{\tilde{Y}_\mathscr{C}}: \tilde{Y}_{\mathscr{C}} \to \mathbb{R}$ defined as \[
    \psi_{\tilde{Y}_\mathscr{C}}(x) = \psi_Y(x), \ \textrm{ where } x \in Y, \textrm{ for } (\psi_Y, Y) \in \mathscr{C}
  \]
  To see this map is well defined, assume $x \in X$ and $x \in Y$ for $(\psi_X, X), (\psi_Y, Y) \in \mathscr{C}$. Then either $(\psi_X, X) \le (\psi_Y, Y)$ or $(\psi_Y, Y) \le (\psi_X, X)$ since $\mathscr{C}$ is totally ordered. WLOG assume $(\psi_X, X) \le (\psi_Y, Y)$, then by definition we get that $\psi_Y|X = \psi_X$. This gives that $\psi_Y(x) = \psi_X(x)$. Hence we get that $\psi_{\tilde{Y}_{\mathscr{C}}}$ is well defined. In a similar fashion we can verify that $\psi_{\tilde{Y}_{\mathscr{C}}}$ is a linear functional.

  Now we claim that $(\tilde{Y}_{\mathscr{C}}, \psi_{\tilde{Y}_\mathscr{C}})$ is the upper bound of the chain $\mathscr{C}$. By the definition of $\tilde{Y}$, we see that there cannot be an element $(\psi_Y, Y)$ in the chain $\mathscr{ C}$, with $\tilde{Y} \subset Y$. Hence the only remaining thing to show is that for all $(\psi_X, X) \in \mathscr{ C}$, we have $\psi_{\tilde{Y}_\mathscr{C}}|_X = \psi_X$. But this also follows from the definition of the map $\psi_{\tilde{Y}_\mathscr{C}}$.

  Since $\mathscr{C}$ was taken to be an arbitrary chain in the collection $\mathcal{S}$, we get that every chain in $\mathcal{S}$ has an upper bound. Then by Zorn's lemma, the collection $\mathcal{S}$ has a maximal element $(\psi, Y)$. We claim that in this maximal element, $Y = X$. If not, we can extend $\psi$ to a space one dimension above $Y$ like we did in the beginning contradicting the maximality of $(\psi, Y)$. Hence the maximal element is $(\psi, X)$. This by definition of the collection $S$, is the required extension for $(\phi, Z)$.
\end{proof}
\begin{remark}
   Note that in the proof above, we only used the scaling property and triangle inequality of the norm, hence we can relax the condition for norm and replace it with a seminorm, without messing up the proof.
\end{remark}

\begin{theorem}[Hahn-Banach Extension Theorem for $\mathbb{C}$]
  Same statement of \autoref{thm:hahn-banach-extension} with only the field changed to $\mathbb{C}$.
\end{theorem}
\begin{proof}
  Consider $X$ as a normed linear space over $\mathbb{R}$. Let $f = \Re \circ \phi: Z \to \mathbb{R}$ and apply \autoref{thm:hahn-banach-extension} on $f$ to get a real linear functional $\tilde{f}: X \to \mathbb{R}$ with the required properties. Now we claim that $\tilde{ \phi}$ defined as $\tilde{ \phi}(x) = \tilde{f}(x) - i \tilde{f}(ix)$ is the required extension.

   First we show $\tilde{ \phi}_Z = \phi$. To see this first we notice that if $ \phi$ can be written as  $\phi(x) = f( x) + ig(x)$ where $ f, g$ are real valued functionals, then since $-\phi(x) = i\phi(ix) = if(ix) - g(ix)$. Hence $0 = \phi(x) - \phi(x) = (f(x) - g(ix)) + i(g(x) + f(ix))$. Since real part and imaginary part must be equal to 0, we get that $g(x) = -f(ix)$. Therefore we get $\phi(x) = f(x) - if(ix)$.
  Now we get $\tilde{\phi}|_Z = \phi$ immediately since $ \tilde{f}|_Z = f$.
  To finish the proof, we also have to show that $\|\phi\| = \|\tilde{\phi}\|$. But this follows easily from \autoref{lem:C_linear_map_from_R_linear} as $\|\phi\| = \|f\| = \|\tilde{f}\| = \|\tilde{\phi}\|$.
\end{proof}

\begin{remark}
  It is quite natural to be confused about the well defineness of the expression $f(ix)$ when we are considering $X$ as a normed linear space over $ \mathbb{R}$ in the beginning of the proof. But note that since $X$ initially was a complex normed linear space, viewing it as a space over $\mathbb{R}$ doesn't change or remove any elements from the space. Hence $ix \in X$ even though $X$ is viewed as a real normed linear space.
\end{remark}

% TeX_root = ../main.tex

\chapter{}

\begin{definition}
  A sublinear map is a functoin $\rho: X \to \mathbb{R}$ with the properites \begin{itemize}[]
    \item $\rho(rx) = r \rho(x), \forall r \in \mathbb{R}$
    \item $\rho(x+y) \le \rho(x)+\rho(y)$
  \end{itemize}
\end{definition}

\begin{definition}
  Let $X$ be a normed space. Then the dual of $X$, denoted by $ X^{*}$, is the space $B(X, \mathbb{F})$
\end{definition}

\begin{lemma}
  \label{lem:existence_linear_functions_with_norm_peaking_at_point}
  Let $X$ be a normed space and $x \in X$. Then $\exists f \in X^{*}$ such that \[
    \|f\| = 1 \ \textrm{and } \ f(x) = \|x\|
  \]
\end{lemma}
\begin{proof}
  Let $Z = \textrm{Span}\{ x \}$. Define $g: Z \to \mathbb{F}$ as $g(\alpha  x) = \alpha \|x\|$. Then $\|g\| = 1$. By the Hahn Banach theorem, $g$ has an extension $f$ which preserve the norm and extends $g$ to $X$.
\end{proof}

\begin{corollary}
  \label{cor:dual_maps_norm_elements}
   Let $X$ be a normed space and $x \in X$, then we have \[
     \|x\| = \sup \{ |f(x)| \ : \ f \in X^{*}, \|f\| \le 1 \}
   \]
\end{corollary}
\begin{proof}
  If $f$ is any linear functional with $\|f\| \le 1$, then $|f(x)| \le \|f\|\|x\| = \|x\|$. Hence $\|x\| \le \sup \{ |f(x)| \ : \ f \in X^{*}, \|f\| \le 1 \}$.
  Now let $f_x$ be the functional we get from \autoref{lem:existence_linear_functions_with_norm_peaking_at_point}. Then $f_x \in X^{*}$ and $\|f_x\| = 1$ with $f_x(x) = |f(x)| = \|x\|$. Hence we get that the inequality is actually an equality, and this proves the corollary.
\end{proof}

\begin{definition}
  For every $x \in X$, define a linear map $\hat{x}: X^{*} \to \mathbb{F}$ by $\hat{x}(f) = f(x)$
\end{definition}

\begin{theorem}
  For every $x \in X$, $\hat{ x } \in (X^{*})^{*}$. The map $\rho: x \to \hat{x}$ is an isometric linear map.
\end{theorem}
\begin{proof}
  The fact that $ \hat{x}$ is linear and bounded and the map $X \ni x \to \hat{x} \in X^{**}$ is linear follows from the definition of $f+g$ and $\lambda f$.

  By definition and   \autoref{cor:dual_maps_norm_elements}
  \begin{align*}
    \|\hat{x}\| &= \sup \{ |\hat{x}(f) \ : \ f \in X^{*}, \|f\| \le 1 \} \\ 
    & = \sup \{ |f(x)| \ : \ f \in X^{*}, \|f\| \le 1 \} \\ 
    & = \|f\|
  \end{align*}
\end{proof}

\begin{definition}
  A normed space $X$ is said to be reflexive if the map $\rho: X \to X^{**}:= x \to \hat{x}$ is surjective. (This is a stronger condition than $X \equiv X^{**}$)
\end{definition}

\begin{theorem}
  There are isometric isomorphisms between \begin{itemize}[]
    \item $(\textbf{c}_0)^*$ and $\ell^1$
    \item 
    \item $(\ell^1)^{*} \textrm{ and } \ell^\infty$
  \end{itemize} 
\end{theorem}
\begin{proof}
  \begin{itemize}[]
    \item Let $(x_{n}) \in \ell^1$. Then consider the map $\phi_{(x_n)}: \textbf{c}_0 \to \mathbb{F}$ defined as $$\phi_{(x_n)}: (y_n) \to \sum_{n \in \mathbb{N}} x_ny_n$$
      We claim that $\phi_{(x_n)}$ is a continuous linear functional. But first we should see that the sum is well defined. Since $y_n \to 0$, there is an $N \in \mathbb{N}$ such that $|y_n| < 1$ for all $n \ge N$. Since \[
        \Big|\sum_{i = N}^{\infty} x_ny_n \Big| \le \sum_{i = N}^{\infty}  |x_n||y_n| \le \|(x_n)\|_1
      \]
      we see that the sum is well defined and the map makes sense. Also since $(y_n)+(z_n) = (y_n +z_n) \in \textbf{c}_0$ whenever $(y_n), (z_n) \in \textbf{c}_0$, we get that \[
        \sum_{n \in \mathbb{N}} x_n (y_n + z_n) = \sum_{n \in \mathbb{N}} x_n y_n + \sum_{n \in \mathbb{N}} x_n z_n
      \]
      which shows the linearity  of the map $\phi_{(x_n)}$.

      Now we show that $\|\phi_{(x_n)}\| = \|(x_n)\|_1$. We immediately see that for $(y_n) \in c_0$ with $\|(y_n)\|_{\textrm{sup}} = \sup_{n \in \mathbb{N}} y_n = 1$, \[
        |\phi_{(x_n)}((y_n))| = \Big|\sum_{n \in \mathbb{N}} x_ny_n \Big| \le \|(y_n)\|_{\textrm{sup}} \bigg( \sum_{n \in \mathbb{N}} |x_n| \bigg) \le \|(x_n)\|_1
       \]
      which gives $\|\phi_{(x_n)}\| \le \|(x_n)\|_1$. Now let $\theta_j \in [0, 2\pi)$ such that $|x_j| = e^{i \theta_j}x_j$. Now consider the sequence $s_m \in \textbf{c}_0$ defined as $s_m = \sum_{j = 1}^{m} e^{i \theta_j}e_j$, where $e_j$ is  the sequence with $j$th entry $1$ and the rest of the entries $0$. 
      Since $(x_n) \in \ell_1$, for all $\epsilon \ge 0$ there exists an $N_\epsilon \in \mathbb{N}$ such that \[
        \sum_{i = N_\epsilon+1}^{\infty} |x_i| < \epsilon
      \]
      Then since \[
        |\phi_{(x_n)}(s_{N_\epsilon})| = \Big| \sum_{n = 1}^{N_\epsilon} e^{i \theta_j}x_n \Big| = \sum_{i = 1}^{N_\epsilon} |x_n| = \|(x_n)\| - \sum_{i = N_\epsilon +1}^{\infty} |x_n| \ge \|(x_n)\| - \epsilon
      \]
      and $\epsilon > 0$ was arbitrary, we get that $\|\phi_{(x_n)}\| = \|(x_{n})\|$

      Hence we see that the map $(x_n) \to \phi_{(x_n)}$ is an isometric linear map. Now for surjectivity, let $\phi \in \textbf{c}_0^{*}$. We claim that the sequence $(y_n) = (\phi(e_n)) \in \ell^1$ and $\phi = \phi_{(y_n)}$. Let $\theta_j \in [0, 2\pi)$ such that $e^{ i \theta_j}y_j = |y_j|$. Then for any $N \in \mathbb{N}$, we have \begin{align*}
        \sum_{j = 1}^{N} |\phi(e_j)| &=  \sum_{j = 1}^{N} e^{i \theta_j}\phi(e_j) \\ 
        &= \phi \Big(\sum_{j = 1}^{N} e^{i \theta_j} e_j \Big) \\ 
        &\le \|\phi\| \Big \| \sum_{j = 1}^{N} e^{i \theta_j} e_j \Big \| \\ 
        &= \|\phi\|
        \end{align*}
       Since this is true for all $N \in \mathbb{N}$, taking the limits as $N \to \infty$, the inequality is preserved and we get that $(y_n) \in \ell^1$. Moreover $\phi = \phi_{(y_n)}$ follows from the definition of $ \phi_{(x_n)}$. Hence we get that $  \textbf{c}_0^{*} \cong^{\textrm{iso}} \ell^1$.
    \item 
    \item The proof of this will be extremely similar to what we attempted before when we proved $\textbf{c}_0^{*} \cong^{\textrm{iso}} \ell^1$. Let $(x_n) \in \ell^\infty$. Then consider the map $\phi_{(x_n)}: \textbf{c}_0 \to \mathbb{C}$ defined as $$\phi_{(x_n)}: (y_n) \to \sum_{n \in \mathbb{N}} x_ny_n$$
      By a similar way as we did in the above equivalence we see that $\phi_{(x_n)}$ is linear. Moreover since \[
        \Big | \sum_{n \in \mathbb{N}} x_n y_n \Big | \le \|(x_n)\|_\infty \Big | \sum_{n \in \mathbb{N}} y_n \Big | = \|(x_n)\|_\infty \|(y_n)\|_1
      \]
      we see that $\|\phi_{(x_n)}\| \le \|(x_n)\|_{\infty}$. To get the reverse inequality, Let $\|(x_n)\|_\infty = M$, then for any $\epsilon >0$, there exist some $x_k$ in the sequence $(x_n)$ such that $|x_k - M| < \epsilon$. Now consider the sequence $e_k \in \ell^1$ with $k$th entry $1$ and all the rest of them $0$. We get that \[
        |\phi_{(x_n)}(e_k)| = |x_k| \ge \|(x_n)\|_\infty - \epsilon
      \]
      Since $\epsilon$ was arbitrary, we get that $\|\phi_{(x_n)}\| = \|(x_n)\|_{\infty}$. Hence the map $(x_n) \to \phi_{(x_n)}$ is an isometry. To show that it is indeed a bijection, assume $\phi \in (\ell^1)^{*}$, then consider the sequence $y_n = \phi(e_n)$. Since $\phi$ is continuous, it is bounded above by $ \|\phi\|$ and we get that $ y_n \le \|\phi\|$. Therefore $(y_n) \in \ell^\infty$. Moreover we can verify like above that $\phi = \phi_{(y_n)}$ from the definition of $\phi_{(y_n)}$. Hence we get $(\ell^1)^{*} \ \cong^{\textrm{iso}} \ell^\infty$.
  \end{itemize}
\end{proof}

\begin{theorem}
  Let $1 < p < \infty$, and $ q \in \mathbb{R}$ such that $ \frac{1}{p} + \frac{1}{q} = 1 $. Then $(\ell^p)^{*} \cong \ell^q$
\end{theorem}
\begin{proof}
  Let $(a_n) \in \ell^p, (b_n) \in \ell^q$, then $\sum_{n \in \mathbb{N}} a_n \bar{b_n}$ is the map to check for isometric isomorphism.
  Use Holder's inequality as needed. \textcolor{red}{verify}
\end{proof}

\begin{theorem}
  There exists $\phi \in (\ell^\infty)^{*}$ satisfying the following \begin{enumerate}[label=\arabic*]
    \item $\forall (a_n) \in \ell^\infty$ with $a_n \ge 0$ for all $ n \in \mathbb{N}$, $\phi((a_n)) \ge 0$
    \item If $(a_n)$ is convergent, then $\phi((a_n)) = \lim_{n \to \infty}  a_n$
    \item If $(a_n) \in \ell^\infty$ and $b_n = a_{n+1}$, then $\phi((b_n)) = \phi((a_n))$
  \end{enumerate}
  Moreover such $\phi$ is called a Banach limit.
\end{theorem}
\begin{proof}
  We'll prove this later.
\end{proof}


\begin{corollary}
  $\ell^1$ is not reflexive
\end{corollary}
\begin{proof}
  Let $\phi \in (\ell^\infty)^{*}$ be a Banach limit. FTOC, assume $\exists f = (\alpha_n) \in \ell^1$ such that  \[
    \phi((a_n)) = \sum_{i = 1}^{\infty}  a_n  \overline{\alpha_n}
  \]
  Then for all $m \in \mathbb{N}$, $\overline{\alpha_m} = \phi(\delta_m) = 0$, where $\delta_m = (0, 0, \ldots, 1, 0, 0, \ldots)$. But this contradicts since we assumed $\phi \neq 0$ by the Hahn Banach rextension from $c_0$
\end{proof}

\begin{lemma}
  Let $\psi \in (\ell^\infty)^{*}$. then the following are equivalent. \begin{enumerate}[label=\arabic*]
    \item $\|\psi\| = \psi((1, 1, 1, \ldots))$
    \item If $(a_n) \in \ell^\infty$ with $a_n \ge 0, \forall n \in \mathbb{N}$. Then $ \psi((a_n)) \ge 0$
  \end{enumerate}
\end{lemma}
\begin{proof}
  ($1 \implies 2$) FTSOC assume $\exists (a_n) \in \ell^\infty$, $\psi((a_n)) < 0$. WLOG, assume $|a_n| \le 1$ for all $n \in \mathbb{N}$. let $b_n = 1- a_n$. Then $0 \le b_n \le 1$ and \[
    \psi((b_n)) > \psi((1, 1, 1, \ldots)) -    \psi((a_n)) > \psi((1, 1, 1, \ldots))
  \]
   So \[
     \|\psi\| \ge |\psi((b_n))| \ge \psi((1, 1, \ldots))
   \]

   ($2 \implies 1$) Let $(a_n) \in \ell^\infty$ with $|a_n| \le 1$, then $0 \le 1-a_n$. So $\psi((1-a_n)) \ge 0$ and therefore $\psi((1, 1, 1, \ldots)) \ge \psi((a_n))$. Similarly $\psi((-a_n)) \le \psi((1, 1, 1, \ldots))$ which gives $|\psi((a_n))| \le \psi((1, 1, 1, \ldots))$
\end{proof}




% TeX_root = ../main.tex

\chapter{}

\begin{theorem}
  There exists $\phi \in (\ell^\infty)^{*}$ satisfying the following \begin{enumerate}[label=\arabic*]
    \item $\forall (a_n) \in \ell^\infty$ with $a_n \ge 0$ for all $ n \in \mathbb{N}$, $\phi((a_n)) \ge 0$
    \item If $(a_n)$ is convergent, then $\phi((a_n)) = \lim_{n \to \infty}  a_n$
    \item If $(a_n) \in \ell^\infty$ and $b_n = a_{n+1}$, then $\phi((b_n)) = \phi((a_n))$
  \end{enumerate}
  Moreover such $\phi$ is called a Banach limit.
\end{theorem}
\begin{proof}
  Let $S: \ell^\infty(\mathbb{R} \to \ell^\infty(\mathbb{R})$ and $T = I - S$ where $I$ is the identity map. Also let $V = \textrm{Range}(T) + c$ where $ c \in \textbf{c}$, the set of convergent sequences.

  Define $\phi: V \to \mathbb{R}$, $\phi(a_n - a_{n+1} + x_n) = \lim_{n \to \infty}  x_n$.
  \begin{itemize}[]
    \item Claim 1: $\phi$ is well defined
    \item Claim 2: $\|\phi\| = 1$
  \end{itemize}

  Assuming the claims, by Hahn Banach, $\phi$ extends to $\tilde{ \phi} \in \ell^\infty(\mathbb{R})$ with $\|\tilde{\phi}\|= 1$. Then by the last lemma we get $ \tilde{\phi}((y_n)) \ge 0$ for all $(y_n) \in ell^\infty(\mathbb{R})$ with $y_n \ge 0$
\end{proof}
 \begin{proof}[Proof of Claim 1]
   Suppose that $(a_n) \in \ell^\infty$ is a sequence such that $a_{n} - a_{n+1}$ converges, say $a_n \to a_{n+1} \to \alpha$. If $ \alpha > 0$, then $\exists N \in \mathbb{N}$ such that for all $n > N$, $ a_n - a_{n+1} > \frac{\alpha}{2}$. So $a_N > \frac{\alpha}{2} + a_{N+1} > \ldots > k\frac{\alpha}{2} + a_{N+k}$. So for all $k \in \mathbb{N}$, $a_N - a_{N+k} = k \frac{\alpha}{2} \to \infty$ contradicting our assumption that $a_n - a_{n+1}$ converges. 

   Now assume that $(a_n), ( b_n) \in \ell^\infty(\mathbb{R})$ with $(x_n), (y_n) \in \textbf{c}$ such that $a_n - a_{n+1} + x_n = b_n - b_{n+1} + y_n$. Then $(a_n - b_n) - (a_{ n+1} - b_{n+1}) = y_n -x_n$. Then since  RHS is a convergent limit, LHS must be convergent, which we get from above that it must converge to zero. Then $\lim_{n \to \infty} x_n = \lim_{n \to \infty} y_n$
 \end{proof}
\begin{proof}[Proof Claim 2]
   \textcolor{red}{verify}
\end{proof}
To complete the proof, define $ \Psi: \ell^\infty \to \mathbb{C}$ by $\Psi((a_n + ib_n)) = \tilde{\phi}(a_n) + i \tilde{ \phi}(b_n)$ \textcolor{red}{verify}

\section{Quotient Spaces}
\begin{definition}
  Let $X$ be a normed space and $Y \leqslant X$ be a closed subspace. For every $x \in X$, define \[
       \|x + Y\| = \inf \{ \|x+y\| \ : \ y \in Y \}
  \]
\end{definition}
\begin{lemma}
  This defines as norm on  $\frac{X}{Y}$. If $X$ is complete, then $ \frac{X}{Y}$ is complete.
\end{lemma}
\begin{proof}
  Obviously, $\|x+Y\| \ge 0 $ for all $ x \in X$, and $\|x+z + Y\| \le \|x+Y\| + \|y + Y\|$. Similarly, we can also show homogeneity.

  Now assume $x \in X$ is such that $\|x+Y\| = 0$. Then there is a sequence $(y_n) \in Y$ such that $ \|x - y_n\| \to 0$, that is $y_n \to x$. Since $Y$ is closed, we get $x \in Y$.

  To show the second part of the lemma, consider the sequence $(x_n + Y) \in X/Y$ such that $\sum_{n \in \mathbb{N}} \|x_n - Y\| < \infty$. For each $ n \in \mathbb{N}$, choose $y_n \in Y$ such that \[
    \|x_n + y_n\| \le \|x_n + Y\| + \frac{1}{2^n}
  \]
  Then $\sum_{n \in \mathbb{N}} \|x_n + y_n\| \le \infty$. Since $X$ is complete, the sequence $ \sum_{n \in \mathbb{N}} x_n + y_n$ converges to say $z \in X$. Then \begin{align*}
    \|(z+Y) - \sum_{n = k}^{n} (x_n +Y)\| &= \|\Big( z - \sum_{n = k}^{n} x_n \Big) + Y\| \\ 
    & = \|\Big( z - \sum_{i = 1}^{k}(x_n + y_n) \Big) + Y\| \\ 
    & \le \|\Big( z - \sum_{i = 1}^{k}(x_n + y_n) \Big) \|
  \end{align*}
  which converges to 0 as $k \to \infty$
\end{proof}

\begin{lemma}
  The canocial map, $q: X \to \frac{X}{Y}$ is a continuous open map. A subset $E \subset X/Y$ is open iff $ q^{-1}(E) \subset X$ is open.
\end{lemma}
\begin{proof}
  Since $\|x + Y\| \le \|x\|$,  for all $x \in X$, we see that the map $q$ is a contraction. Thus  for all open $E \subset X/Y$, we get $ q^{-1}(E)$ is open.

  Conersely, assume that $  A \subset X$ is open. Let $ x \in A$ and $ r > 0$ such that $B_r(a) \subset A$. Let $  z \in X$ such that $\|q(a) - q(z)\| < r$. So, $\|(a-z) + Y\| < r$. Then $\exists y \in Y$ such that $ \|a-z-y\| < r$. So $z+y \in B_r(a), q( z+y) = q( z) \in q( B_r(a))$. So $B_r(q(a)) \subset q(  B_r(a)) \subset q(A)$. Thus $q(A)$ is open.
\end{proof}

% TeX_root = ../main.tex

\chapter{}

\section{Integrals}

\begin{definition}
  Define the integral of a measurable simple function $s: X \to [0, \infty]$ defined in the standard form as \[
      s = \sum_{j = 1}^{n} \alpha_j \chi_{A_j}
  \]
  with $\{ \alpha_1, \alpha_2, \ldots, \alpha_n \}$ as the range of $S$ and $A_j = s^{-1}(\{ \alpha_j \})$ by \[
    \int s \ d \mu = \sum_{j = 1}^{n} \alpha_j \mu(A_j)
  \]
\end{definition}
We adopt the convention $0\times \infty = 0$ from now onwards.

\begin{lemma}
  Let $(X, \mathcal{M}, \mu)$ be a measure space. Let $A_1 , A_2 , \ldots , A_n \in \mathcal{M}$ and $B_1 , B_2 , \ldots , B_{n^\prime} \in \mathcal{M}$ with the $A_j$s are mutually disjoint, as well as $B_j$s, and \[
      \bigcup_{j = 1}^{n}A_j = X = \bigcup_{j = 1}^{n^\prime}B_j
  \]
  Let $\alpha_1 , \alpha_2 , \ldots , \alpha_n \in [0, \infty]$ and $  \beta_1 , \beta_2 , \ldots , \beta_n^\prime \in [0, \infty]$ such that \[
        t = \sum_{j = 1}^{n^\prime} \beta_j \chi_{B_j} \le s = \sum_{j = 1}^{n} \alpha_j \chi_{A_j}
  \]
  then \[
    \sum_{j = 1}^{n^\prime} \beta_j \mu(B_j) \le \sum_{j = 1}^{n} \alpha_j \mu(A_j)
  \]
\end{lemma}
\begin{proof}
  \begin{align*}
    \sum_{j = 1}^{n^\prime} \beta_j \mu(B_j) &=  \sum_{j = 1}^{n} \beta_j \mu\Big(B_j \bigcap \big(\bigcup_{l = 1}^{n} A_l\big)\Big) \\ 
    &= \sum_{j = 1}^{n^\prime} \beta_j \mu \Big(\bigcup_{l = 1}^{n}B_j \cap A_l\Big) \\ 
    &= \sum_{ j = 1}^{n^\prime} \sum_{l = 1}^{n} \beta_j\mu \Big( B_j \cap A_l\Big) \\ 
  \end{align*}
  By a similar deduction, we get that \[
    \sum_{l = 1}^{n} \alpha_j \mu(A_j) = \sum_{ l = 1}^{n} \sum_{j = 1}^{n^\prime} \alpha_l \mu(A_l \cap B_j)
  \]
  Since we know that $t \le s$, comparing the values of the function at $A_l \cap B_j$, we get that $\beta_j \le \alpha_l$. This immediately gives us our needed result.
\end{proof}

\begin{corollary}
  If a measurable simple function has two representations \[
      s = \sum_{j = 1}^{n} \alpha_j \chi_{A_j} = \sum_{j = 1}^{n^\prime} \beta_j \chi_{B_j}
  \]
  with disjoint measurable sets as before, then \[
    \int s \ d \mu = \sum_{j = 1}^{n} \alpha_j \mu(A_j) = \sum_{ j = 1}^{n^\prime} \beta_j \mu(B_j)
  \]
\end{corollary}
\begin{proof}
   Use the fact that $a = b$ is equivalent to $ a \le b$ and $ b \le a$ and use above lemma.
\end{proof}

\begin{definition}
  Let $(X, \mathcal{M}, \mu)$ be a mesurable space, $s: X \to [0, \infty]$ a measurable simple function, \[
      s = \sum_{j = 1}^{n} \alpha_j \chi_{A_j}
  \]
  with $\{ A_j \}_{j=1}^n$ disjoint, measurable, then we define for $E \in \mathcal{M}$ \[
    \int_E s \ d \mu = \sum_{j = 1}^{n} \alpha_j \mu(A_j \cap E)
  \]
\end{definition}

\begin{lemma}
   If $s, t$ are non-negative measurable, simple fucntions and $t \le s$ and $E \in \mathcal{M}$, then \[
       \int_E t \ d \mu \le \int_E s \ d \mu
   \]
\end{lemma}
\begin{proof}
  Proof is exactly like before lemma, just replacing $\mu(A_j)$ with $\mu(A_j \cap E)$.
\end{proof}

\begin{remark}
  If $s: X \to [0, \infty]$ is simple and measurable, then   \[
      \int s \ dx = \sup \{ \int_E t d \mu \ : \ 0 \le t \le s \textrm{ is measurable and simple.} \}
  \]
\end{remark}

\begin{definition}
  For $f: X \to [0, \infty]$ measurable, we define \[
    \int_E f d \mu = \sup_{\substack{ 0\le t\le f \\ t \textrm{ is simple}}} \int_E t \ d \mu
  \]
\end{definition}
\begin{example}
  We will give some examples of measurable functions.
  \begin{itemize}[]
    \item $X = \mathbb{N}, \mathcal{M} = P(\mathbb{N}), \mu$ is the counting measure. $f: \mathbb{N} \to [0, \infty]$. Then let \[
        s_N(n) = \begin{cases}
          f(n), & n \le N \\ 
          0, &\textrm{otherwise}
        \end{cases}
    \]
      Now if $\sum_{j = 1}^{\infty} f(j) \le \infty$, then $f(j) \to \infty$ as $j \to \infty$. Thus if $t \le f$ and $t$ is simple, then there is $ N \in \mathbb{N}$ such that $t(j) = 0$ for each $ j \ge N$. Then by comparison, $0 \le t \le s_n \le f$ and finally, we have \[
        \sum_{j = 1}^{\infty}  t(j) \le \sum_{ j = 1}^{\infty}  s_N(j) \le \sum_{ j = 1}^{\infty}  f(j)
      \]
      so taking supremums, we get \[
        \sup_{\substack{0\le t \le f\\ t \textrm{ is simple}}}  \sum_{j = 1}^{\infty}  t(j) = \sup_{N \in \mathbb{N}}\sum_{ j \in \mathbb{N}} s_N(j) = \sum_{ j = 1}^{\infty}  f(j)
      \]
  \end{itemize}
\end{example}


























% TeX_root = ../main.tex

\chapter{}

\begin{remark}
  \label{remark:measure_from_integral}
  Let $(X, \mathcal{M}, \mu)$ be a measure space, a simple function $s: X \to [0, \infty]$, then $\phi: \mathcal{M} \to [0, \infty]$ defined as \[
    \phi(E) = \int_E s \ d \mu
  \]
  is a measure.
\end{remark}
\begin{proof}
  Since our definiton demands that measure of some set should be finite, we verify this first. We see that \[
    \phi(\emptyset) = \int_\emptyset s \ d \mu = 0
  \]
  Now to prove countable disjoint additivity, consider the disjoint collection $\{ E_l \}_{l \in \mathbb{N}}$. And assume that $s = \sum_{j = 1}^{n} \alpha_j \chi_{A_j}$ with $ \alpha_j \in [0, \infty]$, with $A_j$s disjoint. Then for $E = \cup_{l = 1}^{\infty}E_l$, we have \begin{align*}
    \phi(E) &= \sum_{j = 1}^{n} \alpha_j \mu(A_j \cap E) \\ 
    &= \sum_{ j = 1}^{n} \sum_{l \in \mathbb{N}} \alpha_j \mu(A_j \cap E_l) \\ 
    &= \sum_{ l \in \mathbb{N}} \sum_{j = 1}^{n} \alpha_j \mu(A_j \cap E_l) \\ 
    &= \sum_{ l \in \mathbb{N}} \int_{E_l} s \ d \mu
  \end{align*}
\end{proof}


\section{Properties of Integrals}

\begin{theorem}
  \label{thm:properties_of_integrals}
  The interal of a non-negative measurable function from a measure space $(X, \mathcal{M}, \mu)$ has the following properties \begin{enumerate}[label=(\arabic*)]
    \item If $0 \le f \le g$, then $\int_ E f(x) \ dx \le \int_E g \ d \mu$
    \item If $A \subset B$, $A, B \in \mathcal{M}$, then $\int_A f \ d \mu \le \int_B f \ d \mu$
    \item If $c \in [0, \infty)$, $ E \in \mathcal{M}$, then $\int_E cf \ d \mu = c \int_E f \ d \mu$
    \item If $f = 0$, or $ \mu(E) = 0$, then $\int_E f \ d \mu = 0$
    \item For all $ E \in \mathcal{M}$, \[
        \int_E f \ d \mu = \int_X f \chi_{E} \ d \mu
    \]
  \end{enumerate}
\end{theorem}
\begin{proof}
  \begin{enumerate}[label=(\arabic*)]
    \item By definition \[
        \int f \ d \mu = \sup_{\substack{t \textrm{ is simple} \\ t \textrm{ is measurable} \\ 0 \le t \le f}} \int_E  t \ d \mu
    \]
      then the simple function  $t \le f$ is also $ t \le g$. Hence suping over simple functions under $g$, every simple function under $ f$ is included.
    \item Let $s = \sum_{i = 1}^{n} \alpha_i \chi_{A_i}$ be a simple function $0 \le s \le f$ with $\int s \ dx + \epsilon > \int f \ d \mu$.
      Using the inclusion $ A \subset B$, we get \begin{align*}
        \int_A s \ d \mu &= \sum_{n \in \mathbb{N}} \alpha_n
      \end{align*}

    \item Suppose $s = \sum_{j = 1}^{n} \alpha_j \chi_{A_j}$ is a simple function with disjoint $A_j$s. Then $s \chi_{E} = \sum_{j = 1}^{n} \alpha_j \chi_{A_j \cap E}$ is also simple (and measurable), and \[
        \int_E s \ dx = \sum_{j = 1}^{n} \alpha_j \mu(A_j \cap E) = \int  s \chi_{E} \ dx
    \]
      Hence the statement is true for simple measurable functions. Next, consider $f$ non-negative measurable, then for $\epsilon \ge 0$, we have a simple measurable function $s$ with $\int_E s \ d \mu + \epsilon > \int_E f \ d \mu$. Then by preceding part, \[
          \int s \chi_{E} \ d \mu + \epsilon > \int_E f \ d \mu
      \]
      Also $s \chi_E \le f \chi_E$. So \[
          \int f \chi_E \ d \mu + \epsilon \ge \sup_{t \textrm{ is simple}} \int s \chi_E \ d \mu + \epsilon > \int f \ d \mu
      \]
      Taking $\epsilon \to 0$ gives \[
          \int f \chi_E \ d \mu \ge \int_E f \ d \mu
      \]
      For the reverse inequaltiy, note that $f \chi_E \le f$, and use similar circus.
  \end{enumerate}
\end{proof}

\begin{theorem}[Monotone convergence theorem]
  Let $(X, \mathcal{M}, \mu)$ be a measure space, given a sequence $f_n: X \to [0, \infty]$ of measurable functions and they are monotone increasing, i.e for each $x \in X$, $0 \le f_1(x) \le f_2(x) \le \ldots $, then \[
       \lim_{n \to \infty} \int f_n \ d \mu = \int \lim_{n \to \infty} f_n \ d \mu
  \]
\end{theorem}
\begin{proof}
  Let $f = \lim_{n \to \infty} f_n$ be the pointwise limit. Then $ f$ is measurable. From $f_n \le f_{n+1}$, we get that \[
       \int f_n \ d \mu \le \int f_{n+1} \ d \mu
  \]
  so both sides of the claimed identity exist, and from $f_n \le f$, we also know that \[
       \int f_n \ d\mu \le \int f \ d \mu
  \]
  which taking the limits give us, \[
       \lim_{n \to \infty} \int f_n \ d\mu \le \int f \ d \mu
  \]

  Now let $s: X \to [0, \infty]$ be a simple measurable function $s \le f$. Choose $0 \le c < 1$, and define $ E_n = \{ x \in X \ : \ f_n(x) \ge cs(x) \} = (f_n - s)^{-1}([0, \infty])$. \textcolor{red}{Verify that difference between an extended real valued function and a real valued function is measurable, then $E_n$ is measurable}. This gives a nested sequence $E_1 \subset E_2 \subset \ldots $.
  If $ f(x) > 0$, then by $ f(x) > cs(x)$ and $  f_n(x) \to f(x)$, there is $n \in \mathbb{N}$ such that $x \in E_n$. 
  On the other hand if $f(x) = 0$, then $cs(x) = 0 = f(x)$, so $ x \in E_n$ for all $n \in \mathbb{N}$.
  We see that each $ x \in X$ is in the  union $\cup_{n = 1}^{\infty}E_n$. Hence $X = \cup_{n = 1}^{\infty}E_n$. Now we define $ \phi: \mathcal{M} \to [0, \infty]$ by \[
    \phi(E) = \int_E s \ d \mu
  \]
  which is a measure and $\phi(X) = \phi(\cup_{n = 1}^{\infty}E_n) =  \lim_{n \to \infty} \phi(E_n)$ by \autoref{thm:properties_of_integrals}. We rewrite this as \begin{align*}
    \int_X s \ d \mu &= \lim_{n \to \infty} \int_{E_n} s \ d \mu \\ 
    &= \lim_{n \to \infty} \int_X s \chi_{E_n} \ d \mu \\ 
    &\le \lim_{n \to \infty} \int_X \frac{1}{c} f_n \ d \mu
  \end{align*}
  Now take sup over all such simple (bounded) functions $s \le f$ and let $c \to 1$.
  \textcolor{red}{Finish this proof}.
\end{proof}

% TeX_root = ../main.tex

\chapter{}

\begin{theorem}[Tychonoff]
  The product of compact sets is compact.
\end{theorem}
\begin{corollary}
   Let $X$ be a compact Hausdorff space. Then for any set $S$, the set $\{ \phi: S \to X \} = X^S$ is compact wrt to pointwise convergence.
\end{corollary}

\begin{theorem}[Banach-Alaoglu Theorem]
  Let $X$ be a normed space. Then the closed unit ball $\overline{B_{X^*}(0, 1)} = \{ f \in X^* \ : \ \|f\| \le 1 \} = E$ is weak * compact.
\end{theorem}
\begin{proof}
  Let $\bar{B}$ be the closed unit ball of $X$. Then by Tychonoff theorem, $\bar{D}^{\bar{B}}$ is compact. Define $\phi: E \to \bar{D}^{\bar{B}}$ as $\phi(f)(x) := f(x)$. Observe that $\phi$ is injective.
  Also observe that $\phi$ is continuous (weak * in LHS, and pointwise in RHS).

  Next we show that the image of $\phi$ a closed subset of $\bar{D}^{ \bar{B}}$, hence compact. Let $f_i$ be a net in $E$ such that $\phi(f_i) \to \psi$ pointwise for some $\psi \in \bar{D}^{\bar{B}}$. 

  Define $g: X \to C$ as $g(x) = \alpha \psi( \frac{x}{ \alpha} )$ where $\|x\| \le \alpha$. For this to be well defined we must have $\|x\| \psi(\frac{x}{\|x\|}) = \alpha \psi( \frac{x}{ \alpha})$ for any $ \alpha >  \|x\|$. But we get this since $\psi$ is a pointwise limit of linear functionals. Moreover we get that $g$ is linear for the same reason. Thus $\psi = \phi(g)$ and so $\phi(E)$ is closed.

  It only remains to show that the inverse of $\phi$ is continuous. \textcolor{red}{verify}.
\end{proof}

\begin{remark}
  The closed unit ball of a normed space $Y$ is compact w.r.t the norm topology if and only if $Y$ is finite dimensional.
\end{remark}
\begin{proof}
   \textcolor{red}{verify}
\end{proof}

\begin{theorem}
  Let $X$ be a normed space. Then $E$ is weak * metrizable iff $X$ is separable. 
\end{theorem}
\begin{proof}
  Assume $X$ is separable. Let $\{ x_n  \ : \  n \in \mathbb{N} \}$ be a dense subset of $X$. For every $f, g \in E$, define $d(f, g) := \sum_{n \in \mathbb{N}}  \frac{1}{2^n}|f( \frac{x_n}{\|x_n\|} - g( \frac{x_n}{\|x_n\|} )|$. Check that $d$ is a metric.

  Now asssume $f_i \to f$ weakly in $E$. Then $f_i(x_n) \to f(x_n)$ for all $n \in \mathbb{N}$ and $d(f_i, f) \to 0$ (\textcolor{red}{verify}). 

  % Conversely if $d(f_i, f) \to 0$ then $f_i(x_n) \to f(x_n)$ for all $n \in \mathbb{N}$. 
  Assume $E$ is metrizable. Then $\exists \{ U_n \ : \ n \in \mathbb{N} \}$ of weak * open neighborhoods of $0$ such that $\cap_{n = 1}^{\infty}U_n = \{ 0 \}$. So, for each $ n \in \mathbb{N}$, there exists a finite set $ A_n \in X$ and $\epsilon > 0$ such that the (subbasis sets) $\{ f \in E \ : \ |f|\le \epsilon \forall x \in A_n  \subset U_n$. Now let $A = \cup_{n = 1}^{\infty}A_n$. Let $\phi \in E$ such that $\phi(x) = 0$ for all $x \in A$. 
\end{proof}

% TeX_root = ../main.tex

\begin{definition}
  Let $X$ and $Y$ be normed spaces and $T \in B(X, Y)$. The adjoint
  of $T$, denoted by $T^{*} \in B(Y^{*}, X^{*})$, is the map $T^{*}:
  f \to f\circ T$
\end{definition}

\begin{proposition}
  $\|T\| = \|T^*\|$
\end{proposition}
\begin{proof}
  $|T^*(f)| \le \|f \circ T\| \le \|T\|\|f\|$ implies $ \|T^{*}\| \le \|T\|$
  \begin{align*}
    \|T^*\| & = \sup \{ \|T^{*}(\phi)\| \ : \ \phi \in Y^*, \|\phi\| \le 1 \} \\
    &= \sup \{ |\phi(T(x))| \ : \ \phi \in Y^*, x \in X, \|\phi\| \le
    1, \|x\| \le 1 \} \\
    &= \|T\|
  \end{align*}
  Note that the last equality is a consequence of HBT since it
  guarantees the existence of $\phi_y \in Y^*$ with $\|\phi_y\|\le 1$
  and $\phi_y(y) = |y|$.
\end{proof}

\begin{lemma}
  For any $T \in B(X, Y)$, $T^*: Y^* \to X^*$ is weak * continuous
  (in both spaces)
\end{lemma}
\begin{proof}
  Let $\phi_i \to \phi$ weakly in $Y^*$. Then by definition for all
  $y \in Y$, $\phi_i(y) \to \phi(y)$. Then for $x \in X$,
  $T^*(\phi_i)(x) = \phi_i(T(x)) \to \phi(T(x)) = (T^*(\phi))(x)$
  which shows the continuity of $T^*$.
\end{proof}

\begin{lemma}
  For any normed space $X$, $i_X(X)$ is weak * dense in $X^{**}$.
\end{lemma}
\begin{proof}
  \textcolor{red}{verify}
\end{proof}

\begin{example}
  Is $i_{X^*}$ weak * - weak * continuous.
\end{example}

\section{Locally Convex Topological Vector Spaces}

\begin{lemma}
  Let $X$ be a normed space and $x_1 , x_2 , \ldots , x_n \in X$ and
  $\epsilon_1 , \epsilon_2 , \ldots , \epsilon_n \ge 0$. Then the set \[
    \bigcup_{x_1 , x_2 , \ldots , x_n, \epsilon_1 , \epsilon_2 ,
    \ldots , \epsilon_n}(\phi)
  \]
  is convex.
  \marginnote{ \scriptsize Review convexity arguments in Arveson's
  `Subalgebras of C* Algebras'}
  Moreover any topological vector spaces with the topology induced by
  a family of seminorms is locally convex.
\end{lemma}
\begin{proof}
  \textcolor{red}{verify}
\end{proof}

\begin{definition}
  Let $X$ be a vector space and $E \subset X$ be a convex subset. An
  element $a \in E$ is called an extreme point of $E$ if whenever $x,
  y \in E$, $0 \le t \le 1$ with $a  = tx + (1-t) y$, then $x = y = a$.
\end{definition}
\begin{example}
  Let $\bar{D} = \{ \alpha \in \mathbb{C} \ : \ |\alpha| \le 1 \}$.
  Then $\bar{D}$ is convex with $\textrm{Ext}(\bar{D}) = S^1$
\end{example}

\begin{theorem}[Krein-Milman Theorem]
  Let $X$ be a locally convex space, and let $K$ be a compact convex
  subset of $X$. Then the $ \textrm{Ext}(K) \neq \emptyset$ and
  indeed $K = \overline{\textrm{co}}(\textrm{Ext}(K))$
\end{theorem}

\begin{definition}
  Let $V$ be a vector space and $S\subset V$. The convex hull of $S$
  is defined as \[
    \textrm{co}(S) = \Big \{ \sum_{i = 1}^{n} t_i x_i \ \Big| \ 0 \le
    t \le 1, \sum t_i = 1, x_i \in S \Big\}
  \]
\end{definition}


% TeX_root = ../main.tex

\chapter{}

\begin{lemma}
  Let $K$ be a convex set. $x_0 \in \textrm{Ext}(K)$ if and only if
  $K \setminus \{ x_0 \}$ is convex.
  \label{lem:extreme_point_of_convex_set}
\end{lemma}
\begin{proof}
  If $K \setminus \{ x_0 \}$ is not convex, since $K$ is convex,
  $x_0$ can be written as the convex combination of elements in
  $K\setminus \{ x_0 \}$
  which makes $x_0 \notin \textrm{Ext}(K_0)$. Conversely if $x_0 \in
  \textrm{Ext}(K)$, then
  $x_0$ cannnot be written as the convex combination of elements of
  $K$. Hence $K \setminus \{ x_0 \}$
  is closed under convex combinations, making in convex.
\end{proof}

\begin{theorem}[Krein-Milman Theorem]
  Let $X$ be a locally convex space, and let $K$ be a compact convex
  subset of $X$. Then the $ \textrm{Ext}(K) \neq \emptyset$ and
  indeed $K = \overline{\textrm{co}}(\textrm{Ext}(K))$
\end{theorem}
\begin{proof}
  We first prove that the $\textrm{Ext}(K) \neq \emptyset$.
  Note that $K \setminus \{ x_0 \}$ is a
  relatively open subset of $K$ since $\{ x_0 \}$ is closed and $K
  \setminus \{ x_0 \} = \{ x_0 \}^c$ relative to $K$.

  Now let $ \mathcal{A}$ be the collection of all relatively open
  convex proper subsets of $K$. Note that $\emptyset \in
  \mathcal{A}$, therefore $\mathcal{A}$ is nonempty. Equip
  $\mathcal{A}$ with the partial order defined by the set inclusion.
  Let $\mathscr{C}$ be a chain in $\mathcal{A}$ and $F_{\mathscr{C}}
  = \cup_{C \in \mathscr{C}} C$. $F_{ \mathscr{C}}$ is relatively
  \marginnote{\scriptsize Zorn's Lemma to find a maximal proper open
  convex subset of $K$}
  open being the union of relatively open subsets of $K$. To see that
  $F_{\mathscr{C}}$ is convex, let $x,  y \in F_{\mathscr{C}}$. Then
  since $\mathscr{C}$ is a chain, there exist a $C \in \mathscr{C}$
  such that $x, y \in \mathscr{C}$. Then by the convexity of $C$, $tx
  + (1-t)y \in C
  \subset F_{\mathscr{C}}$ for all $t \in [0, 1]$.

  We claim that $F_{ \mathscr{C}}$ is a proper subset of $K$. For the
  sake of contradiction, assume $F_{ \mathscr{C}} = K$. Since $K$ is
  compact and $C$ is open in $K$ for all $C \in \mathscr{C}$, there
  are finitely many $C_1 \subset C_2 \subset \ldots \subset C_k \in
  \mathscr{C}$ which cover $K$ (i.e $K = \cup_{n = 1}^{k}C_n$). Hence
  we get $C_k = K$, which is absurd since $C_k$ must be a proper subset of $K$.
  Hence $F_{\mathscr{C}} \in \mathcal{A}$ and thus every chain must
  have an upper bound in $\mathcal{A}$. Now by Zorn's lemma,
  $\mathcal{A}$ has a maximal element $K_0$.

  Since $K$ is a connected space (path connected by a straight line,
  being convex), we know that the only
  clopen subsets are $\emptyset$ and $K$. Since we know that $K_0$ is
  \marginnote{\scriptsize Constructing and open convex subset containing $K_0$}
  open being in $\mathcal{A}$, we see that $K_0 \neq K$ and $K_0 \neq
  \emptyset$. Therefore $\overline{K_0} \neq K_0$. Let $x_0 \in
  \overline{K_0} \setminus K_0$, $y_o \in K_0$ and $0 < t <1$. Define
  $\varphi_{t, y_0}: K \to K$ such that $\varphi_{t, y_0}(z) = ty_0 +
  (1-t) z$. Then $\varphi_{t, y_0}$ is ($1-t$ Lipschitz) continuous
  relative to $K$ and thus $\varphi_{t, y_0}^{-1}(K_0)$ is open in
  $K$.
  By the convexity of $K_0$, we get $K_0 \subset \phi^{-1}_{t, y_0}(K_0)$.

  Also $\varphi_{t, y_0}^{-1}(K_0)$ is convex. Let $a, b \in
  \phi^{-1}_{t, y_0}(K)$. Then $ty_0 + (1-t)a, ty_0 + (1-t)b \in
  K_0$. By the convexity of $K_0$ we get $r(ty_0 + (1-t)a) +
  (1-r)(ty_0 + (1-t)b) = ty_0 + (1-t)(ra + (1-r)b) = \phi_{t, y_0}(ra
  + (1-r)b) \in K_0$ for all $r \in [0, 1]$. Thus $ra + (1-r)b \in
  \phi^{-1}_{t, y_0}(K_0)$ for all $r \in [0, 1]$. Hence we get
  $\phi^{-1}_{t, y_0}(K_0)$ is convex.

  We claim, $x_0 \in \varphi_{t, y_0}^{-1}(K_0)$, then the maximality
  of $K_0$ will force $\phi^{-1}_{t, y_0}(K_0) = K$.  Let $U$ be a
  convex neighborhood of $0 \in X$ containing $-x$ for all $x \in U$
  \marginnote{\scriptsize \textcolor{red}{I can't picturize the choice of $z$}}
  (just take $-U$ and intersect with $U$) such that $y_o + E
  \subset K_0$, where $E = K \cap U$. Let $w = \varphi_{t,
  y_0}(x_0)$. Since $x_0 \in
  \overline{K_0}$, for any $r>0$, there exists $x_r \in K_0$ such
  that $x_r \in (x_0 +
  rE) \cap K_0 \neq \emptyset$. In particular, let $r =
  \frac{t}{1-t}$. Then by linearity, we get $\big(x_0 +
  \frac{t}{1-t}E\big) \cap K_0 = ( \frac{t}{1-t}  )E\cap
  (K_0 - x_0) \neq \emptyset$. Choose $z$ in the above set. Then \[
    y_0 - \Big( \frac{1-t}{t} \Big)z \ \in \ y_0 + E \subset K_0
  \]
  and $x_0 + z \in K_0$. Since $K_0$ is convex, \[
    t\Big(y_0 - \frac{(1-t)}{t}z\Big) + (1-t)(x_0 + z)  = \phi_{t,
    y_0}(x_0) \in K_0
  \]
  Thus $\phi^{-1}_{t, y_0}(K_0) = K$.

  Now we claim that $K = K_0 \cup \{ x_0 \}$. For the sake of
  contradiction assume $\exists p \in K$ such that $p \notin K_0 \cup
  \{ x_0 \}$. Since the space is Hausdorff and locally convex, $x_0$ has an
  open convex neighborhood $E$ in $X$ such that $p \not\in E$. Let
  $E^\prime = E \cap K$, $a \in K_0, b \in E^\prime$ and $0 < r < 1$.
  Then since $\phi_{t, y_0}(K) = K_0$ for all $t \in [0, 1], y_0 \in K_0$, we
  get $\phi_{r, a}(b) = ra + (1-r)b \in K_0$. So $K_0 \cup
  E^\prime$ is convex (Sine we know that $K_0, E^\prime$ are convex,
    we only need to worry about $rx + (1-r)y$ for $x \in K_0, y \in
  E^\prime$. But $\phi_{r, x}$ takes care of that). $K_0 \cup
  E^\prime$ is also open in $K$. Hence by maximality, we get $K_0
  \cup E' = K$. But this is a contradiction since $ p \not\in K_0
  \cup E^\prime$. Thus by \autoref{lem:extreme_point_of_convex_set}, we see
  that $x_0 \in \textrm{Ext}(K)$.

  Next we prove $K = \overline{co}(\textrm{Ext}(K))$. Let $P =
  \overline{co}(\textrm{Ext}(K))$ and for the sake of contradiction
  assume $P \neq K$. Let $ x_0 \in K \setminus P$.
  %
  % Let $E$ be an open
  % convex neighborhood of $0 \in X$ such that $(x_0 + E^\prime) \cap P
  % = \emptyset$ for $E^\prime = E \cap K$. Existence of such an $E$
  % is guaranteed
  % since $P$ is compact and the space $X$ is Hausdorff.
  % \marginnote{\scriptsize \textcolor{red}{Don't know how to proceed after}}
  %
  Now by the geometric Hahn-Banach separation theorem, we get that
  there is a continuous linear functional
  $\phi: X \to \mathbb{R}$ and a number $\alpha, \epsilon \in
  \mathbb{R}$ such that \[
    \Re\phi(x_0) \le  \alpha < \alpha + \epsilon \le \Re\phi(p),
    \quad  \forall p \in P
  \]

  % Define $\phi: X \to
  % \mathbb{R}$ such that \[
  %   \phi(x) = \inf \{ 0 \le t \big | x \in tE \}
  % \]
  % Observe that $E = \{ x \in X \ : \phi(x) < 1 \}$. For every $r \ge
  % 0$ and $x \in X$, $\phi(rx) = r \phi(x)$, and for all $x, y \in X$,
  % $ \phi(x+y) \le \phi(x) + \phi(y)$. Define $f:  \mathbb{R}\{ x_0 \}
  % \to \mathbb{R}$, $f(rx_0) = r \phi(x_0)$, for all $r \in
  % \mathbb{R}$. For every $r \ge 0$, we have $f(rx_0) = r \phi(x_0) =
  % \phi(rx_0)$. For $r < 0$, we have $f(rx_0) = r \phi(x_0) =
  % -\phi(-rx_0) \le 0 \le \phi(rx_0)$. So $f \le \phi$ on $\mathbb{R}
  % x_0$. Then by Hahn-Banach separation theorem, there is an
  % extension $ \tilde{f}: X \to
  % \mathbb{R}$ such that $\tilde{f}(x) \le \phi(x) $ for all $x \in X$.

\end{proof}



% TeX_root = ../main.tex

\chapter{}

\begin{lemma}
  Let $K_1, K_2$ be compact convex subsets of a locally compact TVS $X$. Then \[
    \overline{\textrm{co}}(K_1 \cup K_2) = (\textrm{co})( K_1 \cup K_2)
  \]
  \label{lem:convex_hull_of_the union_of_compact_convex_sets_are_compact}
\end{lemma}
\begin{proof}
  \textcolor{red}{verify}.
  We'll show that $\textrm{co}(K_1 \cup K_2)$ is compact and hence
  closed. Let $x = \alpha_1a_1 , \alpha_2a_2 , \ldots , \alpha_na_n +
  \beta_1b_1 , \beta_2b_2 , \ldots , \beta_mb_n \in \textrm{ co}(K_1
  \cup K_2)$, where $\sum_{i = 1}^{n} \alpha_i + \sum_{i = 1}^{m}
  \beta_i = 1$. Then \[
    x = \big(\sum_{i = 1}^{n} \alpha_i\big) \underbrace{\Bigg(
        \sum_{i = 1}^{n} \Big( \frac{\alpha_i}{\sum_{i = 1}^{n}
    \alpha_i}\Big) a_i\Bigg)}_{\in \ K_1}+ \big(\sum_{i = 1}^{m}
    \beta_i \big) \underbrace{\Bigg( \sum_{i = 1}^{m} \Big(
    \frac{\beta_i}{\sum_{i = 1}^{m} \beta_i}\Big) b_i\Bigg)}_{\in \ K_2}
  \]
  Hence every element $x \in \textrm{co}(K_1 \cup K_2)$, can be
  written as $x = ta + (1-t)b$ where $ a \in K_1, b \in K_2$.

  Now let $x_\lambda = t_\lambda a_\lambda + (1-t_\lambda)b_\lambda$
  be a net in $\textrm{co}(K_1 \cup K_2)$, for $  \lambda \in
  \Lambda$, $a_\lambda \in K_1, b_\lambda \in K_2$. Since
  $(a_\lambda)$ is a net in the compact set $K_1$, there is a subnet
  $a_\sigma$ for $\sigma \in \Sigma \subseteq \Lambda$, such that
  $a_\sigma \to a \in K_1$. By similar reasoning $b_\sigma$ has a
  convergent subnet $b_\pi$ for $\pi \in \Pi \subseteq \Sigma$, such
  that $b_\pi \to b \in K_2$. Again $t_\pi$ is a net in the compact
  space $[0, 1]$, hence is has a convergent subnet $t_\omega$ for $
  \omega \in \Omega \subseteq \Pi$ such that $t_\omega \to t$ in $[0, 1]$.

  Now consider the subnet $x_\omega = t_\omega a_\omega +
  (1-t_\omega)\beta_\omega$ of $x_\lambda$. Since $ \Omega \subseteq
  \Pi \subseteq \Sigma$, $t_\omega \to t, \beta_\omega \to b$ and
  $a_\omega \to a$. Therefore by the continuity of the scalar product
  and addition in the TVS, we get $x_\omega \to t \alpha + (1-t)
  \beta \in \textrm{co}(K_1 \cup K_2)$. Hence we get $\textrm{co}(K_1
  \cup K_2)$ is compact.
\end{proof}

\begin{theorem}[Inverse Krein-Milman]
  Let $K$ be a compact convex subset of a locally convex topological
  vector space $X$. Let $A \subset K$ be a closed subset of $K$. If
  $K = \overline{\textrm{co}}(A)$, then $\textrm{Ext}(K) \subset A$.
\end{theorem}

Note that $\textrm{Prob}[0, 1]$, the collection of probability
measures identified as a subspace of a $C([0, 1])^*$ is convex, weak
* compact with $\textrm{Ext}(K) = \{ \delta_x  \ : \  x \in [0, 1] \}$
\begin{proof}
  FSTOC, assume $\exists x_0 \in \textrm{Ext}(K)$, $x_0 \not \in A$.
  Since $A$ is compact, $\exists y_1 , y_2 , \ldots , y_n \in A$ and
  an open convex neighborhood $B$ of $0$ such that $$A \subset
  \cup_{i = 1}^{n}(y_i + B)$$
  and $x_0 \not\in y_i + \overline{B}$ for all $i = 1, 2, \ldots, n$.
  Let $B_i = (y_i + \overline{B}) \cap K$. Then $B_i$ is a compact
  convex subset of $K$ for each $i$. Hence by the previous lemma, we get \[
    \textrm{co}(B_1 \cup B_2 \cup \ldots \cup B_n) =
    \overline{\textrm{co}}(B_1 \cup B_2 \cup \ldots \cup B_n) \supset
    \overline{\textrm{co}}(A) = K
  \]
  Thus $\exists b_i \in B_i$ and $0 \le t_i \le 1$, $\sum_{i = 1}^{n}
  t_i = 1$ such that
  \marginnote{\scriptsize \textcolor{ red}{I struggle at finding the
  contradiction}}
  \[
    x_0 = t_n b_1 + t_n b_2 + \ldots + t_n b_n
  \]
  Since $x_0 \in \textrm{Ext}(K)$, this forces $x_0 = b_j$ for some
  $1 \le j \le n$.
  This contradicts the assumption that $x_0 \not\in y_i +
  \overline{B}$. Hence $x_0 \in A$.
\end{proof}

Note that in the following attempt to prove the theorem, it is not
obvious why $U$ is convex. If we try to argue using arguments to the
proof of separating a compact set and a point using open sets in a
Hausdorff space, we will eventually need to show that the finite
open subcover of $A$ sits inside a closed convex set that does not
contain $x_0$, which again is not obvious.
\begin{proof}
  FTSOC, assume that $\exists x_0 \in \textrm{Ext}(K)\setminus A$.
  \marginnote{ \scriptsize \textcolor{red}{Why is $U$ convex?}}
  Since the TVS is Hausdorff, there exist convex open sets $U,
  V$ such that $A \subset U, x_0 \in V$, and $U \cap V = \emptyset$.
  Moreover we claim that $\overline{U} \cap V = \emptyset$. Otherwise if $x \in
  \overline{V} \cap U$, then for any net $(x_\lambda) \in V$ that
  converge to $x$, by the definition of convergence $x_{\lambda_n}
  \in U$ for all $\lambda_n$ greater that some $\lambda_N$. This
  would contradict the assumption that $U\cap V = \emptyset$. Hence
  we see that $A \subset \overline{V}$, and therefore
  $\overline{\textrm{co}}(A) \subset \overline{V}$. But this would
  again contradict the fact that $\overline{\textrm{co}}(A) = K$
  since $x_0 \notin \overline{V}$.
\end{proof}



% TeX_root = ../main.tex

% \part

\chapter{}

\section{Recap on topology}
\begin{definition}
  Let $(X, \tau)$ be a topological space. A set $E$ is called closed if its complement is open. The closure of $E$ is the smallest closed subset containing $E$. \[
    \overline{E} = \bigcap_{\substack{F^c \in \tau \\ E \subset F}} F
  \]
  We can check $\overline{E}$ is closed by looking at $\overline{E}^c$.
\end{definition}

\begin{definition}
  A set $K \subset X$ is called compact if every open cover of $K$ has a finite subcover.
\end{definition}

\begin{definition}
  $(X, \tau)$ is Hausdorff ($T_2$) if for any $p \neq q \in X$ there are open sets $U, V \in \tau$ such that $ p \in U, q \in V$ and $ U \cap V  = \emptyset$. 
\end{definition}

\begin{definition}
   A neighborhood of $p \in X$ is an open set $U \in \tau$ containing $p$.
\end{definition}

\begin{definition}
  $X$ is called locally compact if any point $p \in X$ has a neighborhood $V$ with compact $\overline{V}$.
\end{definition}

\begin{theorem}
   Let $X$ be a topological space. If $K \subset X$ is compact and $F\subset K$ is closed, then $F$ is compact.
\end{theorem}
\begin{proof}
  Make any covering of $F$ into a covering of $K$, by adding $F^c$, the get a finite subcover for $K$, then remove $F^c$ from this subcover if its there. Now you got a finite subcover for $F$.
\end{proof}

\begin{theorem}
  let $X$ be a topological Hausdorff space. Then if $K \subset X$ is compact, $p \notin K$, then there are open set $U, V$ such that $K \subset V$, $p \in U$, $U \cap V = \emptyset$. (not that we are not claiming regularity).
\end{theorem}
\begin{proof}
  For each $q \in K$, there is an open set $U_q, V_q$ with $q \in V_q, p \in V_q, V_q \cap U_q = \emptyset$. Then $K \subset \cap_{q \in K}V_q$. Then since $ K$ is compact, there is a finite subcover $V_{q_1} , V_{q_2} , \ldots V_{q_n}$ of $K$. Now let $V = \cup_{i = 1}^{n}V_{q_i}$ and $U = \cap_{i = 1}^{n}U_{q_i}$ both of which are open. Then $K \subset V, p \in U$ and $U \cap V = \emptyset$.
\end{proof}

\begin{theorem}
   If $K_\alpha$ is a collection of nonempty compact subsets of a topological Hausdorff space $X$ indexed by $ A$, and if for each finite subset $B \subset A$, $\cap_{\beta \in  B}K_{\beta} \neq \emptyset$ then  \[
      \cap_{\alpha \in  A}K_\alpha \neq \emptyset
   \]
\end{theorem}
\begin{proof}
  If $\cap_{\alpha \in  A}K_\alpha = \emptyset$, then $K_\alpha^c$ forms an open cover for $K_{\alpha_0}$. Now use the compactness property. \textcolor{red}{verify}
\end{proof}

\begin{theorem}
  If $X, Y$ are topological spaces, if $f: X \to Y$ is continuous, and $K$ is compact, then $f(K)$ is compact.
\end{theorem}
\begin{proof}
  Let $U_\alpha$ be an open cover for $f(K)$, then $f^{-1}(U_\alpha)$ forms an open cover for $K$. Now by the compactness there is a finite cover $f^{-1}(U_{\alpha_1}), f^{-1}(U_{\alpha_2}), \ldots , f^{-1}(U_{\alpha_n})$. Therefore $U_{\alpha_1} , U_{\alpha_2} , \ldots , U_{\alpha_n}$ is a finite subcover of $f(K)$.
\end{proof}

\begin{definition}
  Let $X$ be a topological space, $f: X \to \mathbb{C}$. Then the support of $f$ is defined as $\textrm{  supp} f = \overline{\{ x \in X  \ : \  f(x) \neq 0 \}}$. 
  See that $\textrm{supp}(f+g) \subset \textrm{supp}(f) \cup \textrm{supp}(g)$
\end{definition}

  We denote $C_c(X)$ to be the set of continuous functions which have compact support. $C_c(X)$ is a subspace of the vector space $C(X)$.

\begin{theorem}[Urysohn Lemma]
  Let $X$ be a locally compact Hausdorff space. If $X$ is compact, $V$ is open and $K \subset V$, then there is a function $f \in C_c(X)$ with $$\chi_K \le f \le \chi_V$$.
\end{theorem}
















% TeX_root = ../main.tex

\begin{example}
  Let $S = \{ n \delta_n  \ : \  n \in \mathbb{N} \} \subset
  \ell^\infty$. We show that $0 \in \overline{S}^{w^*}$
\end{example}
\begin{proof}
  Let $f \in \ell^1$. Then the set $\{ n \in \mathbb{N}  \ : \
  |f(n)| < \epsilon/n \}$ is infinite. (Otherwise this would
  contradict $f \in \ell^1$). Thus $\exists N \in \mathbb{N}$ such
  that $N|f_i(N)| < \epsilon$  for all $i = 1, 2, \ldots m$. And
  therefore $$N \delta_n \in \bigcup_{f_1 , f_2 , \ldots , f_N, \epsilon} (0)$$
\end{proof}

\begin{definition}
  A subset $S$ of a vector space $V$ is called  balanced if $\forall
  s \in S, \alpha \in \mathbb{F}$ with $|\alpha| \le 1, \alpha s \in S$.
\end{definition}

\begin{lemma}
  Let $X$ be a topological vector space, then every open neighborhood
  of ${0}$ contains a balanced open neighborhood of $O$.
\end{lemma}
\begin{proof}
  \textcolor{red}{verify}
\end{proof}

\begin{lemma}
  All n-dimensional topological vector spaces are isomorphic as
  topological vector spaces.
\end{lemma}
\begin{proof}
  For the case $n = 1$, and $\mathbb{F} = \mathbb{C}$.

  Assume $\tau$ is a topology on $\mathbb{C}$ that turns it into a
  topological vector space. Now think of $i : \mathbb{C} \to
  (\mathbb{C}, \tau) := x \to x$  as the composition of $\mathbb{C}
  \to \mathbb{C} \times (\mathbb{C} , \tau):= x \to (x, 1)$ and
  $\mathbb{C} \times (\mathbb{C} , \tau) \to (\mathbb{C} , \tau):=
  (x, y) \to xy$. Then we see  that $i$ is the composition of these
  maps which are continuous by the definition of the product topology
  and the TVS. Hence, $i$ is continuous.

  To show that $i^{-1}$ is continuous, consider the annulus $A = \{
  \alpha \in \mathbb{C}  \ : \  1 \le |\alpha| \le 2 \}$. Then since
  $A$ is compact in the usual topology and $i(A) = A$ is a continuous
  image, we get that $A$ is open in $\tau$. Hence $A^c \ni 0$ is open
  and by the lemma above has a balanced open neighborhood of $0$ in
  it. \textcolor{red}{(Show that this is actually an open disk)}.
\end{proof}

\begin{theorem}
  Let $X$ be a normed space. Then the closed unit ball of $X$ is
  compact in norm topology if and only if $X$ is finite dimensional.
\end{theorem}
\begin{proof}
  Suppose $X$ is infinite dimensional normed space and let $\bar{B}$
  be the closed unit ball. Let $x_1 \in \bar{B}$ and let $Y_1 =
  \textrm{span}\{ x_1 \}$. Then $Y_1$ is a closed subspace of $X$.
  Since $X$ is a non-zero normed space, let $x_2 \in X$ such that
  $\|x_2 + Y_1\| = \frac{1}{2}$.
  Repeat the construction in the proof of Reisz lemma.
\end{proof}

\begin{lemma}
  Let $X$ be a normed space. Then $i(X)$ is weak * dense in $X^{**}$
\end{lemma}
\begin{proof}
  Let $C = \overline{B}^{w*}$, where $B$ is the closed unit ball.
  Then $C$ is compact convex.
  FSTOC, assume $\exists \phi \in X^{**}$ such that $\|\phi\| \le 1$,
  \marginnote{ \scriptsize Show that $\Re f(x) \le r \|x\|$ imply $
  \|f\| \le \alpha$}
  $\phi \notin C$. Then by HBT, there is a $f \in X^{**}$ and $r \in
  \mathbb{R}$, $\epsilon >0$ such that $\Re f(y) \le r < r+ \epsilon
  \le \Re f(\phi)$ for all $y \in i(X)$. This implies $\|f\| \le r$,
  hence $|f(\phi)| = |\phi(f)| \le \|\phi\|\|f\| < r$ which gives a
  contradiction.
\end{proof}

\begin{theorem}
  Let $X$ be a Banach space. Then the closed unit ball $\bar{B}$ is
  weakly compact if and only if $X$ is reflexive.
\end{theorem}
\begin{proof}
  If $X$ is reflexive, the weak and weak * topology coincides and the
  Banach Alaouglu  gives the proof. \textcolor{red}{verify}

  Assume $\bar{B}$ is weakly compact. Observe that then the map $i: X
  \to X^{**}$ is continuous when we equip $X$ with weak topology and
  $X^{**}$ with weak topology. Thus $i(\bar{B})$ is weak * compact.
  Moreover $ i(\bar{B})$ is weak * dense in the closed unit ball of
  $X^{**}$. Hence the result.
\end{proof}


% TeX_root = ../main.tex

\chapter{}

\begin{theorem}
  Let $H$ be a Hilbert space and let $C$ be a non-empty closed convex
  subset of $H$. Then there exist a unique vector $x \in C$ such that
  $\|x\| \le \|\eta\|$ for all $\eta \in C$.
\end{theorem}
\begin{proof}
  Let $d = \inf \{ \|\eta\|  \ : \ \eta \in C  \}$ and choose a
  sequence $ \eta_n \in C$  such that $\|\eta_n\| \to d$. Let $
  \epsilon > 0$. Choose $N \in \mathbb{N}$ such that $\|\eta_n\|^2 <
  d^2 + \epsilon$ for all $n \ge N$. Then for all $m,n \ge N$, we have \[
    \|\eta_n - \eta_m\|^2 = 2(\|\eta_n\|^2 + \|\eta_m\|^2) - 4 \|
    \frac{1}{2}(\eta_n + \eta_m) \|^2 \le 4(d^2  + \epsilon) - 4d^2 = 4 \epsilon
  \]
  Hence the sequence $\eta_n$ is Cauchy and hence convergent since
  the space is complete. Let $\eta = \lim_{n \to \infty} \eta_n$.
  Since $C$ is closed $\eta \in C$ and clearly $\|\eta\| = d$.

  To see the uniqueness, assume $\alpha \in C$, and $\| \alpha\| = d$. Then
  \begin{align*}
    \|\eta - \alpha\|^2 &= 2(\|\eta\|^2 + \|\alpha\|^2) - 4 \|
    \frac{1}{2}(\eta + \alpha) \|^2 \\
    & \le 4d^2 - 4d^2 = 0
  \end{align*}
  Verify the second inequality.
\end{proof}

\begin{corollary}
  Let $\eta \in H$ and $C$ be as before. Then there exist a unique
  vector $x \in C$ such that $d(\eta, C) = \|x - \eta\|$
\end{corollary}
\begin{proof}
  \textcolor{red}{Apply above theorem to $C^\prime = C - \eta$.}
\end{proof}

\begin{proposition}[Pythagoras Theorem]
  Let $ x, y \in H$ an inner product space, and $x \perp y$, then
  $\|x + y\|^2 = \|x\|^2 + \|y\|^2$.
\end{proposition}

\begin{lemma}
  \label{lem:15}
  Let $H$ be a Hilbert space and $K$ be a nontrivial closed subspace.
  Let $\eta \in H$. Then $\xi \in K$ satisfy $\|\xi - \eta\| = d(\eta, K)$ iff
  $\xi - \eta \perp K$.
\end{lemma}

\begin{theorem}[Reisz Representation Theorem]
  Let $H$ be a Hilbert space and $f \in H^*$. Then there exists a
  unique $\eta_f \in H$ such that $f(x) = \langle x , \eta_f \rangle $
  for all $ x \in H$. The map $\phi: H^* \to H := f \to \eta_f$ is
  conjugate linear isometric bijection.
\end{theorem}
\begin{proof}
  \textcolor{red}{verify}
\end{proof}

\begin{theorem}
  Let $H_1, H_2$ be Hilbert spaces, and $T: H_1 \to H_2$ be a bounded
  linear map. Then there exists a unique bounded linear map $T^*: H_2
  \to H_1$ satisfying $\langle  Tx , y \rangle_{H_2} = \langle x ,
  T^*y \rangle_{H_1}$ for all $x \in H_1, y \in H_2$.
\end{theorem}
\begin{proof}
  For every given $y \in H_2$ define a linear functional $f^y: H_1
  \to \mathbb{C}$ as $f^y(x) = \langle Tx , y \rangle$. Since $f^{y}$
  is bounded, $f^y \in H_1^*$. Hence by Reisz representation, there
  is a unique $T^*(y) \in H_1$ such that $\langle  Tx , y \rangle =
  \langle x , T^*y \rangle$.

  Uniquness follows from the fact that in any inner product space
  $X$, if $x, y \in X$ such that $ \langle x , z \rangle = \langle  y
  , z \rangle $ for all $z \in X$, then $x = y$
  \textcolor{red}{Verify the linearity}.
\end{proof}

\begin{theorem}
  Let $H$ be a Hilbert space and $K$ a closed subspace. For every
  $\eta \in H$, denote by $P_K(\eta)$, the unique closest vector in
  $K$, closest to $\eta$. Then
  \begin{enumerate}[label=(\arabic*)]
    \item $P_K: H \to H$ is linear, bounded with $\|P_K\| = 1$ and idempotent.
    \item $P_K^* = P_K$ (self-adjoint)
  \end{enumerate}
\end{theorem}
\begin{proof}
  \begin{enumerate}[label=(\arabic*)]
    \item Let $\eta_1, \eta_2 \in H$ and $\alpha \in \mathbb{C}$.
      Then for all $\xi \in K$, we have
      \begin{align*}
        \langle  \alpha \eta_1
        +\eta_2 - \alpha P_K( \eta_1) - P_K(\eta_2) , \xi \rangle &=
        \alpha \langle \eta_1 - P_K(\eta_1) , \xi  \rangle + \langle
        \eta_2 - P_K(\eta_2) , \xi  \rangle  \\
        &= 0
      \end{align*}

      If $K \neq \{ 0\}$ and $0 \neq $
    \item
  \end{enumerate}
\end{proof}


% TeX_root = ../main.tex

\chapter{}

\section{Vitali Sets}

\begin{theorem}
  If $\mathcal{M}$ is a $\sigma$-algebra on $\mathbb{R}$ and
  $\lambda: \mathcal{M} \to [0, \infty]$ is a translation invariant
  measure with $0 < \lambda([0, 1)) < \infty$, then there is $ E
  \subset [0, 1)$ such that $E \notin \mathcal{M}$.
\end{theorem}
\begin{proof}
  Endow $[0, 1)$ with an equivalence relation $a \sim b \iff a-b \in
  \mathbb{Q}$. This gives a partition of $[0, 1)$ by the equivalence classes.
  Now from each of these classes pick (by AOC) one representative element and
  build the set $E$. Observe that for $r, s \in \mathbb{Q}$, $(E + s)
  \cap (E + r) = \emptyset$ if and only if $r = s$.

  Also note that  \[
    [0, 1) \ \subset \ \cup_{r \in \mathbb{Q} \cap [-1, 1]}(E+r)
  \]
  Therefore \[
    E \ \subset [0, 1) \ \subset  \ \cup_{r \in \mathbb{Q} \cap [-1, 1]}(E+r)
    \ \subset \ [-1, 2)
  \]
  \textcolor{red}{verify the rest, its easy}.
\end{proof}

\begin{theorem}[Luzin's theorem]
  Let $X$ be a locally compact Hausdorff space.
  \begin{enumerate}[label=(\arabic*)]
    \item $\mu$ is a regular measure on a $\sigma$-algebra
      $\mathcal{M}$ containing $B(X)$
    \item $f: X \to \mathbb{C}$ is measurable
    \item there is a $A \in \mathcal{M}$ such that $\mu(A) < \infty$
      and $f = 0$ on $A^c$
  \end{enumerate}
  Given $\epsilon> 0$ there is a $g \in C_c(X)$ such that $
  \mu(\{ x \in X  \ : \  f(x) \neq g(x) \}) < \epsilon$
\end{theorem}



% TeX_root = ../main.tex

\chapter{}

\begin{theorem}[Luzin's theorem]
  Let $X$ be a locally compact Hausdorff space.
  \begin{enumerate}[label=(\arabic*)]
    \item $\mu$ is a regular measure on a $\sigma$-algebra
      $\mathcal{M}$ containing $B(X)$
    \item $f: X \to \mathbb{C}$ is measurable
    \item there is a $A \in \mathcal{M}$ such that $\mu(A) < \infty$
      and $f = 0$ on $A^c$
  \end{enumerate}
  Given $\epsilon> 0$ there is a $g \in C_c(X)$ such that $
  \mu(\{ x \in X  \ : \  f(x) \neq g(x) \}) < \epsilon$ and $\sup \{
  |g(x)|  \ : \ x \in X \} \le \sup \{ |f(x)| \ : \  x \in X \}$.
\end{theorem}
\begin{proof}
  Suppose for now $A$ is compact. (We can assume this since the
    measure is regular and we can find a compact set $K \subset A$ such
  that $f = 0$ almost everywhere in $K^c$.) We'll do the $A$ not
  compact case later.

  Choose $V$ open such that $ A \subset V$ and $\overline{V}$ is
  compact. We'll first prove the existence of the desired $g$ if $f$
  is simple. Let \[
    f = \sum_{j = 1}^{n} \alpha_j \chi_{A_j}
  \]
  where each $A_j$ is disjoint and $\cup_{j = 1}^{n}A_j = A$. Again
  each of the $\mu(A_j) \le \mu(A) < \infty$. Hence by the regularity
  of the measure there are compact sets $K_j \subset A_j$ such that $
  \mu(A_j \setminus K_j) < \frac{\epsilon}{2^{j+1}}$.

  Since $K_j$ are compact and disjoint, we can find collection of
  disjoint open sets $V_j$ such that $K_j \subset V_j$.
  \textcolor{red}{verify this, I am not sure}.

  Moreover by replacing $V_j$ with $V_j \cap V$, we can assume $V_j \subset V$.
  Now by the outer regularity of the measure, we can assume $\mu(V_j
  \setminus K_j) < \frac{\epsilon}{2^{j+1}}$. Now by Urysohn, there is a
  $g_i \in C_c(X)$ such that $\chi_{K_j} \le g_j \le \chi_{V_j}$. Let   \[
    g = \sum_{j = 1}^{n} \alpha_j g_j
  \]
  Then $g$ is continuous being the finite sum of continuous function.
  Moreover since $\cup_{j = 1}^{n}V_j \subset V$, we get
  $\textrm{supp}(g) \subset \overline{V}$. Also
  \begin{align*}
    |g(x)| & \le \max \{ |\alpha_j| \} \\
    & \max_{x \in A} | f(x)|
  \end{align*}

  Now we see that $f(x) = g(x)$ for all $x \in K_j$ and $x \in
  (A_j \cup V_j)^c$. Since $K_j \subset V_j$, the set where they possibly
  \marginnote{\scriptsize Add a diagram for ease of reasoning}
  disagree is $$D = \bigcup_{j = 1}^{n}(V_j \setminus K_j) \quad \cup
  \quad \bigcup_{j = 1}^{n}(A_j \setminus K_j)$$.
  Now by the subadditivity of $\mu$, we get $\mu(D) < \epsilon$.

  Now for the general case, assume $0\le f < 1$ and $s_n$ be the
  sequence of simple functions $0 \le s_1 \le s_2 \le \ldots \le $
  with $\lim_{n \to \infty} s_n(x) = f(x)$. Let $t_n = s_n -
  s_{n-1}$, where $s_0 = 0$. Each $t_n$ is simple and $t_n = 0$ on
  $A^c$ and by construction, we get \[
    t_n \le \frac{1}{2^{n-1}} \chi_{B_n}
  \]
  for some  set $B_n$.

  Now we use the first part of the proof on $t_n$s to get a
  corresponding $g_n \in C_c(X)$. Then $g_n$ satisfy
  \begin{enumerate}[label=(\arabic*)]
    \item
    \item
    \item
  \end{enumerate}
  Let $g = \sum_{n \in \mathbb{N}} g_n$, which converges uniformly as
  $|g_n| \le \frac{1}{2^{n-1}}$ by Wierestrass. Hence $g \in C_c(X)$
  and $  \textrm{supp}(g) \subset \overline{V}$.

  We know that $f = \sum_{n = 1}^{\infty}  t_n$  from
  the definition of $t_n$. So the set $D = \{ x \in X  \ : \  f(x)
  \neq g(x) \}$ is a subset of $\cup_{n = 1}^{\infty} \{ x \in X  \ :
  \ t_n(x) \neq g_n(X)  \}$. Now the subadditivity of $\mu$ gives
  that $\mu(D) < \epsilon$.

  Next, if $f$ is real valued, bounded, the result follows from
  scaling $f$. Again if $f \ge 0$ is measurable, then we have
  $\cap_{n = 1}^{\infty} \{ x \in X  \ : \  f(x) \ge n \} =
  \emptyset$. Moreover $\mu(\{ f \ge 1 \}) \le \mu(A) < \infty$.
  Hence by the continuity of the measure from above, we get $\mu(\{ f
  \ge n \}) \to 0$. Hence we can replace $f$ with $f \chi_{f< n}$ for
  some appropriate $n$.
\end{proof}



% TeX_root = ../main.tex

\chapter{}
\begin{lemma}
  If $T \in B(H)$ is compact, then $T(H)$ is separable.
\end{lemma}
\begin{proof}
  \textcolor{red}{verify}
\end{proof}

\begin{corollary}
  The set $K(H)$ of all compact operators on $  H$ is a closed
  two-sided ideal in $B(H)$.
\end{corollary}
\begin{proof}
  \textcolor{red}{verify}. Use the fact that compact operators are
  the closure of finite rank operators.
\end{proof}

\begin{corollary}
  $T \in K(H)$ implies $T^* \in K(H)$.
\end{corollary}
\begin{proof}
  \textcolor{red}{verify}. Use the fact that $T$ is finite rank
  implies $T^*$ is finite rank.
\end{proof}

\begin{example}
  Let \[
    T: L^2([0, 1], m) \to L^2([0, 1], m)
  \]
  be defined as $T(f)(x) = xf(x)$. Prove that $T = T^*$ and that $T$
  has no eigenvectors.

  Only for joseph: If $\xi = \eta$ almost everywhere, then show that  $f \xi = f
  \eta$ almost everywhere if $ f \in C([0, 1])$
\end{example}
\begin{proof}
  \textcolor{red}{Homework}
\end{proof}

\begin{example}
  Let $\alpha_n$ be a bounded sequence in $\mathbb{C}$. Consider $T :
  \ell^2 \to \ell^2$, such that \[
    T(\delta_n) = \alpha_n \delta_n
  \]
  for all $n \in \mathbb{N}$. Then $T$ is bounded with $\|T\| =
  \|(\alpha_n)\|_\infty$.
\end{example}

\begin{example}
  Prove that the $T$ above is compact if and only if $(\alpha_n) \in c_{\bf 0}$
  Prove that $T^*(\delta_n) = \overline{\alpha_n}\delta_n$
\end{example}
\begin{proof}
  \textcolor{red}{Homework}
\end{proof}

\begin{theorem}
  Let $H$ be separable Hilbert space and $T\in K(H)$ be normal. i.e
  $TT^* = T^*T$. Then there exist an orthonormal basis $\{ e_n  \ :
  \  n \in \mathbb{N} \}$ of $H$, and a sequence $(\alpha_n) \in
  c_{\bf 0}$, such that $T(e_n) = \alpha_n e_n$.
\end{theorem}

\begin{lemma}
  Let $T \in K(H)$ be self-adjoint. Then $\exists 0 \neq \eta \in
  \mathbb{C}$ such that $ \textrm{Ker}(T - \lambda I) \neq \{ 0 \}$.
\end{lemma}

\begin{lemma}
  If $T$ is a compact operator, and $0 \neq \lambda \in \mathbb{C}$,
  Then $\textrm{Ker}(T - \lambda I)$ if finite dimensional.
\end{lemma}

\begin{definition}
  Let $T \in B(H)$. A subspace $W \leqslant H$ is said to be
  invariant under $T$ if $T(W) \subset W$. We say $W$ reduces $T$, if
  $T(W) \subset W$ and $T(W^\perp) \subset W^\perp$
\end{definition}

\begin{example}
  Let $T \in B(H)$.
  Prove that $W \subset H$ is invariant under $T$ if and only if $P_WT = TP_W$
  Prove that $W$ reduces $T$ if and only if $P_WT = TP_W$ and
  $P_{W^\perp}T = T P_{W^\perp}$.
\end{example}
\begin{proof}
  \textcolor{red}{Homework}
\end{proof}



% TeX_root = ../main.tex

\chapter{$L^{p}$ Spaces}

\begin{definition}
  A function $\phi: (a,  b) \to \mathbb{R}$ is called convex if
  $$\phi(tx + (1-t)y) \le t \phi(x) + (1-t) \phi(y)$$
  for all $x, y \in
  (a, b)$ and $0 \le t \le 1$.
\end{definition}

\begin{proposition}
  A function $\phi: (a, b) \to \mathbb{R}$ is convex if and only if
  for $u, s, t$ with $ a < u \le t \le s < b$, we have \[
    \phi(t) \le \phi(s) \frac{u-t}{u-s} + \phi(u) \frac{t-s}{u-s}
  \]
  or equivalently using \[
    \phi(t) - \phi(s) = \frac{t-s}{u-s} (\phi(u) - \phi(s))
  \]
  satisfies \[
    \frac{\phi(t)-\phi(s)}{t-s} \le \frac{\phi(u)-\phi(s)}{u-s}
  \]
\end{proposition}

\begin{theorem}
  A function $\phi: (a, b) \to \mathbb{R}$ that is convex is continuous.
\end{theorem}
\begin{proof}
  Let $S = (s, \phi(s)), X = (  x, \phi(x)), Y = (y, \phi(y))$, with
  $a < s \le x \le y < b$.

  Draw secands and refer Rudin.
\end{proof}

\begin{theorem}[Jensen's Inequality]
  Let $(X, \mathcal{M}, \mu)$ be a measure space with $\mu(X) = 1$.
  If $f \in L^1(\mu)$ and for each $x \in X$, $a < f(x)< b$ and $
  \phi$ is convex on $(a, b)$, then \[
    \phi \Bigg(\int  f \ d \mu \Bigg) \le \int (\phi \circ f)\ d \mu
  \]
\end{theorem}
\begin{proof}
  We know by convexity that for $u \le s \le t$, \[
    \frac{\phi(t)-\phi(s)}{t-s} \le \frac{\phi(u)-\phi(s)}{u-s}
  \]
  Then there is $\beta$ such that \[
    \frac{\phi(t)-\phi(s)}{t-s} \le \beta \le \frac{\phi(u)-\phi(s)}{u-s}
  \]
  Consider LHS Inequality to get
  \begin{align*}
    \phi(t) - \phi(s) & \le \beta (t-s) \\
    \phi(s) & \ge \phi(t) + \beta(s-t)
  \end{align*}
  for $s < t$, and similarly by the RHS we get\[
    \phi(u)-\phi(s) \ge \beta(u-s)
  \]
  Hence in both the cases $(t = f(x), \ u = f(x))$ \[
    \phi(f(x)) - \phi(s) - \beta(f(x) - s) \ge 0
  \]
  Now integrating this gives \[
    \int \phi \circ f\ d \mu - \phi(t) - \beta \Big( \int  f \ d \mu -
    s \Big) \ge 0
  \]
  Choosing $s = \int  f \ d \mu$ gives out inequality.
\end{proof}

\begin{example}
  Take $\mu$ to be the probablity measure on $X = \{ 1, 2, 3, \ldots
  n \}$, assume $\mu(\{ j \})= \alpha_j> 0$. Then for $b_1 , b_2 ,
  \ldots , b_n > 0$, we have \[
    b_1^{\alpha_1}b_2^{\alpha_2}\ldots b_n^{\alpha_n} \le \sum_{j =
    1}^{n} \alpha_jb_j
  \]
\end{example}
\begin{proof}
  Use the convexity of $x \to e^x$, and let $ b_j = e^{c_j}$.
\end{proof}

\begin{theorem}[Holder's Inequality]
  Let $(X, \mathcal{M},  \mu)$ be a measure space, $f, g : X \to [0,
  \infty]$ be measurable. Then for $1 < p < \infty$, with $1/p + 1/q
  = 1$, then \[
    \int  fg \ d \mu \le \Bigg(\int  f^p \ d \mu \Bigg)^{\frac{1}{p}}
    \Bigg(\int  g^q \ d \mu\Bigg)^{\frac{1}{q}} \equiv \| f\|_p \|g\|_q
  \]
  and \[
    \Bigg(\int(f+g)^p \ d \mu\Bigg)^{\frac{1}{p}} \le \|f\|_p + \|g\|_p
  \]
\end{theorem}



% TeX_root = ../main.tex

\chapter{}
\begin{theorem}[Holder's \& Minkowski Inequality]
  Let $(X, \mathcal{M},  \mu)$ be a measure space, $f, g : X \to [0,
  \infty]$ be measurable. Then for $1 \le p < \infty$, with $1/p + 1/q
  = 1$, then \[
    \int  fg \ d \mu \le \Bigg(\int  f^p \ d \mu \Bigg)^{\frac{1}{p}}
    \Bigg(\int  g^q \ d \mu\Bigg)^{\frac{1}{q}} = \| f\|_p \|g\|_q
  \]
  and \[
    \Bigg(\int(f+g)^p \ d \mu\Bigg)^{\frac{1}{p}} \le \|f\|_p + \|g\|_p
  \]
\end{theorem}
\begin{proof}
  Let $A = \|f\|_p, B=\|g\|_p$. If $A = 0$ or $A = \infty$, or $ B =
  0$, or $ B = \infty$, we have nothing to show. Hence assume that $0
  < A, B < \infty$. Let $F(x) = \frac{f(x)}{A}, G(x) =
  \frac{g(x)}{B}$. We also define $ s, t : X \to \mathbb{R}$ as \[
    F(x) = e^{\frac{s(x)}{p}}, \quad G(x) = e^{\frac{t(x)}{q}}
  \]
  By convexity of the exponential function, we have $$e^{s/p + t/q}
  \le \frac{1}{p}e^s + \frac{1}{q}e^t$$
  In terms of $F, G$, this is \[
    F(x) G(x) \le \frac{1}{p}(F(x))^p + \frac{1}{q}(G(x))^p
  \]
  Hence integrating both sides, we get \[
    \int F(x) G(x) \ d \mu \  \le \ \frac{1}{p} \int (F(x))^p \ d \mu
    + \frac{1}{q} \int (G(x))^p \ d \mu
  \]

  Now writing this in terms of $f, g$ gives us
  \begin{align*}
    \frac{1}{AB} \int fg \ d \mu &\le \frac{1}{p} \frac{1}{A^p} \int
    f^p \ d \mu + \frac{1}{q} \frac{1}{B^q} \int g^q \ d \mu \\
    &= \frac{1}{p}\frac{1}{A^p}\|f\|_p^p + \frac{1}{q}\frac{1}{B^q} \|g\|_q^q \\
    &= 1/p + 1/q = 1
  \end{align*}
  Thus we get Holder inequality.

  For Minkowski, consider
  \begin{align*}
    (f+g)^p &= (f+g)(f+g)^{p-1} \\
    &= f(f+g)^{p-1} + g(f+g)^{p-1}
  \end{align*}
  Now integrating both sides and carefully applying Holder's inequality, we get
  \begin{align*}
    \int (f+g)^p \ d m &= \int f(f+g)^{p-1} \ d   \mu + \int
    g(f+g)^{p-1} \ d  \mu \\
    &= \Bigg(\int f^p \ d  \mu \Bigg)^p  \Bigg(\int
    (f+g)^{(p-1)q} \ d \mu \Bigg)^q + \Bigg(\int
    g^q  \ d \mu\Bigg)^{q} \Bigg(\int (f+g)^{(p-1)p} \ d \mu \Bigg)^{p} \\
    &=
  \end{align*}
  \textcolor{red}{verify}
\end{proof}

\begin{definition}
  Let $0< p< \infty$. $ f: X \to \mathbb{C}$ measurable on $(X,
  \mathcal{M}, \mu)$. We define \[
    \|f\|_p = \Big( \int |f|^p \ d \mu\Big)^p
  \]
  We also write $L^p(\mu) = \{ f : X \to \mathbb{C}  \ : \  \|f\|_p <
  \infty  \}$
\end{definition}

\begin{definition}
  Let $(X, \mathcal{M}, \mu)$ be a measure space. Let $f: X \to [0,
  \infty]$ be measurable. The essential supremeum of $f$ is \[
    \textrm{ess}\sup f = \inf \{ \alpha  \ : \  \mu(\{ f> \alpha \}) = 0  \}
  \]
\end{definition}

\begin{proposition}
  With $(X, \mathcal{M}, \mu), f$ be as above. $  \beta =
  \textrm{ess}\sup f$. Then \[
    \mu(\{ f > \beta \}) = 0
  \]
\end{proposition}

\begin{definition}
  For $(X, \mathcal{M}, \mu ), f$ as above,  \[
    \|f\|_\infty = \textrm{ess}\sup \|f\|
  \]
  and $L^\infty(\mu)$ be the set of all $f$ with $\|f\|_\infty < \infty$
\end{definition}

We add a case of Holder's inequality for $\|\cdot\|_\infty$.

\begin{theorem}
  If $(X, \mathcal{M}, \mu)$ is as usual $f, g$ measurable, $ f \in
  L^1(\mu), g \in L^\infty(\mu)$, then $ fg \in L^1(\mu)$ and \[
    \|fg\|_1 \le \|f\|_1 \|g\|_\infty
  \]
\end{theorem}
\begin{proof}
  Take $E = \{ x \in X  \ : \  |g(x)|> \|g\|_\infty \}$. Then $E$ has
  measure zero, and
  \begin{align*}
    \int |fg| \ d \mu &= \int_{X\setminus E}  |fg| \ d \mu + \int_E
    |fg| \ d \mu \\
    & \le \|g\|_\infty \int_{X\setminus E}  |f| \ d \mu \\
    & \le \|g\|_\infty \|f\|_1
  \end{align*}
\end{proof}

\begin{theorem}
  let $(X, \mathcal{M}, \mu)$ be as usual, $f, g$ measurable $ f, g
  \in L^\infty(\mu)$. Then \[
    \|f+g\|_\infty \le \|f\|_\infty + \|g\|_\infty
  \]
\end{theorem}
\begin{proof}
  Notice that
  \begin{align*}
    \{ x  :  |f(x) + g(x)| > \| f\|_\infty + \|g\|_\infty \} & \subset
    \{ x  :  |f(x)| + |g(x)| > \| f\|_\infty + \|g\|_\infty \} \\
    & \subset \{ x :  |f(x)| > \|f\|_\infty \} \ \cup \ \{ x
    : |g(x)|> \|g\|_\infty \}
  \end{align*}
  Since both the sets at the end is of measure zero. Hence we get the
  inequality.
\end{proof}

\begin{theorem}
  For each $1 \le p \le \infty$, $L^p(\mu)$ is a normed vector space
  over $\mathbb{C}$ provided we identify functions that are equal
  almost everywhere.
\end{theorem}
\begin{proof}
  Positive definiteness follows from the identification of functions
  in the space. Homogeneity follows from the definition of
  $\|\cdot\|_p$. And triangle inequality is the Minkowski inequality.
  We have shown that for the cases $1 \le p < \infty$, that
  $\|\cdot\|_p$ is a norm.
\end{proof}

\begin{lemma}
  Let $(f_n) \in L^p( \mu)$ be a Cauchy sequence in $1 \le p \le
  \infty$. Then there exists a subsequence $(f_{n_j})$ which is
  pointwise almost everywhere.
\end{lemma}



% TeX_root = ../main.tex

\chapter{}

\begin{remark}
  Consider the counting measure $\mu$, on $\mathbb{N}$. Find a
  sequence of functions $f_n : \mathbb{N} \to [0, \infty)$, such that
  $\|f_n\|_1 \to 0$ and $g = \sup_{n} f_n \notin L^1(\mu)$.
\end{remark}

\begin{lemma}
  Let $(f_n) \in L^p( \mu)$ be a Cauchy sequence in $1 \le p \le
  \infty$. Then there exists a subsequence $(f_{n_j})$ which is convergent
  pointwise almost everywhere.
\end{lemma}
\begin{proof}
  First suppose, $p < \infty$. Starting from a Cauchy sequence,
  choose a subsequence $n_1 < n_2 < \ldots$ such that for each $ k
  \in \mathbb{N}$ \[
    \|f_{n_k} - f_{n_{k+1}}\| < \frac{1}{2^k}
  \]
  Let \[
    g_l = \sum_{k = 1}^{l} |f_{n_{k+1}} - f_{n_k}| \quad g = \sum_{k
    = 1}^{\infty} |f_{n_{k+1}} - f_{n_k}|
  \]
  Then $g_n^p \le g_{n+1}^p \le \ldots$ and $g_n^p \to g^p$. Then by
  monotone convergence theorem, \[
    \int g_n^p \ d \mu \to \int g^p \ d \mu
  \]
  Moreover, using Minkowski's inequality, we get
  \begin{align*}
    \|g_l\|_p &\le \sum_{k = 1}^{l} \|f_{n_{k+1}} - f_{n_k}\| \\
    &\le \sum_{k = 1}^{\infty}  \|f_{n_{k+1}} - f_{n_k}\| \\
    &\le 1
  \end{align*}
  By monotone convergence, we get $\|g\|_p \le 1$. In particular $g$
  is finite almost everywhere. Hence \[
    f = \sum_{k = 1}^{\infty} (f_{n_{k+1}} - f_{n_k})
  \]
  is absolutely convergent almost everywhere. So by telescoping
  series for almost every $x \in X$
  \begin{align*}
    f(x) &= \lim_{l \to \infty} \sum_{k = 1}^{l} (f_{n_{k+1}} - f_{  n_k})(x) \\
    &= \lim_{ l \to \infty} (f_{n_{l+1}}(x) - f_{n_1}(x))
  \end{align*}
  So $f_{n_l}$ converges for almost every $x \in X$.

  Next, we consider $p = \infty$.  For $n, k \in \mathbb{N}$, let
  \begin{align*}
    E_{n, k} = \{ x \in X  \ : \  |f_n(x) - f_k(x)|> \|f_n - f_k\|_\infty \}
  \end{align*}
  Then $\mu(E_{n, k}) = 0$, by the definition of essential supremum.
  Moreover $E = \cup_{n, k = 1}^{\infty} E_{n, k}$ also has measure $0$.
  On $E^c$, for each $k , n \in \mathbb{N}$, we have \[
    |f_n(x) - f_k(x)| \le \| f_n - f_k\|
  \]
  This means $f_n|_E^c$ converges uniformly.
\end{proof}

\begin{theorem}
  For $1 \le p \le \infty$, $L^p(\mu)$ is a complete metric space.
  (After identifying functions that are equal almost everywhere.)
\end{theorem}
\begin{proof}
  \begin{enumerate}[label=(\arabic*)]
    \item For $p = \infty$, the proof in the above lemma is the proof
    \item For the rest of  the $p$, consider the Cauchy sequence
      $f_n$ in $L^p(\mu)$, $p < \infty$. It has c pointwise almost
      everywhere converging subseqence converging to $f$. We need to
      show that $ f \in L^p(\mu)$ and convergence is in norm. That is
      $  \|f_n - f\|_p \to 0$.

      We apply Fatou's lemma to the function $g_k = |f_n - f_{n_k}|^p$ to get
      \begin{align*}
        \lim_{k \to \infty} \inf \int |f_n - f_{n_k}|^p \ d \mu &\ge
        \int \lim_{k \to \infty} \inf |f_n - f_{n_k}|^p \ d \mu \\
        & =  \|f_n - f\|^p
      \end{align*}

      Given $\epsilon > 0$, since $f_n$ is Cauchy in $ L^p(\mu)$,
      there is a $ N$ such that for $n, m \ge N$, we have \[
        \epsilon^p > \|f_n - f_m\|^p_p = \int |f_n - f_m|^p \ d \mu
      \]
      By taking $m = n_k \to \infty$, we then get  \[
        \epsilon^p \ge \|f_n - f\|_p^p
      \]
      This implies $f \in L^p(\mu)$, by \[
        \|f\|_p \le \|f - f_n\|_p  + \|f_n\|_p
      \]
      Now that fact that $\|f - f_n\|_p \to 0$, we get $f \in L^1(\mu)$.
  \end{enumerate}
\end{proof}



% TeX_root = ../main.tex

\chapter{}

\section{Approximations by simple or continuous functions}
\begin{theorem}
  Let $(X, \mathcal{M}, \mu)$ be a measure space, denote by $S$, the
  collection of simple measurable functions with finite measurable support.
  Then for $1 \le p < \infty$, $ S \subset L^p(\mu)$ and $S$ is
  dense in $L^p(\mu)$.
\end{theorem}
\begin{proof}
  Given $f \in L^p(\mu)$, we need to find a sequence $s_n$ in $S$
  such that $s_n \to f$ in $L^1(\mu)$. First suppose that $f: X \to
  [0, \infty)$. We know a sequence of simple measurable functions
  $s_n$ such that $0 \le s_1 \le s_2 \le \ldots$ and \[
    \lim_{n \to \infty} s_n(x) = f(x)
  \]
  for each $x \in X$. Applying dominated convergence theorem, since
  $|s_n - f| \le f$, for $f \in L^p(\mu)$ gives \[
    \|f - s_n\|^p_p = \int |f - s_n|^p \ d \mu \le \int |f|^p \ d \mu < \infty
  \]
  we get $\|f - s_n\|_p \to 0$

  Now taking a general $f \in L^p(\mu)$, writing $f = u_+ - u_- +
  i(v_+ - v_-)$ and repeating the process for these gives $s = s_+ -
  s_- + i(t_+ - t_-)$ where $s_\pm, t_\pm \in S$ and \[
    \|s_\pm - u_\pm\|_p, \|t_\pm - v_\pm\|_p < \varepsilon
  \]
  hence by triangle inequality, we get \[
    \|s - f\|_p < 4 \varepsilon
  \]
  We can make RHS arbitarily small, so $S$ is dense in $L^p(\mu)$.
\end{proof}

\begin{theorem}
  Let $X$ be a locally compact Hausdorff space with $1 \le p <
  \infty$, then $C_c(X)$ is dense in $L^p(\mu)$.
\end{theorem}
\begin{proof}
  It is enough to show $\overline{C_c(X)}$ includes $S$. Given $s \in
  S$, let $A = \{ s \neq 0  \}$ with $ \mu(A) < \infty$. Then by
  Luzin's theorem, there is a $g \in C_c(X)$ such that \[
    \|g\|_\infty \le \|s\|_\infty \quad \textrm{and} \quad
    \mu(E_\varepsilon) < \varepsilon
  \]
  where $E_\varepsilon = \{ x \in X \:\ g(x) \neq s(x) \}$. Since
  $|g(x) - s(x)| \le 2 \|s\|_\infty$, we get
  \begin{align*}
    \|g - s\|_p &= \Big( \int |g - s|^p \ d \mu\Big)^{\frac{1}{p}} \\
    &= \Big( \int_{E_\varepsilon} | g - s|^p \ d \mu \Big)^{\frac{1}{p}}
  \end{align*}

  On this set, $|g - s| \le 2 \|s\|_\infty$ gives
  \begin{align*}
    \|g - s\|_\infty & \le \Big( \int_{E_\varepsilon} (2
    \|s\|_\infty)^p \ d \mu \Big)^{\frac{1}{p}} \\
    & < 2 \|s\|_\infty \varepsilon^{1/p}
  \end{align*}
  Since we can make $\varepsilon$ arbitrarily small, we get the density.
\end{proof}

\begin{remark}
  This theorem proves that $L^p(\mu)$ is the completion of
  $(C_c(\mathbb{R}^k), d_p)$ where for $f, g \in C_c(\mathbb{R}^k)$,
  $ d_p(f, g) = \| f - g\|_p$.
  The limit of a Cauchy sequence in $C_c(\mathbb{R}^k)$ is determined
  almost everywhere.

  If $p = \infty$, then the completion of $C_c(\mathbb{R}^k)$ is not
  $L^\infty(m)$, but $C_o(\mathbb{R}^k)$.
\end{remark}

\begin{definition}
  Let $X$ be locally compact Hausdorff, we say a continuous function
  $f$ vanishes at infinity and write $f \in C_o(X)$ if for
  $\varepsilon>0$, we can find a compact set $K$ such that $|f(x)| <
  \varepsilon$ for all $x \notin K$.
\end{definition}

\begin{theorem}
  Let $X$ be locally compact Hausdorff, then $C_o(X)$ is the
  completion of $C_c(X)$ with $\|\cdot\|_\infty$.
\end{theorem}
\begin{proof}
  Let $f \in C_o(X), \varepsilon > 0$, we can choose $K$ such that
  $K$ is compact and $|f(x)| < \varepsilon$ for all $x \in K^c$.
  Using Urysohn's lemma, threre is a $g \in C_c(X)$ such that $\chi_K
  \le g \le 1$, then $h = fg \in C_c(X)$ and
  \begin{align*}
    \|h - f\|_\infty &= \|f(1 - g)\|_\infty \\
    &= \|f(1-g) \chi_{K^c}\|_\infty \\
    &\le \varepsilon \|1 -g\|_\infty \\
    &\le \varepsilon
  \end{align*}
\end{proof}

\marginnote{ \scriptsize 12-11-2024}
\begin{proposition}
  Show that if $\mu(X) < \infty$, with $p \le r \le \infty$, then
  \begin{align*}
    L^r(\mu) \subset L^p(\mu)
  \end{align*}
  Given $f \in L^r(\mu)$.
\end{proposition}
\begin{proof}
  Let $f \in L^r(\mu)$. Then,
  \begin{align*}
    \|f\|_p^p &= \int |f|^p \ d m \\
    &\le \Bigg(\int |f|^{r} \ d m \Bigg) ^{p/r} \Bigg(\int 1 \ d
    m\Bigg)^{1 - p/r} \\
    &\le \|f\|_r^p \mu(X)^{1- \frac{p}{r}} \\
    &< \infty
  \end{align*}
\end{proof}



% TeX_root = ../main.tex

\chapter{Inner Product Spaces}

\begin{definition}
  Let $\mathcal{H}$ be a vector space over $\mathbb{C}$. A
  sesquilinear form is a function
  \begin{align*}
    \langle \cdot , \cdot \rangle : \mathcal{H} \times \mathcal{H}
    \to \mathbb{C}
  \end{align*}
  satisfying
  \begin{itemize}[]
    \item $\langle  x , y \rangle  = \overline{\langle y , x \rangle }$
    \item $\langle  x + \alpha z , y \rangle  = \langle x , y \rangle
      + \alpha \langle x , z \rangle $
  \end{itemize}
  for all $x, y, z \in \mathcal{H}, \alpha \in \mathbb{C}$.
  It is said to be positive semidefinite (positive definite) if $
  \langle x , x \rangle
  \ge 0$ ($\langle  x , x \rangle > 0$ for all $x \in
  \mathcal{H}\setminus \{0\}$) for all $x \in \mathcal{H}$.

  A positive definite sesquilinear forms makes $\mathcal{H}$ an inner
  product space.
\end{definition}

\begin{example}
  Take $L^2(\mu)$ (functions identified almost everywhere) with the
  natural inner product is an inner product.
\end{example}

\begin{proposition}
  If $\mathcal{H}$ is a complex vector space with a positive
  semidefinite sesquilinear form and $\langle  x , x \rangle = 0$,
  then $\langle  x , y \rangle  = 0$ for all $y \in \mathcal{H}$.
\end{proposition}
\begin{proof}
  Take $\alpha \in \mathbb{C}$ and consider
  \begin{align*}
    \langle x + \alpha y , x  +\alpha y \rangle  & = \langle x , x
    \rangle  + \alpha \langle y , x \rangle  + \overline{\alpha}
    \langle y , y \rangle  \\
    &= 2 \Re \big(  \overline{\alpha} \langle x , y \rangle  \big) +
    |\alpha|^2 \langle y , y \rangle
  \end{align*}
  Now if $\langle  x , y \rangle \neq 0$, then either $\langle  y , y
  \rangle  = 0$ or nonzero. If $\langle  y , y \rangle = 0$, take
  $\alpha = - \langle x , y \rangle $ to get
  \begin{align*}
    \langle x +\alpha y , x + \alpha y \rangle = \underbrace{2\Re \big( -
    \overline{\langle x , y \rangle }\langle x , y \rangle \big)}_{< 0}
  \end{align*}
  which is a contradiction.

  Now if $\langle  y , y \rangle  \neq 0$, take $\alpha = i \langle
  x , y \rangle $ to get a similiar contradiction, which makes $\Re(
  \overline{\alpha} \langle x , y \rangle) = 0$
\end{proof}

\begin{definition}
  If $\langle  \cdot , \cdot \rangle $ is a positive semidefinite
  sesquilinear form, then
  \begin{align*}
    \|x\| =  \langle x , x \rangle^{\frac{1}{2}}
  \end{align*}
  is a seminorm.

  If $\langle \cdot , \cdot \rangle $ is positive definite, then $x
  \to \|x\|$ is a norm.
\end{definition}

\begin{theorem}[Cauchy-Schwarz]
  If $\langle \cdot , \cdot \rangle $ is a positive semidefinite
  sesquilinear form on $\mathcal{H}$, then for $x, y \in \mathcal{H}$
  \begin{align*}
    |\langle x , y \rangle | \le \|x\| \|y\|
  \end{align*}
\end{theorem}
\begin{proof}
  If $\|y\| = 0$, then previous proposition takes care of the proof.
  If not, choose $ \alpha =  \frac{\langle x , y \rangle }{ \langle y
  , y \rangle } $ and consider
  \begin{align*}
    0 &\le \langle x - \lambda y , x- \lambda y \rangle \\
  & = \|x\|^2 - 2 \Re \big( \lambda \langle y , x \rangle \big)) +
  |\lambda|^2 \langle y , y \rangle \\
  &= \|x\| - 2 \frac{\langle x , y \rangle^2}{\|y\|^2} +
  \frac{\langle x , y \rangle^2}{\|y\|^2} \\
  &= \|x\|^2 - \frac{|\langle x , y \rangle |^2}{\|y\|^2}
\end{align*}
which gives our inequality.
\end{proof}



% TeX_root = ../main.tex

\marginnote{ \scriptsize 13/11/2024}

\begin{theorem}
  Let $\mathcal{H}$ be a vector space over $\mathbb{C}$ with a
  positive semidefinite sesquilinear $ \langle \cdot , \cdot \rangle
  $ form and the associated seminorm $ \|\cdot\|$, then for all $x, y
  \in \mathcal{H}$,
  \begin{align*}
    \|x+y\| \le \|x\| + \|y\|
  \end{align*}
\end{theorem}
\begin{proof}
  \textcolor{red}{verify}
\end{proof}

\begin{remark}
  If $\langle \cdot , \cdot \rangle $ is an inner product space, then
  $\|\cdot\|$ defines a norm in $ \mathcal{H}$.
\end{remark}

\begin{definition}
  If $ \mathcal{H}$ be an inner product. If $\mathcal{H}$ is complete
  with respect to the topology induced by the inner product, then it
  is called a Hilbert space.
\end{definition}

\begin{example}
  $L^2(\mu)$, with functions identified that agrees almost everywhere
  is a Hilbert space when endowed with the inner product
  \begin{align*}
    \langle f , g \rangle  = \int f \overline{g} \ d \mu
  \end{align*}
\end{example}

\begin{proposition}
  Let $\mathcal{H}$ be a Hilbert space. Then for $g \in \mathcal{H}$
  \begin{align*}
    \lambda_g : \mathcal{H} \to \mathbb{C} := f \to \langle f , g \rangle
  \end{align*}
  is a linear, uniformly continuous functional.
\end{proposition}
\begin{proof}
  Use Cauchy-Schwarz inequality.
\end{proof}

\begin{definition}
  Let $H$ be a Hilbert space. We say $x, y \in \mathcal{H}$ are
  orthogonal if $\langle  x , y \rangle  = 0$. We also write $x \perp y$.

  If $ S \subset \mathcal{H}$, define
  \begin{align*}
    S^\perp = \{ x \in \mathcal{H}  \ : \   x \perp s, \forall s \in S \}
  \end{align*}
\end{definition}

\begin{theorem}
  If $S \subset H$, then $S^\perp$ is a closed subspace of $\mathcal{H}$.
\end{theorem}
\begin{proof}
  Let $z \in \mathcal{H}$. Then $K_z = z^\perp =
  \textrm{Ker}(\lambda_z)$ is closed. Observe that
  \begin{align*}
    S^\perp = \bigcap_{s \in S}K_s
  \end{align*}
  is closed as well.
\end{proof}

\begin{lemma}
  Let $\mathcal{M}$ be a closed subspace of a Hilbert space
  $\mathcal{H}$, and $h \in \mathcal{H}$, then there is a unique $m
  \in \mathcal{M}$ that minimizes the distance to $h$
\end{lemma}
\begin{proof}
  We recall the parallelogram law
  \begin{align*}
    \|x + y\|^2 + \|x -y\|^2 = 2 ( \|x\|^2 + \|y\|^2)
  \end{align*}
  and write for $x, y \in \mathcal{H}$,
  \begin{align*}
    \|x - y\|^2 = 2 (\|x\|^2 + \|y\|^2) - \|x + y\|^2
  \end{align*}
  Let $\delta = \inf \{ \|m - h\|  \ : \  m \in \mathcal{M}  \}$.
  There there is a sequence of $m_j \in \mathcal{M}$ such that $\|m_j
  - h\| \to \delta$. To show that $m_j$ is a Cauchy seqeunce,
  consider $x = m_j -h, y = m_i - h$. Then
  \begin{align*}
    \frac{x+y}{2} = \frac{m_i + m_j}{2} - h
  \end{align*}
  and we see that
  \begin{align*}
    \Big\|\frac{x+y}{2}\Big\| = \Big|\frac{m_i+m_j}{2} - h\Big|
  \end{align*}
  Then by prarllelogram law,
  \begin{align*}
    \|m_j - m_i\|^2 &= 2(\|x\|^2 + \|y\|^2) - \|x + y\|^2 \\
    &=2(\|m_j - h\|^2 + \|m_i - h\|^2 - \|m_i + m_j - 2h\|^2) \\
  \end{align*}
  \textcolor{red}{verify}

  This shows that, we can make $\|m_i - m_j\|$ arbitrarily small by
  requiring $ i, j \in \mathbb{N}$ for a similarly large
  $\mathbb{N}$, meaning $ m_j $ is Cauchy. Since $ \mathcal{M}$ is
  closed and a closed and a closed subset of a complete metric space,
  $\mathcal{M}$ is complete, so there is a point $m \in \mathcal{M}$,
  where $m_j$ converges to.
  We'll prove the uniqueness in the next lecture.
\end{proof}



% TeX_root = ../main.tex

\marginnote{ \scriptsize 19/11/24}

\begin{theorem}[Orthogonal Projections]
  If $M$ is a closed subspace of a Hilbert space $\mathcal{H}$, then
  for each $h \in \mathcal{H}$, there is a unique pair $m \in M$ and
  $ n \in M^\perp$ such that $h = m + n$ and $\|h\|^2 = \|m\|^ +
  \|n\|^2$. Moreover, the maps $P(h) = m$, $Q(h) = n$ are linear and
  write $m = Ph, n = Qh$
\end{theorem}
\begin{proof}
  Fix $h \in \mathcal{H}$. Pick $m \in M$ that is nearest to $h$,
  using precedent lemma. Let $n = h-m$. We will show $ n \in
  M^\perp$. For any $ x \in M$, $\alpha \in \mathbb{C}$, consider
  \begin{align*}
    \|n - \alpha x\|^2 &= \|n\|^2 - 2 \Re (\alpha \langle x , n
    \rangle ) + |\alpha|^2 \|x\|^2
  \end{align*}
  and $\|h\|^2 = \|m\|^2 + \|n\|^2$. Suppose $\langle  x , n \rangle
  \neq 0$. Choose $\alpha = \frac{t}{\langle x , n \rangle }$ for $t \in
  \mathbb{R}$. Then
  \begin{align*}
    \|n - \alpha x\|^2 &= \|n\|^2 - 2t + \frac{t^2\|x\|^2}{|\langle x
    , n \rangle |^2}
  \end{align*}
  For sufficiently small $t$, we have $2t >
  \frac{t^2\|x\|^2}{|\langle x, n \rangle |^2}$. Then we'd get
  \begin{align*}
    \|n - \alpha x\|^2 < \|n\|^2
  \end{align*}
  Replacing $n$ with $h -m$, we get
  \begin{align*}
    \|h - (m + \alpha x)\|^2 < \| h -m\|^2
  \end{align*}
  which contradicts the optimality of $m$ for distance to $h$. We
  conclude $ \langle x , n \rangle = 0$.
  This is true for each $x \in M$. Thus, we get $ n \in M^\perp$.

  Now to see that the choice of $m$(and $n$) is unique, let $h =
  m^\prime + n^\prime$, with $m^\prime \in M, n^\prime \in M^\perp$.
  Then $m + n = m^\prime + n^\prime$, which implies
  \begin{align*}
    \underbrace{m - m^\prime}_{\in M} = \underbrace{n^\prime - n}_{\in M^\perp}
  \end{align*}
  which forces $ m = m^\prime, n = n^\prime$, since $ M \cap M^\perp = \{ 0 \}$

  Now for the linearity of $P, Q$, let $h = h_1 + \alpha h_2$, where
  $h_1 = m_1 + n_1, h_2 = m_2 + n_2$ for $ m_i \in M,
  n_i \in M^\perp$. Then
  \begin{align*}
    h = \underbrace{m_1 + \alpha m_2}_{\in M} + \underbrace{ n_1 +
    \alpha n_2}_{\in M^\perp}
  \end{align*}
  This shows $P: h \to m, Q: h \to n$ are linear maps.
\end{proof}

\begin{definition}
  The maps $P, Q$ above are called orthogonal projections onto $M$
  and $M^\perp$, respectively.
\end{definition}

\begin{corollary}
  Let $M$ be a proper closed subspace in a Hilbert space
  $\mathcal{H}$. Then $M^\perp \neq \{ 0\}$.
\end{corollary}

\begin{exercise}
  Let $M \subset L^2(\mathbb{R})$ such that
  \begin{align*}
    M = \{ f \in L^2(\mathbb{R})  \ : \  f(x) = \alpha_n  \textrm{
    for almost every } x \in [n, n+1) \}
  \end{align*}
\end{exercise}

\section{Reisz Representation Theorem}

\begin{theorem}[Reisz Representation Theorem]
  Let $\Lambda: \mathcal{H} \to \mathbb{C}$ be a continuous linear
  functional on a Hilbert space. Then there is a unique $y \in
  \mathcal{H}$ such that $\Lambda(x) = \langle x , y \rangle $
\end{theorem}
\begin{proof}
  Assume that $\Lambda \neq 0$. Then $M = \textrm{Ker}(\Lambda)$ is a proper
  closed linear subspace of $\mathcal{H}$. Then so is $M^\perp$. Let
  $0 \neq v, w \in M^\perp$. Then $\Lambda(v) \neq 0 \neq
  \Lambda(w)$. Then since
  \begin{align*}
    \Lambda \Big( \frac{v}{\Lambda(v)} - \frac{w}{\Lambda(w)} \Big) = 0
  \end{align*}
  forces $\frac{v}{\Lambda(v)} - \frac{w}{\Lambda(w)} \in M \cap
  M^\perp = \{ 0 \}$. Hence $v \in \textrm{span}(w)$. Thus we see
  that $M$ has co-dimension 1. i.e $M^\perp$ has dimension $1$.
\end{proof}



% TeX_root = ../main.tex

\marginnote{\scriptsize 21/11/2024 }

\section{Reisz Representation Theorem}

\begin{theorem}[Reisz Representation Theorem]
  Let $\Lambda: \mathcal{H} \to \mathbb{C}$ be a continuous linear
  functional on a Hilbert space. Then there is a unique $y \in
  \mathcal{H}$ such that $\Lambda(x) = \langle x , y \rangle $
\end{theorem}
\begin{proof}
  Assume that $\Lambda \neq 0$. Then $M = \textrm{Ker}(\Lambda)$ is a proper
  closed linear subspace of $\mathcal{H}$. Then so is $M^\perp$. Let
  $0 \neq v, w \in M^\perp$. Then $\Lambda(v) \neq 0 \neq
  \Lambda(w)$. Then since
  \begin{align*}
    \Lambda \Big( \frac{v}{\Lambda(v)} - \frac{w}{\Lambda(w)} \Big) = 0
  \end{align*}
  forces $\frac{v}{\Lambda(v)} - \frac{w}{\Lambda(w)} \in M \cap
  M^\perp = \{ 0 \}$. Hence $v \in \textrm{span}(w)$. Thus we see
  that $M$ has co-dimension 1. i.e $M^\perp$ has dimension $1$.

  Now consider $P$, the orthogonal projection to $M$ and $Q = 1 - P$.
  Then $ x = Px + Qx$ for all $ x \in \mathcal{H}$. Hence
  \begin{align*}
    \Lambda(x) = \Lambda(Px + Qx)  = \Lambda(Qx)
  \end{align*}
  Since $M^\perp$ is a one dimensional subspace $ Qx = \alpha v$ for
  $ \alpha \in \mathbb{C}$ and $0 \neq v \in M^\perp$ with
  $\Lambda(v) = 1$. This gives that
  \begin{align*}
    \Lambda(x) &= \alpha \Lambda(v)  \\
    &= \alpha \\
    &= \alpha \langle v , \frac{v}{\|v\|^2} \rangle \\
    &= \langle \alpha v , \frac{v}{\|v\|^2} \rangle \\
    &= \langle Qx , \frac{v}{\|v\|^2} \rangle  + 0\\
    &=\langle Qx , \frac{v}{\|v\|^2} \rangle + \langle Px ,
    \frac{v}{\|v\|^2} \rangle \quad (\textrm{Since } Px \perp M^\perp) \\
    &= \langle Qx + Px , \frac{v}{\|v\|^2} \rangle \\
    &= \langle x , \frac{v}{\|v\|^2} \rangle
  \end{align*}
\end{proof}

\section{Orthonormal Sets}
\begin{definition}
  A family $\{ u_\alpha \}_{\alpha \in A}$ is called orthonormal if $
  \langle u_\alpha ,  u_\beta \rangle = \delta_{a, b}$ for each
  $\alpha, \beta \in A$.

  If $x \in \mathcal{H}$, then $\langle  x , u_\alpha \rangle $ is
  called a Fourier coefficient of $x$ relative to $u_\alpha$.
\end{definition}

We consider finite orthonormal sets first.
\begin{proposition}
  Let $\{ u_\alpha \}_{\alpha \in A}$ be an orthonormal set and $ F
  \subset A$ be finite. Let $M_F = \textrm{span}\{ u_\alpha \}_{\alpha \in F}$.
  \begin{enumerate}[label=(\alph*)]
    \item  If $\phi: A \to \mathbb{C}$, $\phi|_{A \setminus F}  = 0$,
      then there is $ y \in M_F$ such that
      \begin{align*}
        y = \sum_{\alpha \in  F} \phi(\alpha)   u_\alpha
      \end{align*}
      and $\phi(\alpha) = \langle y , u_\alpha \rangle $ for each $a
      \in A$. Also
      \begin{align*}
        \|y\|^2 = \sum_{\alpha \in  F} |\phi(\alpha)|^2
      \end{align*}

    \item If $x \in \mathcal{H}$, then
      \marginnote{ \scriptsize This says that the orthogonal projection is
      the best approximation to the subspace}
      \begin{align*}
        \Big \| x = \sum_{\alpha \in  F} \langle x , u_\alpha \rangle
        u_\alpha \Big \| \le \|x - s\| \quad \textrm{for any } s \in M_F
      \end{align*}
  \end{enumerate}
\end{proposition}
\begin{proof}
  \begin{enumerate}[label=(\alph*)]
    \item Straightforward.
    \item Let $S(x) = \sum_{\alpha \in  F} \langle x , u_\alpha
      \rangle  u_\alpha$. Note that $ \langle S(x) ,  u_\alpha
      \rangle = \langle x , u_\alpha \rangle$ for each $ \alpha \in
      F$. Thus we see that $ \langle x - S(x) , u_\alpha \rangle = 0$
      for each $ \alpha \in F$. Because $M_F = \textrm{span}\{
      u_\alpha  \ : \   \alpha \in F \}$, we see that $(x - S(x))
      \perp v$ for any $v \in M_F$. Thus $(x - S(x))\perp (S(x) -
      v)$. Thus be Pythagoras theorem, we get
      \begin{align*}
        \|x - v\|^2 &= \|x - S(x)\|^2 + \|v - S(x)\|^2 \ge \|x - S(x)\|^2
      \end{align*}
      Thus we see that $S(x)$ is the best approximation of $x$ on to the
      space $M_F$. Thus we see that $S$ is the orthogonal projection to $ M$.
  \end{enumerate}
\end{proof}



\printbibliography[heading=bibintoc]

\end{document}


% TeX_root = ../main.tex

\chapter{}

Assignment 1 is posted. Submissions due Aug 29.

\section{Warm up}
\begin{example}
  Let $X = \{ 1, 2, 3 \}, F = \{ \{ 1, 2 \}, \{ 1, 3 \} \}$. Then the smallest topology containing $F$ is $\{ \emptyset, X, \{ 1 \}, \{ 1, 2 \}, \{ 1, 3 \} \}$, and the $\sigma$-algebra generated by $F$ is the power set, $P(X)$.
\end{example}
\section{continues}
\begin{proof}
  Proof of \autoref{thm:sigma_algeba_generated}.

  Consider all $\sigma$-algebras containing $F$, let $\Omega = \{ \mathscr{N} \subset P(X) \ : \ \mathscr{N} \supset F, \mathscr{N} \textrm{ is a } \sigma-\textrm{algebra} \}$. $\Omega$  is non-empty since $P(X) \subset \Omega$. Let \[
      \mathscr{M} = \cap_{\mathscr{N} \in  \Omega} \mathscr{N}
  \]
   Then we claim $\mathscr{M}$ is a $\sigma$-algebra. To see this 
   \begin{itemize}
     \item $X \in \mathscr{M}$, because $X \in \mathscr{N}$, for each $\mathscr{N} \in  \Omega$.
     \item If $E \in \mathscr{M}$, then $E \in \mathscr{N}$ for each $\mathscr{N} \in \Omega$. Then $E^c \in \mathscr{N}$ for each $\mathscr{N} \in \Omega$ and thus $E^c \in \mathscr{M}$.
     \item If $A_1, A_2, \ldots \in \mathscr{M}$, then $\cup_{j = 1}^{\infty}A_j \in \mathscr{M}$ because since each $A_i \in \mathscr{N}$ and $\mathscr{N}$ is a $\sigma$-algebra, $\cup_{j = 1}^{\infty}A_j \in \mathscr{N}$ for each $\mathscr{N} \in \Omega$.
   \end{itemize}
  Moreover, $F \subset \mathscr{M}$ since $F \subset \mathscr{N}$ for each $\mathscr{N} \in \Omega$.
  Finally, if $\mathscr{N}$ is a $\sigma$-algebra with $\mathscr{N} \supset F$, then $\mathscr{N} \in \Omega$. Then $\mathscr{M} \subset \mathscr{N}$.
  To prove uniqueness, let $\mathscr{M}_0$ be a $\sigma$-algebra which satisfies the required properties defining $\Omega$. By intersection operation giving $\mathscr{M}$, and $\mathscr{M}_0 \in \Omega$, $M \subset M_0$. Additionally, if $\mathscr{M}_0$ satisfies that $\mathscr{M}_0 \subset \mathscr{N}$ for each $\mathscr{N} \in \Omega$, then $\mathscr{M}_0 \subset \mathscr{M}$. Thus $\mathscr{M}_0 = \mathscr{M}$.
\end{proof}

We combine concepts of topologies and $\sigma$-algebras. 
\begin{definition}
  Let $(X, \tau)$ be any topological space. The $\sigma$-algebra, $\mathscr{B}$ generated by the topology $\tau$ is called the Borel $\sigma$-algebra. Elements of $\mathscr{B}$ are called Borel sets.
\end{definition}
\begin{definition}
  Let $X, Y$ be topological spaces. A map $f: X \to Y$ is continuous if the inverse image of any open set is open. The map $f$ is continuous at $x \in X$ if every open set $V \subset Y$ with $f(x) \in V$, there is an open set $W \subset X$ with $f(W) \subset V$.
\end{definition}

\begin{theorem}
  A map $f: X \to Y$ is continuous if and only if it is continuous at each $x \in X$.
\end{theorem}
\begin{proof}
  ($\implies$) If $f$ is continuous and $x \in X$, $V \subset Y$ is open and $f(x) \in V$, then by continuity, $f^{-1}(V)$ is open and $x \in f^{-1}(V)$. This holds for any such $x$ and $V$, thus $f$ is continuous at $x \in X$. Since $x$ was arbitrarily chosen, $f$ is continuous at each $x \in X$.

  ($\impliedby$) Suppose $f$ is continuous at each $x \in X$. Let $V$ be an open subset of $Y$. Need to show that $W = f^{-1}(V)$ is open. For each $x \in W$, there is a $W_x \subset X$ which is open with $x \in W_x$ and $f(W_x) \subset V$ by the continuity of $f$ at $x$. Now take $$Y = \bigcup_{x \in W} W_x$$
  Then $Y$ is open being a union of open sets. Also it contains each $x \in W$. Hence $W \subset Y$. But again, $W_x \subset W = f^{-1}(V)$ for each $x \in W$ and taking the unions preserve the inclusion. Hence we get $W = Y$. Since we already know $Y$ is open, this gives us $W = f^{-1}(V)$ is open.
\end{proof}

\begin{proposition}
  If $f: X \to Y$ and $f: Y \to Z$ are continuous, then so is $g\circ f: X \to Z$.
\end{proposition}
\begin{proof}
  Let $V \subset Z$ be an open set. Then $f^{-1}(V)$ is open in $Y$ by the continuity of $f$. Similarly, $g^{-1}(f^{-1}(V))$ is open in $X$ by the continuity of $g$. But $ g^{-1}(f^{-1}(V)) = (g \circ f)^{-1}(V)$. Since $V$  was arbitrarily open, we get that $g\circ f$ is continuous.
\end{proof}

\begin{definition}
   Let $X$ be a measurable space and $Y$ a topological space. Then a map $f: X \to Y$ is called measurable, if all inverse images of open sets are measurable.
\end{definition}

\begin{proposition}
  \label{prop:measurabel_sets_with_open_sets}
  Let $X$ be a measurable space, $Y$ be a topological space, then $f: X \to Y$ is measurable if and only if $f^{-1}( B)$ is measurable for each Borel set $B$.
\end{proposition}
\begin{proof}
  ($\implies$) Every open set is a Borel set. So this is true by inclusion.

  ($\impliedby$) Suppose $f$ is measurable. Let $M = \{ E \subset Y \ : \ f^{-1}(E) \textrm{  is measurable } \}$. We know $M$ contains all open sets (Since we assume $f$ is measurable). Moreover since $f^{-1}(\cup_{j \in J} U_j) = \cup_{j \in J}f^{-1}(U_j)$ for any open sets $U_j \subset Y$ with index set $J$, and $ f^{-1}(\cap_{i = 1}^{n}U_i) = \cap_{i = 1}^{n} f^{-1}(U_i)$, we get that $M$ is a $\sigma$-algebra.

  Since $M$ contains all open sets, $M$ contains the Borel $\sigma$-algebra in $Y$. Hence $f^{-1}(B)$ is  measurable for every Borel set $B$.
\end{proof}














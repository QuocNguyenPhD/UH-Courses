% TeX_root = ../main.tex

\marginnote{ \scriptsize 19/11/24}

\begin{theorem}[Orthogonal Projections]
  If $M$ is a closed subspace of a Hilbert space $\mathcal{H}$, then
  for each $h \in \mathcal{H}$, there is a unique pair $m \in M$ and
  $ n \in M^\perp$ such that $h = m + n$ and $\|h\|^2 = \|m\|^ +
  \|n\|^2$. Moreover, the maps $P(h) = m$, $Q(h) = n$ are linear and
  write $m = Ph, n = Qh$
\end{theorem}
\begin{proof}
  Fix $h \in \mathcal{H}$. Pick $m \in M$ that is nearest to $h$,
  using precedent lemma. Let $n = h-m$. We will show $ n \in
  M^\perp$. For any $ x \in M$, $\alpha \in \mathbb{C}$, consider
  \begin{align*}
    \|n - \alpha x\|^2 &= \|n\|^2 - 2 \Re (\alpha \langle x , n
    \rangle ) + |\alpha|^2 \|x\|^2
  \end{align*}
  and $\|h\|^2 = \|m\|^2 + \|n\|^2$. Suppose $\langle  x , n \rangle
  \neq 0$. Choose $\alpha = \frac{t}{\langle x , n \rangle }$ for $t \in
  \mathbb{R}$. Then
  \begin{align*}
    \|n - \alpha x\|^2 &= \|n\|^2 - 2t + \frac{t^2\|x\|^2}{|\langle x
    , n \rangle |^2}
  \end{align*}
  For sufficiently small $t$, we have $2t >
  \frac{t^2\|x\|^2}{|\langle x, n \rangle |^2}$. Then we'd get
  \begin{align*}
    \|n - \alpha x\|^2 < \|n\|^2
  \end{align*}
  Replacing $n$ with $h -m$, we get
  \begin{align*}
    \|h - (m + \alpha x)\|^2 < \| h -m\|^2
  \end{align*}
  which contradicts the optimality of $m$ for distance to $h$. We
  conclude $ \langle x , n \rangle = 0$.
  This is true for each $x \in M$. Thus, we get $ n \in M^\perp$.

  Now to see that the choice of $m$(and $n$) is unique, let $h =
  m^\prime + n^\prime$, with $m^\prime \in M, n^\prime \in M^\perp$.
  Then $m + n = m^\prime + n^\prime$, which implies
  \begin{align*}
    \underbrace{m - m^\prime}_{\in M} = \underbrace{n^\prime - n}_{\in M^\perp}
  \end{align*}
  which forces $ m = m^\prime, n = n^\prime$, since $ M \cap M^\perp = \{ 0 \}$

  Now for the linearity of $P, Q$, let $h = h_1 + \alpha h_2$, where
  $h_1 = m_1 + n_1, h_2 = m_2 + n_2$ for $ m_i \in M,
  n_i \in M^\perp$. Then
  \begin{align*}
    h = \underbrace{m_1 + \alpha m_2}_{\in M} + \underbrace{ n_1 +
    \alpha n_2}_{\in M^\perp}
  \end{align*}
  This shows $P: h \to m, Q: h \to n$ are linear maps.
\end{proof}

\begin{definition}
  The maps $P, Q$ above are called orthogonal projections onto $M$
  and $M^\perp$, respectively.
\end{definition}

\begin{corollary}
  Let $M$ be a proper closed subspace in a Hilbert space
  $\mathcal{H}$. Then $M^\perp \neq \{ 0\}$.
\end{corollary}

\begin{exercise}
  Let $M \subset L^2(\mathbb{R})$ such that
  \begin{align*}
    M = \{ f \in L^2(\mathbb{R})  \ : \  f(x) = \alpha_n  \textrm{
    for almost every } x \in [n, n+1) \}
  \end{align*}
\end{exercise}

\section{Reisz Representation Theorem}

\begin{theorem}[Reisz Representation Theorem]
  Let $\Lambda: \mathcal{H} \to \mathbb{C}$ be a continuous linear
  functional on a Hilbert space. Then there is a unique $y \in
  \mathcal{H}$ such that $\Lambda(x) = \langle x , y \rangle $
\end{theorem}
\begin{proof}
  Assume that $\Lambda \neq 0$. Then $M = \textrm{Ker}(\Lambda)$ is a proper
  closed linear subspace of $\mathcal{H}$. Then so is $M^\perp$. Let
  $0 \neq v, w \in M^\perp$. Then $\Lambda(v) \neq 0 \neq
  \Lambda(w)$. Then since
  \begin{align*}
    \Lambda \Big( \frac{v}{\Lambda(v)} - \frac{w}{\Lambda(w)} \Big) = 0
  \end{align*}
  forces $\frac{v}{\Lambda(v)} - \frac{w}{\Lambda(w)} \in M \cap
  M^\perp = \{ 0 \}$. Hence $v \in \textrm{span}(w)$. Thus we see
  that $M$ has co-dimension 1. i.e $M^\perp$ has dimension $1$.
\end{proof}



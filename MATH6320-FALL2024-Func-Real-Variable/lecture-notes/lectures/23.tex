% TeX_root = ../main.tex

\chapter{Inner Product Spaces}

\begin{definition}
  Let $\mathcal{H}$ be a vector space over $\mathbb{C}$. A
  sesquilinear form is a function
  \begin{align*}
    \langle \cdot , \cdot \rangle : \mathcal{H} \times \mathcal{H}
    \to \mathbb{C}
  \end{align*}
  satisfying
  \begin{itemize}[]
    \item $\langle  x , y \rangle  = \overline{\langle y , x \rangle }$
    \item $\langle  x + \alpha z , y \rangle  = \langle x , y \rangle
      + \alpha \langle x , z \rangle $
  \end{itemize}
  for all $x, y, z \in \mathcal{H}, \alpha \in \mathbb{C}$.
  It is said to be positive semidefinite (positive definite) if $
  \langle x , x \rangle
  \ge 0$ ($\langle  x , x \rangle > 0$ for all $x \in
  \mathcal{H}\setminus \{0\}$) for all $x \in \mathcal{H}$.

  A positive definite sesquilinear forms makes $\mathcal{H}$ an inner
  product space.
\end{definition}

\begin{example}
  Take $L^2(\mu)$ (functions identified almost everywhere) with the
  natural inner product is an inner product.
\end{example}

\begin{proposition}
  If $\mathcal{H}$ is a complex vector space with a positive
  semidefinite sesquilinear form and $\langle  x , x \rangle = 0$,
  then $\langle  x , y \rangle  = 0$ for all $y \in \mathcal{H}$.
\end{proposition}
\begin{proof}
  Take $\alpha \in \mathbb{C}$ and consider
  \begin{align*}
    \langle x + \alpha y , x  +\alpha y \rangle  & = \langle x , x
    \rangle  + \alpha \langle y , x \rangle  + \overline{\alpha}
    \langle y , y \rangle  \\
    &= 2 \Re \big(  \overline{\alpha} \langle x , y \rangle  \big) +
    |\alpha|^2 \langle y , y \rangle
  \end{align*}
  Now if $\langle  x , y \rangle \neq 0$, then either $\langle  y , y
  \rangle  = 0$ or nonzero. If $\langle  y , y \rangle = 0$, take
  $\alpha = - \langle x , y \rangle $ to get
  \begin{align*}
    \langle x +\alpha y , x + \alpha y \rangle = \underbrace{2\Re \big( -
    \overline{\langle x , y \rangle }\langle x , y \rangle \big)}_{< 0}
  \end{align*}
  which is a contradiction.

  Now if $\langle  y , y \rangle  \neq 0$, take $\alpha = i \langle
  x , y \rangle $ to get a similiar contradiction, which makes $\Re(
  \overline{\alpha} \langle x , y \rangle) = 0$
\end{proof}

\begin{definition}
  If $\langle  \cdot , \cdot \rangle $ is a positive semidefinite
  sesquilinear form, then
  \begin{align*}
    \|x\| =  \langle x , x \rangle^{\frac{1}{2}}
  \end{align*}
  is a seminorm.

  If $\langle \cdot , \cdot \rangle $ is positive definite, then $x
  \to \|x\|$ is a norm.
\end{definition}

\begin{theorem}[Cauchy-Schwarz]
  If $\langle \cdot , \cdot \rangle $ is a positive semidefinite
  sesquilinear form on $\mathcal{H}$, then for $x, y \in \mathcal{H}$
  \begin{align*}
    |\langle x , y \rangle | \le \|x\| \|y\|
  \end{align*}
\end{theorem}
\begin{proof}
  If $\|y\| = 0$, then previous proposition takes care of the proof.
  If not, choose $ \alpha =  \frac{\langle x , y \rangle }{ \langle y
  , y \rangle } $ and consider
  \begin{align*}
    0 &\le \langle x - \lambda y , x- \lambda y \rangle \\
  & = \|x\|^2 - 2 \Re \big( \lambda \langle y , x \rangle \big)) +
  |\lambda|^2 \langle y , y \rangle \\
  &= \|x\| - 2 \frac{\langle x , y \rangle^2}{\|y\|^2} +
  \frac{\langle x , y \rangle^2}{\|y\|^2} \\
  &= \|x\|^2 - \frac{|\langle x , y \rangle |^2}{\|y\|^2}
\end{align*}
which gives our inequality.
\end{proof}



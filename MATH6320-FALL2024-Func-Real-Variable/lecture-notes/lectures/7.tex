% TeX_root = ../main.tex

\chapter{}

\begin{remark}
  \label{remark:measure_from_integral}
  Let $(X, \mathcal{M}, \mu)$ be a measure space, a simple function $s: X \to [0, \infty]$, then $\phi: \mathcal{M} \to [0, \infty]$ defined as \[
    \phi(E) = \int_E s \ d \mu
  \]
  is a measure.
\end{remark}
\begin{proof}
  Since our definiton demands that measure of some set should be finite, we verify this first. We see that \[
    \phi(\emptyset) = \int_\emptyset s \ d \mu = 0
  \]
  Now to prove countable disjoint additivity, consider the disjoint collection $\{ E_l \}_{l \in \mathbb{N}}$. And assume that $s = \sum_{j = 1}^{n} \alpha_j \chi_{A_j}$ with $ \alpha_j \in [0, \infty]$, with $A_j$s disjoint. Then for $E = \cup_{l = 1}^{\infty}E_l$, we have \begin{align*}
    \phi(E) &= \sum_{j = 1}^{n} \alpha_j \mu(A_j \cap E) \\ 
    &= \sum_{ j = 1}^{n} \sum_{l \in \mathbb{N}} \alpha_j \mu(A_j \cap E_l) \\ 
    &= \sum_{ l \in \mathbb{N}} \sum_{j = 1}^{n} \alpha_j \mu(A_j \cap E_l) \\ 
    &= \sum_{ l \in \mathbb{N}} \int_{E_l} s \ d \mu
  \end{align*}
\end{proof}


\section{Properties of Integrals}

\begin{theorem}
  \label{thm:properties_of_integrals}
  The interal of a non-negative measurable function from a measure space $(X, \mathcal{M}, \mu)$ has the following properties \begin{enumerate}[label=(\arabic*)]
    \item If $0 \le f \le g$, then $\int_ E f(x) \ dx \le \int_E g \ d \mu$
    \item If $A \subset B$, $A, B \in \mathcal{M}$, then $\int_A f \ d \mu \le \int_B f \ d \mu$
    \item If $c \in [0, \infty)$, $ E \in \mathcal{M}$, then $\int_E cf \ d \mu = c \int_E f \ d \mu$
    \item If $f = 0$, or $ \mu(E) = 0$, then $\int_E f \ d \mu = 0$
    \item For all $ E \in \mathcal{M}$, \[
        \int_E f \ d \mu = \int_X f \chi_{E} \ d \mu
    \]
  \end{enumerate}
\end{theorem}
\begin{proof}
  \begin{enumerate}[label=(\arabic*)]
    \item By definition \[
        \int f \ d \mu = \sup_{\substack{t \textrm{ is simple} \\ t \textrm{ is measurable} \\ 0 \le t \le f}} \int_E  t \ d \mu
    \]
      then the simple function  $t \le f$ is also $ t \le g$. Hence suping over simple functions under $g$, every simple function under $ f$ is included.
    \item Let $s = \sum_{i = 1}^{n} \alpha_i \chi_{A_i}$ be a simple function $0 \le s \le f$ with $\int s \ dx + \epsilon > \int f \ d \mu$.
      Using the inclusion $ A \subset B$, we get \begin{align*}
        \int_A s \ d \mu &= \sum_{n \in \mathbb{N}} \alpha_n
      \end{align*}

    \item Suppose $s = \sum_{j = 1}^{n} \alpha_j \chi_{A_j}$ is a simple function with disjoint $A_j$s. Then $s \chi_{E} = \sum_{j = 1}^{n} \alpha_j \chi_{A_j \cap E}$ is also simple (and measurable), and \[
        \int_E s \ dx = \sum_{j = 1}^{n} \alpha_j \mu(A_j \cap E) = \int  s \chi_{E} \ dx
    \]
      Hence the statement is true for simple measurable functions. Next, consider $f$ non-negative measurable, then for $\epsilon \ge 0$, we have a simple measurable function $s$ with $\int_E s \ d \mu + \epsilon > \int_E f \ d \mu$. Then by preceding part, \[
          \int s \chi_{E} \ d \mu + \epsilon > \int_E f \ d \mu
      \]
      Also $s \chi_E \le f \chi_E$. So \[
          \int f \chi_E \ d \mu + \epsilon \ge \sup_{t \textrm{ is simple}} \int s \chi_E \ d \mu + \epsilon > \int f \ d \mu
      \]
      Taking $\epsilon \to 0$ gives \[
          \int f \chi_E \ d \mu \ge \int_E f \ d \mu
      \]
      For the reverse inequaltiy, note that $f \chi_E \le f$, and use similar circus.
  \end{enumerate}
\end{proof}

\begin{theorem}[Monotone convergence theorem]
  Let $(X, \mathcal{M}, \mu)$ be a measure space, given a sequence $f_n: X \to [0, \infty]$ of measurable functions and they are monotone increasing, i.e for each $x \in X$, $0 \le f_1(x) \le f_2(x) \le \ldots $, then \[
       \lim_{n \to \infty} \int f_n \ d \mu = \int \lim_{n \to \infty} f_n \ d \mu
  \]
\end{theorem}
\begin{proof}
  Let $f = \lim_{n \to \infty} f_n$ be the pointwise limit. Then $ f$ is measurable. From $f_n \le f_{n+1}$, we get that \[
       \int f_n \ d \mu \le \int f_{n+1} \ d \mu
  \]
  so both sides of the claimed identity exist, and from $f_n \le f$, we also know that \[
       \int f_n \ d\mu \le \int f \ d \mu
  \]
  which taking the limits give us, \[
       \lim_{n \to \infty} \int f_n \ d\mu \le \int f \ d \mu
  \]

  Now let $s: X \to [0, \infty]$ be a simple measurable function $s \le f$. Choose $0 \le c < 1$, and define $ E_n = \{ x \in X \ : \ f_n(x) \ge cs(x) \} = (f_n - s)^{-1}([0, \infty])$. \textcolor{red}{Verify that difference between an extended real valued function and a real valued function is measurable, then $E_n$ is measurable}. This gives a nested sequence $E_1 \subset E_2 \subset \ldots $.
  If $ f(x) > 0$, then by $ f(x) > cs(x)$ and $  f_n(x) \to f(x)$, there is $n \in \mathbb{N}$ such that $x \in E_n$. 
  On the other hand if $f(x) = 0$, then $cs(x) = 0 = f(x)$, so $ x \in E_n$ for all $n \in \mathbb{N}$.
  We see that each $ x \in X$ is in the  union $\cup_{n = 1}^{\infty}E_n$. Hence $X = \cup_{n = 1}^{\infty}E_n$. Now we define $ \phi: \mathcal{M} \to [0, \infty]$ by \[
    \phi(E) = \int_E s \ d \mu
  \]
  which is a measure and $\phi(X) = \phi(\cup_{n = 1}^{\infty}E_n) =  \lim_{n \to \infty} \phi(E_n)$ by \autoref{thm:properties_of_integrals}. We rewrite this as \begin{align*}
    \int_X s \ d \mu &= \lim_{n \to \infty} \int_{E_n} s \ d \mu \\ 
    &= \lim_{n \to \infty} \int_X s \chi_{E_n} \ d \mu \\ 
    &\le \lim_{n \to \infty} \int_X \frac{1}{c} f_n \ d \mu
  \end{align*}
  Now take sup over all such simple (bounded) functions $s \le f$ and let $c \to 1$.
  \textcolor{red}{Finish this proof}.
\end{proof}

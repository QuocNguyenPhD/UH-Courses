% TeX_root = ../main.tex

\chapter{}

\begin{definition}
  A function $f$ of a topological space $X$ is called lower
  semi-continuous if for all $\alpha \in \mathbb{R}$, $\{ x \in X
  \ : \  f(x) > \alpha \}$ is open.
\end{definition}
\begin{example}
  If $V$ is open, then $\chi_V$ is lower semi-continuous because the
  $\{ x \in X  \ : \  f(x) > \alpha \}$ has choices $ \phi, V, X$,
  all of them are open.
\end{example}

\begin{definition}
  A function is called upper semi-continuous if for all $\alpha \in
  \mathbb{R}$, the set $\{ x \in X  \ : \  f(x) < \alpha \}$ is open.
\end{definition}

\begin{remark}
  If $f: X \to \mathbb{R}$ is lower semi-continuous, then $-f$ is
  upper semi-continuous.
\end{remark}

\begin{example}
  If $V$ is open, then $\chi_{V^c} = 1- \chi_{V}$ is upper semi-continuous.
\end{example}

\begin{proposition}
  If $f, g$ are lower semi-continuous, so is $f+g$.
\end{proposition}
\begin{proof}
  \begin{align*}
    \{ x \in X  \ : \  f(x)+g(x) > \alpha \} & = \bigcup_{r \in
    \mathbb{R}} \big( \{ x \ : \  f(x) > r \}  \cap  \{ x
    \ : \  g(x)< \alpha - r \}\big)
  \end{align*}
\end{proof}

\begin{proposition}
  If $u_1 \le u_2 \le \ldots$ are all lower semi-continuous, then so
  is $\lim_{n \to \infty} u_n = u$.
\end{proposition}
\begin{proof}
  \[
    \{ u > \alpha \} = \bigcup_{n \in \mathbb{N}} \{ u_\alpha > \alpha \}
  \]
\end{proof}

\begin{corollary}
  A monotone increasing sequence of continuous functions converges to
  a lower semi-continuous function.
\end{corollary}

\begin{theorem}[Vitali-Caratheodory Theorem]
  Let $X$ be locally compact and Hausdorff, $\mu$ be a regular Borel
  measure. If $f: X \to \mathbb{R}$ in $L^1(\mu)$, then there is an
  upper semi-continuous function $u$ and a lower semi-continuous
  function $v$ such that $u \le f \le v$ and $\int(v-u) \ d \mu < \epsilon$.
\end{theorem}
\begin{proof}
  Assume $f \ge 0$. There exists an increasing sequence of simple functions
  $(s_n)$ converging (pointwise) to $f$. Considering  as before,
  $t_n = s_n - s_{n-1}$ with $s_0 = 0$, we see that each $t_n$ is
  simple and $f = \sum_{n \in \mathbb{N}} t_n$.

  Then since of the $t_n$ are simple, expanding them out into the
  standard simple function form and re-indexing them, we get  \[
    f = \sum_{j = 1}^{\infty}  c_j \chi_{E_j}
  \]
  Note that we're not claiming $E_j$s are disjoint. Since $ f \in
  L^1(\mu)$, we can apply monotone convergence theorem. Thus \[
    \sum_{j = 1}^{\infty} \underbrace{ \int c_j \chi_{E_j} \ d \mu
    }_{c_j \mu(E_j)} = \int  f \ d \mu < \infty
  \]
  If $c_j = 0$, discard. Otherwise we see that $\mu(E_j) < \infty$
  for each $j \in \mathbb{N}$. By regularity, $\exists K_j$ compact
  and $ V_j$ open such that $K_j \subset E_j \subset V_j$ and $
  \mu(V_j \setminus K_j) < \frac{\epsilon}{2^jc_j}$. As a consequence
  of convergence of $ \sum_{j = 1}^{\infty}  c_j \mu(E_j)$, we have
  $N \in \mathbb{N}$ such that $ \sum_{j = N+1}^{\infty}  c_j
  \mu(E_j) < \epsilon$. Let \[
    u = \sum_{j = 1}^{N} c_j \chi_{K_j} \quad \textrm{and} \quad v =
    \sum_{j = 1}^{\infty} c_j \chi_{V_j}
  \]
  Then we see that $u$ is upper semi-continuous and $v$ is lower
  semi-continuous and  \[
    v - u = \sum_{j = 1}^{N} c_j \chi_{V_j \setminus K_j} + \sum_{ j
    = N+1}^{\infty}  c_j \chi_{V_j}
  \]
  Thus,
  \begin{align*}
    \int  (v- u) \ d \mu &= \int \big( \sum_{j = 1}^{N} c_j \chi_{V_j
  \setminus K_j} + \sum_{ j = N+1}^{\infty}  c_j \chi_{V_j}\big)) \ d \mu \\
  &= \sum_{j = 1}^{N} c_j \mu(V_j\setminus K_j) + \sum_{ j =
  N+1}^{\infty}  c_j \mu(V_j) \\
  & \le \sum_{j = 1}^{N} c_j \frac{\epsilon}{2^jc_j} +    \\
  & < \epsilon +
\end{align*}

Now to complete the proof, apply this result to $f^+$ and $f^-$. Then
since $f = f^+ - f^-$ and we get upper and lower semi-continuous
functions $u_+, v_+$ for $f^+$ and $u_{-}, v_-$ for $f^-$. Let $u =
u_+ - v_-, v = v_+ - u_-$ gives $u \le f \le v$ and satisfy the properties.
\end{proof}


